\section{Einleitung}
\label{section:intro}

Bei vielen Anwendungen in der wissenschaftlichen Programmierung sind übersichtlicher, handlicher Code und eine schnell mögliche 
Implementierung, für die man auf viele verschiedene, bereits vorhandene, gute Bibliotheken zugreifen kann, wichtiger als 
Schnelligkeit und optimierter Speicherverbrauch. 

Wer sehr viel Wert auf die letzten beiden Eigenschaften legt, ist zum Beispiel mit \CC als Programmiersprache der Wahl gut beraten 
(siehe auch die erste Hälfte dieses sehr guten Kurses \cite{joelixC}). In den meisten Fällen bringt einen \Python jedoch schneller ans 
Ziel.

Eine der Hauptmotivationen von \Python ist es, eine besonders übersichtliche, gut lesbare und einfache Programmiersprache zu sein. 
Die Syntax ist deshalb sehr reduziert und \Python kommt mit wenigen Schlüsselwörtern aus. Die überschaubare Standardbibliothek ist leicht 
zu erweitern -- tatsächlich ist einer der größten Vorteile von \Python die große Fülle an bereits existierenden Modulen und Bibliotheken,
die einem als Nutzer sehr viel Zeit bei der Programmierung ersparen können (mehr dazu kann in \textbf{Abschnitt \ref{section:module}}
nachgelesen werden).

Diese Erweiterbarkeit von \Python sorgt auch dafür, dass Nachteile (wie zum Beispiel die im Vergleich zu maschienennäheren Sprachen 
eher langsame Performance) ausgeglichen werden können. Beispielsweise können performancekritische Routinen in \C implementiert 
und in \Python eingebunden werden um die Vorteile beider Sprachen verbinden zu können (zusammengefasst in \textbf{Tabelle 
\ref{tabelle:effizienz}}). 

\begin{table}[ht]
\centering
 \begin{tabular}{|c|c|c|}
   \hline
                               & Code schnell schreiben & Schnellen Code schreiben\\\hline
   \Python                     &           +            &           -             \\
   \CC                         &           -            &           +             \\
   \Python + \C = $\heartsuit$ &           +            &           +             \\
   \hline
\end{tabular}
\caption{Die awesome Effizienzmatrix}
\label{tabelle:effizienz}
\end{table}

\Python ist keine kompilierte, sondern eine interpretierte Sprache. Der aus \CC bekannte Ablauf Code schreiben -- kompilieren -- 
ausführen entfällt also in der Form. Stattdessen wird der Programmcode dem sogenannten \textbf{Interpreter} übergeben. Hierbei hat man 
zwei Möglichkeiten zur Auswahl: entweder man schreibt den Code direkt in den Interpreter, der die einzelnen Codeblöcke daraufhin sofort 
ausführt, oder man übergibt ihm den Code gebündelt in einer Datei (man hat zudem noch die Möglichkeit, diese Datei direkt ausführbar 
zu machen). Mehr zur Nutzung und Installation von \Python findet sich in \textbf{Anhang \ref{section:installation}}. 

Diese Vorlesung und das begleitende Skript haben zum Ziel, in die Programmiersprache \PythonDrei einzuführen. Dabei richten wir uns 
vor allem an Programmiererinnen, die bereits (fortgeschrittene) Erfahrungen in \CC haben. Es ist jedoch auch möglich, die Vorlesung 
ohne weitreichende Programmierkenntnisse zu besuchen. Der Kurs richtet sich an Bachelorstudenten der Mathematik, die begleitenden 
Übungen sind daher meist mathematisch motiviert.

Aufgrund der knappen Zeit besprechen wir in diesem Skript nur einige Grundlagen von \Python.
Der interessierten Leserin legen wir den \Python-Standard \cite{Python2} und \cite{Python3} aufgrund der hohen Präzision sehr ans Herz.
Das Buchprojekt \cite{LottPython}, welches sich ``nur'' auf \PythonZwei bezieht, ist ebenso (vor allem für weniger erfahrene 
Programmiererinnen) zu empfehlen, da es ausführlich ist, viele gute Beispiele enthält und kaum Programmierkenntnisse voraussetzt.

\vspace{1em}

Dieses Skript ist wie folgt aufgebaut:

In \textbf{Abschnitt \ref{section:datamodel}} erklären wir den wesentlichen konzeptionellen Unterschied zwischen \Python und 
\CC, das \Python-Datenmodel, bevor wir in \textbf{Abschnitt \ref{section:crashkurs}} die grundlegenden 
Bausteine von \Python zusammenfassen. Dieser Crashkurs hat vor allem zum Ziel, Besonderheiten der Syntax aufzuzeigen um bereits 
erfahreneren Programmiererinnen den Einstieg in \Python zu erleichtern. Grundlegende Konzepte wie Bedingungen, Schleifen, Ausgabe 
und Funktionen werden nicht konzeptionell wiederholt.

Die Datentypen, mit denen man in \Python am häufigsten arbeitet, werden in \textbf{Abschnitt \ref{section:std_data_types}} behandelt.
Diese umfassen unter anderem Zahlen, Sequenzen, Mengen, Verzeichnisse und den Umgang mit Dateien. Zu jedem Datentyp listen wir die 
wichtigsten Operationen und Funktionen auf.

Einige Erklärungen zum Namespace befinden sich in \textbf{Abschnitt \ref{section:namespacing}}. Diese sind notwendig um den Umgang 
mit Variablen in den folgenden Abschnitten besser verstehen zu können.

\textbf{Abschnitt \ref{section:funktionen}} widmet sich der Syntax von Funktionen in \Python. Neben Funktionsdefinitionen und -aufrufen 
gibt es noch ein paar Bemerkungen zur Erstellung einer \lpy{main}-Funktion in \Python, sowie den Vorteilen die sich ergeben, wenn 
man Funktionen als Objekte auffasst.

Neben Funktionen sind Klassen eine weitere Möglichkeit, Code übersichtlicher zu gestalten und einfacher nutzbar zu machen. Genauere 
Erklärung zur Syntax, Aufbau und Nutzung von Klassen finden sich in \textbf{Abschnitt \ref{section:klassen}}.  

Ein wichtiger Aspekt des Programmierens ist die Fehlerbehandlung. Diese wird in \Python durch Ausnahmebehandlung vereinfacht, die in 
\textbf{Abschnitt \ref{section:ausnahmen}} ausführlich besprochen wird.

Wie bereits in der Einleitung erwähnt ist einer der größten Vorteile von \Python die riesige Vielfalt an bereits vorhandenen Modulen.
Einen Überblick über der nützlichsten Module die der \Python Standard für Mathematiker bereit hält sowie Anmerkungen zur Nutzung von Modulen 
finden sich in \textbf{Abschnitt \ref{section:module}}.

Zum Abschluss, in \textbf{Abschnitt \ref{section:python_mit_c}} gehen wir noch auf die Möglichkeit ein, bereits vorhandenen 
\CC-Code und sogar bereits kompilierte Bibliotheken mit \Python zu verbinden und somit das beste aus beiden Welten zu vereinen.

In \textbf{Anhang \ref{section:installation}} finden sich zum Schluss noch einige Hinweise zur Installation von 
\Python unter verschiedenen Betriebssystemen.

Das Skript ist Freie Software und verfügbar unter:
\begin{center}
  \url{https://github.com/felixboes/joelix_python}
\end{center}
