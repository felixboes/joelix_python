\section{Einschub: Namespacing}
\label{section:namespacing}
Wie wir bereits wissen, organisieren wir unseren \Python-Code in Codeblöcken.
Immer wenn wir eine neues \Python-Skript beginnt, zählen wir das als eigenen Block;
Immer wenn wir Code einrücken ist dieser Code ein eigener Block.

In so gut wie allen Blöcken definieren wir viele Variablen, zum Beispiel Klassennamen, Objekte und Funktionen.
Es ist unvermeidbar, dass wir Variabelnamen mehrfach vergeben.
Zum Beispiel haben viele Module eine Funktion mit dem Namen \lpy{main}, die nur dann aufgerufen wird,
wenn das Modul das main-Modul ist (vergleiche Abschnitte \ref{section:funktionen:wo_ist_die_main_funktion} und \ref{section:module}).
Bei der Organisation der Variablnamen stellt sich \Python klüger (wenn auch nicht notwendigerweise effizienter) als andere Sprachen an.
In \Python werden Variablen zuerst lokal gesucht und gesetzt und erst dann global.
Insbesondere haben wir nicht ganz viele globale Variablen, die alle \lpy{main} heißen.

Genauer legt \Python die Struktur eines gerichteten Baums an.
Dabei wird jeder Modul- Klassen und Funtktionsblock als Knoten verstanden und es verläuft genau dann einen gerichteten Weg von \lpy{A} nach \lpy{B},
wenn der Codeblock \lpy{B} ein Unterblock vom Codeblock \lpy{A} ist.

Wird eine Variable in einem Codeblock \lpy{A} definiert, so liegt sie am entsprechenden Knoten im Baum.
Wir nennen die Variable ``lokal bezüglich \lpy{A}''.
Wenn wir zum Beispiel in einem Modul \lpy{modul} zwei Funktionen \lpy{f} und \lpy{g} definieren,
so sind \lpy{f} und \lpy{g} lokal bezüglich \lpy{modul}, aber weder \lpy{f} ist lokal bezüglich \lpy{g} noch anders herum.

Wenn eine Variable \lpy{x} im Codeblock \lpy{A} ausgeweret werden soll, sucht \Python den Namen \lpy{x} erstmal im Codeblock \lpy{A}.
Die Variable \lpy{x} kann in \lpy{A} definiert sein, muss sie aber nicht.
Zum Beispiel wollen wir die Möglichkeit haben, Funktionen aufzurufen, die nicht in \lpy{A} definiert sind.
Wurde die Variable gefunden, so nennen wir die Variable ``lokal bezüglich \lpy{A}''.
Wenn sie nicht in \lpy{A} gefunden wurde, durchsucht \Python den genannten Baum skuzessive aufwärts.
Wurde die Variable nicht in \lpy{A}, aber in einem Knoten über \lpy{A} gefunden, so nennen wir die Variable ``global bezüglich \lpy{A}''.

Falls die Variable nicht gefunden werden kann, wird eine Ausnahme ausgelöst (vergleiche Abschnitt \ref{section:ausnahmen}).
