\subsection{Eigene Module in Python bereitstellen}
\label{section:module:module_selbst_schreiben}
Um in \Python Bibliotheken zur Verfügung zu stellen, packt man (ähnlich wie in \CC) seinen gesamten Code in ein Modul.
Das kann zwar ein einziges \Python-Skript sein, aber so gut wie immer macht es viel mehr Sinn, seinen Code in mehreren Dateien (die auf mehrere Ordnern verteilt sind) zu organisieren.
Es ist gängige Praxis seine logischen Einheiten in einem gewurzelten Baum zu organisieren.
Jedes Blatt entspricht einer Datei und alle anderen Knoten sind Ordner, welche die darunter liegenden Knoten (also Ordner und Dateien) enthalten.
Außerdem ist es oft klug einen gewissen Anteil des Codes dem Benutzer nicht direkt zugänglich machen.
So kann man als Anbieter der Bibliothek kleine Details ändern, ohne dass sich der Benutzer der Bibliothek an neue Funktionen gewöhnen muss.

Wir besprechen anhand eines Beispiel die einfachsten Möglichkeiten ein eigenes Modul zu erstellen.
Es sei hier noch gesagt, dass Module auch Untermodule bereitstellen können.


\subsubsection{Skripte als Module einbinden}
Im aller einfachsten Fall ist ein Modul nichts anderes als eine \Python-Skript.
Das haben wir im vergangenen Abschnitt bereits gesehen.
Nehmen wir an, wir haben bereits eine \Python-Skript \lpy{prog.py} in einem Order \lpy{wurzel} erstellt und mit sinnvollem Code gefüllt.
Wenn wir nun ein weiteres \Python-Skript \lpy{hauptprog.py} im selben Ordner (also \lpy{wurzel}) anlegen, können wir \lpy{prog.py} mithilfe von \lpy{import} laden.
Beim Importieren wird die Dateiendung weggelassen:
\begin{lstlisting}
# Datei: hauptprog.py
import prog # Importiert prog.py
...
\end{lstlisting}
Beim importieren des \Python-Skripts \lpy{prog.py} wird dieses Zeile für Zeile gelesen, geprüft und ausgeführt.
Danach kann man auf die Funktionen, Klasses und Variablen mithilfe des Präfixes \lpy{prog.} zugreifen.
Also greifen wir auf die Variable \lpy{x} die in \lpy{prog.py} definiert wurde mit \lpy{prog.x} zu.

\subsubsection{Ordner als Module einbinden}
Im nächst einfachen Fall ist ein Modul nichts anderes als ein Order gefüllt mit \Python-Skripten.
Gehen wir davon aus, dass wir einen Ordner \lpy{skript_bib} haben, in dem die \Python-Skripte \lpy{skript_eins.py}, \lpy{skript_zwei.py} und \lpy{skript_drei.py} liegen haben.
Damit der \Python-Interpreter den Ordner \lpy{skript_bib} als Modul interpretieren kann, muss der Ordner ein \Python-Skript mit dem Namen \lpy{__init__.py} enthalten.
Dieses Skript wird beim Einbinden des Moduls Zeile für Zeile gelesen, geprüft und ausgeführt.
Alle anderen Skripte werden nicht automatisch interpretiert, denn das ist überlicherweise die Aufgabe von \lpy{__init__.py}.
In \lpy{__init__.py} bindet man dann entweder alle Klassen, Funktionen und Variablen der anderen Skripte ein.
Das geschieht mithilfe von
\begin{lstlisting}
from skript_eins.py import *
\end{lstlisting}
oder man bindet nur die benötigen Klassen, Funktionen und Variablen der anderen Skripte ein.
Das geschieht mithilfe von
\begin{lstlisting}
from skript_zwei.py import klasse_a, klasse_b, funktion_z
\end{lstlisting}
Sobald das Modul \lpy{skript_bib} importiert ist, kann man auf alle Klassen, Funktionen und Variablen, die dem Skript \lpy{__init__.py} bekannt sind, mithilfe des Präfixes \lpy{skript_bib.} zugreifen.

Wir besprechen das nun anhand eines Beispiels.
Angenommen wir wollen ein Modul \lpy{JoelixBlas} anfertigen, dass sich wie die Bibliothek \lcpp{joelixblas} verhält (siehe \cite{joelixC}).
Das heißt, wir wollen ein Modul, das Vektoren und (dünnbesetzte Matrizen) anbietet und elementare Operationen (wie zum Beispiel Matrix-Vektor-Multiplikation) bereitstellt.
Zunächst erstellen wir ein Verzeichnis mit dem Namen unseres Moduls, also \lpy{JoelixBlas}.
Die logischen Einheiten \lpy{JoelixVektor} und \lpy{JoelixMatrix} sind zu gewissen Teilen voneinander unabhängig und wir entscheiden uns dazu, sie in zwei separate Dateien zu lagern.
Unsere Verzeichnisstruktur sollte zu diesem Zeitpunkt so aussehen:
\begin{lstlisting}[language=plain]
JoelixBlas
  +--joelix_vektor.py
  +--joelix_matrix.py
\end{lstlisting}
Nun implementieren wir die Klassen \lpy{JoelixVektor} und \lpy{JoelixMatrix}.
Um die Klasse \lpy{JoelixVektor} zu implementieren, brauchen wir die Klasse \lpy{JoelixMatrix} nicht zu kennen.
Die Datei \lpy{joelix_vektor.py} sieht so aus:
\begin{lstlisting}
class JoelixVektor:
  """Vektorklasse der JoelixBlas."""
  ...
  def __add__(self, other): # Implementiert Vektor-Vektor-Addition
    """Implementiert Vektor-Vektor-Addition."""
    if type(other) is type(self):
      # Vektor-Vektor-Addition
  ...
\end{lstlisting}
Die Matrix-Matrix-Multiplikation und die Matrix-Vektor-Multiplikation soll durch eine spezielle Memberfunktion der Klasse \lpy{JoelixMatrix} realisiert werden.
Da wir die Klasse \lpy{JoelixVektor} für die Matrix-Vektor-Multiplikation benötigen, binden wir die Klasse \lpy{joelix_vektor} im Modul \lpy{joelix_matrix.py} ein:
Die Datei \lpy{joelix_matrix.py} sieht so aus:
\begin{lstlisting}
# JoelixMatrix.py
from joelix_vektor import JoelixVektor

class JoelixMatrix:
  """Duennbesetzte Matrix der JoelixBlas."""
  ...
  def __mul__(self, other): # Definiere Matrix-X-Multiplikation
    """Implementiert die Matrix-X-Multiplikation."""
    if type(other) is type(self):
      # Matrix-Matrix-Multiplikation
    elif type(other) is JoelixVektor:
      # Matrix-Vektor-Multiplikation
  ...
\end{lstlisting}
Damit wir den Ordner \lpy{JoelixBlas} als Modul einbinden können, braucht der Ordner \lpy{JoelixBlas} noch das \Python-Skript \lpy{__init__.py}.
In diesem importieren wir die Klasse \lpy{JoelixVektor} aus \lpy{joelix_vektor.py} und die Klasse \lpy{JoelixMatrix} aus der Datei \lpy{joelix_matrix.py}.
\begin{lstlisting}
# Datei: __init__.py
from joelix_vektor import JoelixVektor
# Jetzt kennt __init__.py die Klasse JoelixVektor
from joelix_matrix import JoelixMatrix
# Jetzt kennt __init__.py die Klasse JoelixMatrix
\end{lstlisting}
Unsere Verzeichnisstruktur sollte also so aussehen:
\begin{lstlisting}[language=plain]
JoelixBlas
  +--__init__.py
  +--joelix_vektor.py
  +--joelix_matrix.py
\end{lstlisting}

Jetzt können wir das Modul \lpy{JoelixBlax} in einem \Python-Programm verwenden.
Zum Beispiel erstellen wir ein \Python-Skript mit dem Namen \lpy{test.py} das imselben Order liegt, wie der Ordner \lpy{JoelixBlas}.
Unsere Verzeichnisstruktur sollte also so aussehen:
\begin{lstlisting}[language=plain]
mein_programm
  +--test.py
  +--JoelixBlas
       +--__init__.py
       +--joelix_vektor.py
       +--joelix_matrix.py
\end{lstlisting}
Das \Python-Skript \lpy{test.py} sieht so aus:
\begin{lstlisting}
import JoelixBlas

def main():
  Vek = JoelixBlas.JoelixVektor(dim=3)
  Vek[0] = 5.0
  Vek[1] = 4.6
  Vek[2] = 1.8
  print( Vek )

if __name__ == '__main__':
  main()
\end{lstlisting}
