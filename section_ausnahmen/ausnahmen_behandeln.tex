\subsection{Ausnahmen behandeln}
\label{section:ausnahmen:ausnahmen_behandeln}
Bevor wir erklären, wie eine Ausnahme behandelt wird, gehen wir kurz darauf ein, was eine Ausnahme ist:
Eine Ausnahme ist ein Objekt und somit hat sie einen festen Typ.
Der \Python-Standard definiert eine Hand voll Ausnahmetypen, wie zum Beispiel
den Ausnahmetyp \lpy{ZeroDivisionError} (der bei einer Division durch Null ausgelöst wird) oder
den Ausnahmetyp \lpy{SyntaxError} (den der \Python-Interpreter bei einem Syntaxfehler auslöst).
Der Wert einer Ausnahme \lpy{ausnahme} vom Ausnahmetyp \lpy{Ausnahmetyp} ist (für uns und nur hier) ein String und jede Ausnahme kann als String interpretiert werden.
Eine Liste der wichtigsten, bereits definierten Ausnahmetypen sind in Abschnitt \ref{section:ausnahmen:definierte_und_eigene_ausnahmen} zu finden.

Ein Programmabschnitt in dem Ausnahmen ausgelöst werden können, die wir (im Fall das Fälle) behandeln wollen,
wird der Programmabschnitt in ein \lpy{try-except} Konstrukt eingefasst.
Nach \lpy{try:} folgt unser (eingerückter) Programmabschnitt.
Dann werden die möglichen Ausnahmen behandelt.
Möchte oder muss man eine Ausnahme vom Ausnahmetyp \lpy{Ausnahmetyp} behandeln, geschieht das mit \lpy{except Ausnahmetyp:} gefolgt von der auszuführenden Ausnahmebehandlung.
Hier kann man auch mehrere Ausnahmentypen zusammenfassen mit \lpy{except (Ausnahmetyp_1, Ausnahmetyp_2, ...):}.
Alle anderen Ausnahmentypen sammelt man mit \lpy{except:}.
Das sieht dann zum Beispiel so aus:
\begin{lstlisting}
# Programmfluss

try:
  # Abschnitt der Ausnahmen ausloesen kann, die wir behandeln wollen
except Ausnahmetyp_1:
  # Ausnahmetyp 1 behandeln
except (Ausnahmetyp_2, Ausnahmetyp_3):
  # Ausnahmetyp 2 behandeln
# Hier noch mehr Ausnahmentypen die man behandeln moechte
except:
  # Alle anderen Ausnahmen behandeln

# Hier geht der normale Programmfluss weiter
\end{lstlisting}

Um auf eine Ausnahme des Ausnahmetyps \lpy{Ausnahmetyp} zugreifen zu können nutzt man \lpy{except Ausnahmetyp as ausnahme}.
Hier ein Beispiel, das die Ausnahme ``Division durch Null'' behandelt.
\begin{lstlisting}
try:
  a = 1/0
except ZeroDivisionError as ausnahme:
  print('Ausnahme:', ausnahme)
\end{lstlisting}
Man kann auch mehrere Ausnahmetypen mit \lpy{as} benennen.
\begin{lstlisting}
try:
  a = 1/0
except (ZeroDivisionError, ValueError) as ausnahme:
  print('Ausnahme vom Typ:', type(ausnahme), 'mit Wert:', ausnahme )
\end{lstlisting}

Es gibt Situationen, da möchte man einen Programmabschnitt ausführen der Ausnahmen auslösen kann und diese dann folgendermaßen behandeln.
Wird eine (behandelnare) Ausnahme ausgelöst, so soll sie behandelt werden.
Wird jedoch keine Ausnahme ausgelöst, dann (und nur dann) soll ein weiterer Programmabschnitt ausgeführt werden.
Das funktioniert mit der \lpy{try-except-else} Konstruktion:
\begin{lstlisting}
# Programmfluss

try:
  # Abschnitt A
except ...: # Zu behandelnden Ausnahmetyp festlegen
  # Ausnahmen behandeln
... # Weitere Ausnahmebehandlungen
else: # Wird ausgefuehrt genau dann wenn Absch. A keine Ausnahme ausloest
  # Abschnitt B

# Hier geht der normale Programmfluss weiter
\end{lstlisting}
Der obige Code verhält sich genau wie der nachfolgende Code:
\begin{lstlisting}
try:
  ausnahme_aufgetreten = False
  # Abschnitt A
except ...: # Zu behandelnden Ausnahmetyp festlegen
  ausnahme_aufgetreten = True
  # Ausnahmen behandeln
... # Weitere Ausnahmebehandlungen, die ausnahme_aufgetreten=True setzen
if ausnahme_aufgetreten == False:
  # Abschnitt B
\end{lstlisting}

An dieser Stelle kann man bereits verstehen, warum Ausnahmen ein gutes Konzept sind.
Durch die Aufteilung in einen von \lpy{try} eingeleiteten Block schreibt man den auszuführenden Programmcode und
teilt die Ausnahmebehandlung in die von \lpy{except} eingeleiteten Blöcken ein.
Das führt zu wesentlich übersichtlicherem Code.

Hier noch ein Beispiel:
\begin{lstlisting}[escapechar=|]
import math
def ganzzahlige_wurzel( x ):
  """Diese Funktion zieht die ganzzahlige Wurzel."""
  y = 0 # Wir definieren eine Variable y, die wir am Ende zurueckgeben,
        # unabhaengig davon, ob eine Ausnahme behandelt werden muss oder
        # nicht
  try:  # Versuche die ganzzahlige Wurzel zu siehen
    y = int(math.sqrt(x)) |\label{zeile:sqrt_loest_fehler_aus}|
  except TypeError as ausnahme:  # Ausnahme: x hat den falschen Typ
    print('Falscher Typ:', ausnahme)
  except ValueError as ausnahme: # Ausnahme: x ist negativ.
    print('Falscher Wert:', ausnahme)
  except:                        # Ausnahme: andere Ausnahme
    print('Anderer, komischer Fehler...')
  return y |\label{zeile:y_ist_evtl_nicht_definiert}|

ganzzahlige_wurzel('Suppe') # Druckt: 'Falscher Typ: a float is required'
ganzzahlige_wurzel(-3)      # Druckt: 'Falscher Wert: math domain error'
ganzzahlige_wurzel(6)       # Druckt nix.
\end{lstlisting}
Man beachte, dass die Zeile \lpy{y=0} nicht vergessen werden darf, denn sonst kann es passieren, dass \lpy{y} in Zeile~\ref{zeile:y_ist_evtl_nicht_definiert} nicht definiert ist.
Falls \lpy{math.sqrt(x)} eine Ausnahme auslöst, wird der Programmfluss in Zeile~\ref{zeile:sqrt_loest_fehler_aus} unterbrochen und die Ausnahme behandelt.
Dass heißt, beim Auslösen einer Ausnahme wird \lpy{y} in dieser Zeile weder definiert noch auf ein Objekt gesetzt und kann insbesondere in Zeile~\ref{zeile:y_ist_evtl_nicht_definiert} nicht zurückgegeben werden.
