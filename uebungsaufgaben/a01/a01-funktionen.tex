\begin{aufg}
  Schreibe eine Funktion die ein Argument \lpy{x} bekommt und folgenden Text ausdruckt, 
  wobei \lpy{'x'}, \lpy{'id'} und \lpy{'typ'} sinnvoll ersetzt werden.
  \begin{lstlisting}
"Das Argument x='x' hat die Identitaet 'id' und den Typ 'typ'."
  \end{lstlisting}

  Schreibe noch eine Funktion, die ein Argument \lpy{x} bekommt.
  Wenn \lpy{x} den Typ \lpy{int} hat, soll das Wort \lpy{'Ganzzahl'} zurückgegeben werden.
  Wenn \lpy{x} den Typ \lpy{str} hat, soll das Wort \lpy{'String'} und die Länge des Strings zurückgegeben werden
  (die Länge eines Strings \lpy{s} ist \lpy{len(s)}).
  Sonst sollen die Funktion zwei Strings zurückgeben nämlich deinen Vor- und deinen Nachnamen.
\end{aufg}
