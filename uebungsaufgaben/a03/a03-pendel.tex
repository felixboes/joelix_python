\begin{aufg_schw}
  Schreibe eine Klasse, die sich wie ein mathematisches Pendel verhält%
  \footnote{Die Bewegungsgleichung findest du auf \url{https://de.wikipedia.org/wiki/Mathematisches_Pendel}}:
  Das Pendel soll zu gegebenen Startwerten $\phi$, $g$ und $l$ erstellt werden können,
  wobei $\phi$ der Ausgangswinkel, $g$ die Erdanziehungskraft und $l$ die Fadenlänge ist.
  
  Nun wollen wir das Pendel in diskreten Zeitabschnitten $t \sim 0$ schwingen lassen.
  Dazu soll die Klasse über eine Memberfunktion \lpy{tick()} verfügen, welche die neue Position des Pendels berechnet (nachdem $t \sim 0$ Zeit vergangen ist).
\end{aufg_schw}
