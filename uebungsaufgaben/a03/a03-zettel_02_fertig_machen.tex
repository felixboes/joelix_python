\begin{aufg}
  Mache die Aufgaben zur Polynomklasse und zu den Wür\-fel\-au\-gen\-paaren vom gestrigen Tag fertig.
  Die restlichen Aufgaben sind nicht so wichtig. Die kannst du später bearbeiten.
  Dabei könntest du hier und da mal ins Skript schauen um dein Wissen zu vertiefen.
\end{aufg}

\begin{aufg}
  Da wir nun wissen wie man eine beliebige Anzahl von Argumenten übergibt, verändere den Polynomkonstruktor,
  so dass man ein Polynom vom Grad (kleiner oder gleich) $n$ durch Angabe von $n+1$ Koeffizienten erstellen kann.
  Zum Beispiel so:
  \begin{lstlisting}
p = Polynom(1,4,2) # p = x^2 + 4x^1 + 2x^0
  \end{lstlisting}
\end{aufg}

\begin{aufg}
  Lies im Skript ein wenig mehr über Ausnahmen und erweitere die Polynomklasse um Ausnahmen.
  Zum Beispiel soll im Konstruktor die Ausnahme \lpy{raise TypeError('Falscher Typ. Erwarte int oder float.')} vom Typ \lpy{TypeError} mit Wert \lpy{'Falscher Typ. Erwarte int oder float.'} ausgelöst werden,
  falls einer der übergebenen Parameter nicht vom Typ \lpy{int} oder \lpy{float} ist.
  
  Hierbei machen wir darauf aufmerksam wie man überprüft ob ein Objekt \lpy{x} eine Instanz der Klasse \lpy{kl} ist.
  Das geschieht mit \lpy{isinstance( x, kl )}.
  Zum Beispiel liefert \lpy{isinstance( 'hi', str )} den Wert \lpy{True} zurück.
\end{aufg}

