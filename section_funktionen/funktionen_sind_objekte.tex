\subsection{Funktionen sind auch nur Objekte}
\label{section:funktionen:funktionen_sind_objekte}
Wir wiederholen hier nochmal, dass Funktionen Objekte sind.
Insbesondere können sie Argumente einer anderen Funktion \lpy{F} sein.

Schauen wir uns ein einfaches Beispiel an.
Angenommen, wir haben eine aufwendige Funktion \lpy{aufwendige_funk}, die \lpy{k} Parameter braucht.
Daraus wollen wir eine neue Funktion bauen,
die genau dasselbe macht und genau dasselbe zurückgibt, aber noch zusätzlich die CPU-Zeit misst und ausdruckt.
Dazu schreiben wir eine neue Funktion \lpy{zeit_messen}, die als erstes Argument eine Funktion \lpy{f} bekommt
und als zweites Argument ein Verzeichniss \lpy{vargs} welches den Parametern der Funktion \lpy{f} entsprechen soll.
Innerhalb von \lpy{zeit_messen} nehmen wir erstmal die CPU-Zeit.
Dann rufen wir \lpy{f} mit den Parametern \lpy{vargs} auf und speichern die Rückgabe in \lpy{rueckgabe}.
Dann nehmen wir wieder die CPU-Zeit und drucken die Differenz aus.
Zu guter Letzt geben wir \lpy{rueckgabe} zurück.
Der Code sieht also so aus:
\begin{lstlisting}
import time

def aufwendige_funk(par_1, ... par_k):
  # Aufwendige Funktion

def zeit_messen( f, **vargs ):
  start = time.process_time ()
  rueckgabe = f( **vargs )
  dauer = time.process_time() - start
  print("Die Funktion '{}' hat {} Sekunden CPU-Zeit verbraucht".format(
    f.__name__, dauer))
  return rueckgabe

parameter = {'par_1' : ... }
zeit_messen(aufwendige_funk, parameter)
\end{lstlisting}

Ein ganz wesentlicher Unterschied zwischen \C und \Python ist, dass man in \Python beliebige Funktionen zur Laufzeit erstellen kann.
Das ist in \C nicht möglich und in \CPP erst ab \CPPElf.
Schauen wir uns ein eifnaches Beispiel an.
Aus der Mathematik kennen wir die Ableitungsfunktion:
\[
  \frac{d}{dt} \colon C^\infty(\mathbb R) \to C^\infty( \mathbb R) \quad\quad \frac{df}{dt}(x) = \lim_{t \to 0} \frac{f(x+t) - f(x)}{(x+t)-x}
\]
Diese nimmt eine glatte Funktion $f$ und definiert eine neue Funktion $df/dt$.
Das können wir in \Python ganz genauso machen:
\begin{lstlisting}
def ableiten(f, dt=1e-4):
  dt = float(dt)        # Interpretiert dt als float
  def df_dt(x):         # Definiert df/dt
    df = f(x+dt) - f(x)
    return df / dt
  return df_dt          # Gibt df/dt zurueck

def g(t):               # Definiert g(x) = x*x/2
  return t*t/2.0

h = ableiten(g)         # Approx. die Ableitung von g, also h(x) ~ x
\end{lstlisting}
Unsere Funktion \lpy{ableiten} dient hier nur zur Demonstration.
Für numerische Berechnungen taugt sie nichts, da wir uns nicht um Auslöschung kümmern.

Für den Fall, dass wir eine Funktion definieren möchten, die nur aus einem einzigen Ausdruck besteht (der dann zurückgegeben wird), können wir folgende Abkürzung benutzen.
Genauer, gehen wir davon aus, dass wir folgende Funktion haben:
\begin{lstlisting}
def f(par_1, ..., par_k):
  return ausdruck
\end{lstlisting}
Dann ist diese Funktionsdefinition äquivalent zu:
\begin{lstlisting}
f = lambda par_1, ..., par_k : ausdruck
\end{lstlisting}
So definierte Funktionen nennt man ``Lambdafunktionen''.
Man soll sie dann und nur dann benutzen, wenn der Code dadurch leserlicher wird.
Unser Ableitungsbeispiel sieht mit Lambdafunktionen so aus:
\begin{lstlisting}
def ableiten(f, dt=1e-4):
  dt = float(dt)        # Interpretiert dt als float
  return lambda x : (f(x+dt)-f(x))/dt

def g(t):               # Definiert g(x) = x*x/2
  return t*t/2.0

h = ableiten(g)         # Approx. die Ableitung von g, also h(x) ~ x
\end{lstlisting}

