\subsection{Funktionen definieren und aufrufen}
\label{section:funktionen:funktionen_definieren_und_aufrufen}

Funktionsdefinitionen in \Python haben folgende einfache Syntax:

\begin{lstlisting}
def funktionsname(par_1, ..., par_k, rap_1=wert_1, ..., rap_l=wert_l):
  ausdruck
  ...
\end{lstlisting}

Im Crashkurs, Abschnitt \ref{section:crashkurs:funktionen}, haben wir bereits eine vereinfachte Version der Funktionsdefinition gesehen und auch, dass man 
Parametern Standardwerte zuweisen kann. Dies ist die allgemeine Form. Wir betonen noch einmal (weil es nicht häufig genug betont 
werden kann), dass die Funktionsparameter keinen fest zugewiesenen Datentyp haben, auch nicht die, die einen Standardwert zugewiesen
bekommen haben.

Wie immer gilt, dass man eine gute Programmiererin daran erkennt, dass die Funktionen gut auskommentiert und dokumentiert sind.
Die sinnvollste Art, eine Funktion (oder auch ein anderes Objekt) in Python zu dokumentieren ist der sogenannte Doc-String, der 
direkt unter dem Funktionskopf steht und von jeweils drei doppelten Anführungszeichen pro Seite eingerahmt wird:

\begin{lstlisting}
def funktionsname(par_1, ..., par_k, rap_1=wert_1, ..., rap_l=wert_l):
  """ Ausfuehrliche Dokumentation """
  ausdruck
  ...
\end{lstlisting}

Im Doc-String sollten - falls vorhanden - Eingabeparameter und Rückgabewerte der Funktion erklärt werden, sowie eventuelle Ausgaben.

Der Vorteil des Doc-Strings im Vergleich zu konventionellen Kommenaren ist, dass \Python in der Lage ist, ihn mit Hilfe der 
\lpy{help}-Funktion auszugeben. Dazu ruft man \lpy{help(funktionsname)} auf, woraufhin der Interpreter einen Editor öffnet 
und den Funktionskopf zusammen mit dem Doc-String der Funktion anzeigt. Dieses Feature ist vor allem nützlich, wenn man direkt im 
Interpreter programmiert und sich daran erinnern möchte, wie die Funktion zu benutzen ist. 

Ein kleines Beispiel für eine Funktion, die keine Parameter bekommt und nichts (also immer \lpy{None}) zurückgibt, aber deshalb 
nicht weniger dokumentiert werden sollte:

\begin{lstlisting}
import random
def moechtegern_mathematiker():
  """Approxomiert den Output eines moechtegern Mathematikers."""
  if( random.random() < 0.5 ):
    print("Aber das mit dem Einruecken ist doof!")
  else:
    print("Aber C hat eine bessere Laufzeit!")

help(moechtegern_mathematiker)
moechtegern_mathematiker()
\end{lstlisting}

Führt man diesen Code aus, so wird zunächst \lpy{help(moechtegern_mathematiker)} ausgeführt, d.h.\ es wird ein Editor 
geöffnet, in dem Funktionskopf und Doc-String der Funktion \lpy{moechtegern_mathematiker()} angezeigt werden.
Nachdem der Editor geschlossen wurde, wird die Funktion selbst ausgeführt. 

Es ist möglich, Funktionen eine variable Anzahl von Parametern zu übergeben. Ein Stern vor einem der Parameternamen 
signalisiert \Python, dass es hier eine nicht weiter spezifizierte Anzahl von Elementen zu erwarten hat. Diese Anzahl kann 
auch Null sein. Man kann es sich so vorstellen, dass eine Liste von Argumenten erwartet wird, die natürlich auch leer sein 
kann. 

\begin{lstlisting}
def funktionsname(parameter, *rest_als_liste):
\end{lstlisting}

Der \lpy{*}-Operator kann auch bei einem Funktionsaufruf verwendet werden. In diesem Fall schreibt man ihn vor eine 
Sequenz (meistens vor einen Variablennamen) und signalisiert damit, dass man die Sequenz gerne ``entpacken'' möchte, 
also ihre einzelnen Einträge der Funktion als Parameter übergeben möchte.

Das folgende Beispiel zeigt beide Nutzvarianten des \lpy{*}-Operators in Funktionsdefinitionen und -aufrufen:

\begin{lstlisting}
import random
def moechtegern_mathematiker( *gejammer ):
  """Approxomiert den Output des lieblings moechtegern Mathematikers.
     gejammer ist das lieblings Gejammer."""
  print( random.choice(gejammer) )

moechtegern_mathematiker("Aber C ist besser!", "Aber C++ ist besser!")

lieblings_gejammer = (
  "Aber das mit dem Einruecken ist doof!",
  "Aber C hat eine bessere Laufzeit!",
  "Aber wir koennen doch schon C++!",
  "Aber wenn man reine Mathematik macht, braucht man keine Computer!",
  "Aber im Internet steht, dass Python doof ist!",
  "gcc Warnungen kann ich ignorieren, ich weiss ja was ich tue!",
  "Aber ich kann doch schon Python, nur was ist dieses Objektorientiert?!"
)

moechtegern_mathematiker( *lieblings_gejammer )
\end{lstlisting}

Die Funktion \lpy{moechtegern_mathematiker} erwartet eine beliebige Anzahl an Parametern, bzw. sie erwartet als Parameter eine 
entpackte Sequenz mit einer beliebigen Anzahl Einträge, die unter \lpy{gejammer} zusammengefasst sind. Aus diesen wählt sie einen 
Eintrag zufällig aus und gibt ihn auf der Standardausgabe aus.

Im ersten Aufruf der Funktion werden ihr zwei Strings als Parameter übergeben. Für den zweiten Funktionsaufruf wird zunächst die 
Liste \lpy{lieblings_gejammer} definiert, die eine größere Anzahl an Strings enthält. Danach rufen wir die Funktion mit 
\lpy{*lieblings_gejammer} auf, d.h.\ die Einträge der Liste werden entpackt und der Funktion als Argumente übergeben.

Man kann den \lpy{*}-Operator auch zum Entpacken von Sequenzen verwenden, wenn die Funktion eine feste Anzahl von Parametern 
erwartet. Natürlich muss man dann aufpassen, dass die Länge der Sequenz mit der Anzahl der erwarteten Parameter übereinstimmt.

Es ist auch möglich, eine beliebige Anzahl von Schlüssel-Parameter-Paaren (also ein entpacktes Verzeichnis) zu übergeben. In der 
Funktionsdefinition wird dies mit \lpy{**} vor dem Parameternamen angekündigt:

\begin{lstlisting}
def funktionsname(parameter, **rest_als_verzeichnis):
\end{lstlisting}

Wie auch mit Sequenzen (s.o.) hat man jetzt zwei Möglichkeiten, eine solche Funktion aufzurufen, nämlich einmal indem man die 
Parameter direkt übergibt oder indem man ein Verzeichnis mit dem \lpy{**}-Operator entpackt. Das folgende Beispiel 
zeigt dies etwas genauer:

\begin{lstlisting}
import random
def moechtegern_mathematiker( **gejammer ):
  """Approxomiert den Output des lieblings moechtegern Mathematikers.
     gejammer ist das lieblings Gejammer."""
  eigenschaft, adjektiv = random.choice(list(gejammer.items()))
  print( "Aber " + eigenschaft + " ist " + adjektiv + "!" )

moechtegern_mathematiker(C="besser", Python="doof")

lieblings_gejammer = {
  "Einruecken" : "doof",
  "Laufzeit" : "doof",
  "angewante Mathematik" : "auch doof",
  "Freizeit" : "ganz doof",
  "Algebra" : "besser"
}

moechtegern_mathematiker( **lieblings_gejammer )
\end{lstlisting}

Wenn man die Parameter direkt übergibt, erfolgt die Zuordung Schlüssel-Parameter nicht über den Doppelpunkt wie wenn man ein 
Verzeichnis erstellt, sondern über ein Gleichheitszeichen.

Wie auch bei Sequenzen ist es möglich, ein entpacktes Verzeichnis an eine Funktion zu übergeben, die eine feste Anzahl an 
Argumenten erwartet. In diesem Fall muss man nicht nur darauf aufpassen, dass die Anzahl der Paare des Verzeichnisses mit 
der Anzahl der erwarteten Argumente der Funktion übereinstimmt, sondern auch darauf, dass die Schlüssel im Verzeichnis mit 
den Parameternamen übereinstimmen.