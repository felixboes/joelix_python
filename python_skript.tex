\documentclass[a4paper]{scrartcl}

\usepackage{amsfonts}
\usepackage{amsmath}
\usepackage{amssymb}
\usepackage{amsthm}
\usepackage{array}
\usepackage[german]{babel}
\usepackage{bm}
\usepackage[pdftex]{color}
\usepackage{enumitem}
\usepackage[utf8]{inputenc}
\usepackage{listings}
\usepackage{tikz}
\usepackage{tikz-cd}
\usepackage{mathrsfs}
\usepackage{mathtools}
\usepackage{url}
\usepackage{verbatim}
\usepackage{xspace}

\usepackage{hyperref}
\usepackage[all]{hypcap}        % Convenience: Let hyperref jump to the figure instead to the caption only.

% Common theorem environments
\theoremstyle{plain}
\newtheorem{theorem}{Theorem}[section]
\newtheorem{theoremdefinition}[theorem]{Theorem/Definition}
\newtheorem{proposition}[theorem]{Proposition}
\newtheorem{propositiondefinition}[theorem]{Proposition/Definition}
\newtheorem{lemma}[theorem]{Lemma}
\newtheorem{Korollar}[theorem]{Korollar}
\newtheorem{fakt}[theorem]{Fakt}

\theoremstyle{definition}
\newtheorem{definition}[theorem]{Definition}
\newtheorem{beispiel}[theorem]{Beispiel}

\theoremstyle{remark}
\newtheorem{bemerkung}[theorem]{Bemerkung}
\newtheorem{notation}[theorem]{Notation}
\newtheorem{motivation}[theorem]{Motivation}

% Include pictures made in inkscape:
\newcommand{\inputhack}{\input}                                         % Rename \input, so that a very simple awk-skript can merge all tex-files.
\graphicspath{{pictures/}}                                              % Set path of the pictures. This is used by the pdf_tex-files generated by inkscape.
\newcommand{\inkpic}[2][\columnwidth]{
  \def\svgwidth{#1}
  \inputhack{pictures/#2.pdf_tex}
}

% Git version
\newcommand{\gitversion}{eb9f7c92437845759961b547671529a0a35e1a02}


% Authors and Title
\author{Clelia Albrecht, Felix Boes, Johannes Holke}
\title{Python Skript}
\date{\small Git Version: \gitversion}

% Personal Macros
\input{skript_macros_felix}
\input{skript_macros_code}

%
% Begin actual document.
%
\begin{document}

\maketitle
\thispagestyle{empty}

\newpage

\thispagestyle{empty}
\null{}\vfill

\begin{center}
  \begin{minipage}{.7\columnwidth}
   Für Mama und Papa
  \end{minipage}
  
  \vspace{3ex}
  
  \hspace{10ex}\begin{minipage}{.7\columnwidth}
   Für Mamma und Papa
  \end{minipage}
  
  \vspace{3ex}
  
  \hspace{4ex}\begin{minipage}{.7\columnwidth}
   Für Mama und Papa
  \end{minipage}
\end{center}
\vfill

\newpage

\thispagestyle{empty}
\null{}\vfill
\begin{center}
  \begin{minipage}{.7\columnwidth}
    Python ist ja ganz nett, aber ist C nicht besser und überhaupt was soll dieses Einrücken?
    
    \vspace{5ex}

    \raggedleft
    \emph{Guido von Rossum}
  \end{minipage}
\end{center}
\vfill


\newpage

\tableofcontents

\newpage

\section{Einleitung}
\label{section:intro}

Bei vielen Anwendungen in der wissenschaftlichen Programmierung sind übersichtlicher, handlicher Code und eine schnell mögliche 
Implementierung, für die man auf viele verschiedene, bereits vorhandene, gute Bibliotheken zugreifen kann, wichtiger als 
Schnelligkeit und optimierter Speicherverbrauch. 

Wer sehr viel Wert auf die letzten beiden Eigenschaften legt, ist zum Beispiel mit \CC als Programmiersprache der Wahl gut beraten 
(siehe auch die erste Hälfte dieses sehr guten Kurses \cite{joelixC}). In den meisten Fällen bringt einen \Python jedoch schneller ans 
Ziel.

Eine der Hauptmotivationen von \Python ist es, eine besonders übersichtliche, gut lesbare und einfache Programmiersprache zu sein. 
Die Syntax ist deshalb sehr reduziert und \Python kommt mit wenigen Schlüsselwörtern aus. Die überschaubare Standardbibliothek ist leicht 
zu erweitern -- tatsächlich ist einer der größten Vorteile von \Python die große Fülle an bereits existierenden Modulen und Bibliotheken,
die einem als Nutzer sehr viel Zeit bei der Programmierung ersparen können (mehr dazu kann in \textbf{Abschnitt \ref{section:module}}
nachgelesen werden).

Diese Erweiterbarkeit von \Python sorgt auch dafür, dass Nachteile (wie zum Beispiel die im Vergleich zu maschienennäheren Sprachen 
eher langsame Performance) ausgeglichen werden können. Beispielsweise können performancekritische Routinen in \C implementiert 
und in \Python eingebunden werden um die Vorteile beider Sprachen verbinden zu können (zusammengefasst in \textbf{Tabelle 
\ref{tabelle:effizienz}}). 

\begin{table}[ht]
\centering
 \begin{tabular}{|c|c|c|}
   \hline
                               & Code schnell schreiben & Schnellen Code schreiben\\\hline
   \Python                     &           +            &           -             \\
   \CC                         &           -            &           +             \\
   \Python + \C = $\heartsuit$ &           +            &           +             \\
   \hline
\end{tabular}
\caption{Die awesome Effizienzmatrix}
\label{tabelle:effizienz}
\end{table}

\Python ist keine kompilierte, sondern eine interpretierte Sprache. Der aus \CC bekannte Ablauf Code schreiben -- kompilieren -- 
ausführen entfällt also in der Form. Stattdessen wird der Programmcode dem sogenannten \textbf{Interpreter} übergeben. Hierbei hat man 
zwei Möglichkeiten zur Auswahl: entweder man schreibt den Code direkt in den Interpreter, der die einzelnen Codeblöcke daraufhin sofort 
ausführt, oder man übergibt ihm den Code gebündelt in einer Datei (man hat zudem noch die Möglichkeit, diese Datei direkt ausführbar 
zu machen). Mehr zur Nutzung und Installation von \Python findet sich in \textbf{Anhang \ref{section:installation}}. 

Diese Vorlesung und das begleitende Skript haben zum Ziel, in die Programmiersprache \PythonDrei einzuführen. Dabei richten wir uns 
vor allem an Programmiererinnen, die bereits (fortgeschrittene) Erfahrungen in \CC haben. Es ist jedoch auch möglich, die Vorlesung 
ohne weitreichende Programmierkenntnisse zu besuchen. Der Kurs richtet sich an Bachelorstudenten der Mathematik, die begleitenden 
Übungen sind daher meist mathematisch motiviert.

Aufgrund der knappen Zeit besprechen wir in diesem Skript nur einige Grundlagen von \Python.
Der interessierten Leserin legen wir den \Python-Standard \cite{Python2} und \cite{Python3} aufgrund der hohen Präzision sehr ans Herz.
Das Buchprojekt \cite{LottPython}, welches sich ``nur'' auf \PythonZwei bezieht, ist ebenso (vor allem für weniger erfahrene 
Programmiererinnen) zu empfehlen, da es ausführlich ist, viele gute Beispiele enthält und kaum Programmierkenntnisse voraussetzt.

\vspace{1em}

Dieses Skript ist wie folgt aufgebaut:

In \textbf{Abschnitt \ref{section:datamodel}} erklären wir den wesentlichen konzeptionellen Unterschied zwischen \Python und 
\CC, das \Python-Datenmodel, bevor wir in \textbf{Abschnitt \ref{section:crashkurs}} die grundlegenden 
Bausteine von \Python zusammenfassen. Dieser Crashkurs hat vor allem zum Ziel, Besonderheiten der Syntax aufzuzeigen um bereits 
erfahreneren Programmiererinnen den Einstieg in \Python zu erleichtern. Grundlegende Konzepte wie Bedingungen, Schleifen, Ausgabe 
und Funktionen werden nicht konzeptionell wiederholt.

Die Datentypen, mit denen man in \Python am häufigsten arbeitet, werden in \textbf{Abschnitt \ref{section:std_data_types}} behandelt.
Diese umfassen unter anderem Zahlen, Sequenzen, Mengen, Verzeichnisse und den Umgang mit Dateien. Zu jedem Datentyp listen wir die 
wichtigsten Operationen und Funktionen auf.

Einige Erklärungen zum Namespace befinden sich in \textbf{Abschnitt \ref{section:namespacing}}. Diese sind notwendig um den Umgang 
mit Variablen in den folgenden Abschnitten besser verstehen zu können.

\textbf{Abschnitt \ref{section:funktionen}} widmet sich der Syntax von Funktionen in \Python. Neben Funktionsdefinitionen und -aufrufen 
gibt es noch ein paar Bemerkungen zur Erstellung einer \lpy{main}-Funktion in \Python, sowie den Vorteilen die sich ergeben, wenn 
man Funktionen als Objekte auffasst.

Neben Funktionen sind Klassen eine weitere Möglichkeit, Code übersichtlicher zu gestalten und einfacher nutzbar zu machen. Genauere 
Erklärung zur Syntax, Aufbau und Nutzung von Klassen finden sich in \textbf{Abschnitt \ref{section:klassen}}.  

Ein wichtiger Aspekt des Programmierens ist die Fehlerbehandlung. Diese wird in \Python durch Ausnahmebehandlung vereinfacht, die in 
\textbf{Abschnitt \ref{section:ausnahmen}} ausführlich besprochen wird.

Wie bereits in der Einleitung erwähnt ist einer der größten Vorteile von \Python die riesige Vielfalt an bereits vorhandenen Modulen.
Einen Überblick über der nützlichsten Module die der \Python Standard für Mathematiker bereit hält sowie Anmerkungen zur Nutzung von Modulen 
finden sich in \textbf{Abschnitt \ref{section:module}}.

Zum Abschluss, in \textbf{Abschnitt \ref{section:python_mit_c}} gehen wir noch auf die Möglichkeit ein, bereits vorhandenen 
\CC-Code und sogar bereits kompilierte Bibliotheken mit \Python zu verbinden und somit das beste aus beiden Welten zu vereinen.

In \textbf{Anhang \ref{section:installation}} finden sich zum Schluss noch einige Hinweise zur Installation von 
\Python unter verschiedenen Betriebssystemen.

Das Skript ist Freie Software und verfügbar unter:
\begin{center}
  \url{https://github.com/felixboes/joelix_python}
\end{center}


\newpage

\section{Das Python Datenmodell}
\label{section:datamodel}
Ein wesentlicher Unterschied zwischen \CC und \Python ist das Verhalten von Daten im Speicher und den Zugriff darauf.
Um den Unterschied zu \CC klarer zu machen, beginnen wir mit einer kurzen Wiederholung.


\subsection{Wiederholung: Daten in C und C++}
\label{section:datamodel:cc}
In \CC gibt es Variablen.
Diese bestehen aus einer ``Speicheradresse'', einem ``Typ'' und dem dort gespeicherten ``Wert''.
Sie verhalten sich also wie ein zusammenhängendes Stück Speicher, indem ein Wert gespeichert wird und wir wissen, wie wir diesen Wert zu interpretieren haben.
Nehmen wir beispielsweise folgende Codezeile.
\begin{lstlisting}[style=CPP]
int32_t a = 0;
int32_t b = 3;
a = b;
\end{lstlisting}
In der ersten Zeile erstellen wir eine Variable vom Typ \lcpp{int32_t}, d.h.\ wir haben nun ein zusammenhängendes Stück Speicher indem der Wert einer ganzen 32-bit-Zahl gespeichert werden kann,
und legen dort den Wert der Expression \lcpp{0} ab.
In der zweiten Zeile erstellen wir eine weitere Variable vom Typ \lcpp{int32_t} und legen dort den Wert der Expression \lcpp{3} ab.
In der dritten Zeile legen wir den Wert der Expression \lcpp{b} in den zu \lcpp{a} gehörigen Speicher ab.
Nun ist der in \lcpp{a} gespeicherte Wert \lcpp{3} und der in \lcpp{b} gespeicherte Wert ebenfalls \lcpp{3}.
Wenn \lcpp{b} stattdessen vom Typ \lcpp{double *} wäre, führte Zeile 3 zu einem Compilerfehler.


\subsection{Daten in Python}
\label{section:datamodel:python}
Das \Python Datenmodell ist intelektuell nicht so einfach zu fassen wir das Datenmodell von \CC.
Es ist zwar keine Magie, dennoch kann es einige Tage oder gar Wochen dauern kann, bis man sich daran gewöhnt hat.

In \Python gibt es ``Objekte'' und ``Variablen''.
Objekte repräsentieren Daten und Variablen referenzieren auf Objekte.
Zunächst besprechen wir Objekte genauer und kümmer uns im Anschluss um Variablen.


\subsubsection{Objekte}
\label{section:datamodel:python:objekte}
Objekte bestehen aus einer ``Identität'', einem ``Typ'' und einem ``Wert''.
Sie verhalten sich wie ein zusammenhängendes Stück Speicher, in dem ein Wert gespeichert wird,
also ganz genauso wie sich Variablen in \CC verhalten.
Die Identität eines (existierenden) Objekts ändert sich nie und bestimmt ein Objekt (während der Laufzeit) eindeutig.
Wenn man mag, kann man sich die Identität wie die Speicheradresse des Objekts vorstellen (auch wenn man in \Python keine Pointer verwendet).

Ein jedes Objekt ist entweder ``mutable'' und kann nach seiner Erzeugung verändert werden oder es ist ``immutable'' und ist nach seiner Erzeugung konstant.
Dieses Konzept gibt es ebenfalls in \CC:
Alle Typen in \CC sind ohne weiteres ``mutable'', es sei denn, sie bekommen das Präfix \lcpp{const} bei ihrer Definition, denn dann sind sie ``immutable''.

In \Python sind folgende Objekte immutable:
die ganze Zahl \lpy{3} (siehe Abschnitt \ref{section:std_data_types:zahlen}),
das Tupel \lpy{(1, 'a')} (siehe Abschnitt \ref{section:std_data_types:sequenzen}),
der String \lpy{'Johannes'} (siehe Abschnitt \ref{section:std_data_types:sequenzen:str}) und
der leere Typ \lpy{None} (siehe Abschnitt \ref{section:std_data_types:none}).

Die folgenden Objekte sind mutable:
Die Liste \lpy{[1, 'a']} (siehe Abschnitt \ref{section:std_data_types:sequenzen}),
das Verzeichnis \lstinline[style=Pyinline]|{ 'Felix' : 405, 'Jonathan' : 599  }| (siehe Abschnitt \ref{section:std_data_types:verzeichnisse})
und die meisten Klassen (siehe Abschnitt \ref{section:klassen}).


\subsubsection{Variablen}
\label{section:datamodel:python:variablen}
In \Python haben Variablen keinen Typ.
Wir wiederholen nochmal: In \Python haben Variablen keinen Typ.
Und ja, in der Tat, wir wiederholen nochmal: In \Python haben Variablen keinen Typ.
Variablen referenzieren auf Objekte; sie sind sozusagen die (temporären) Namen von Objekten.
Wenn man mag, kann man sich Variablen wie \lcpp{(void)}-Pointer vorstellen (auch wenn man in \Python keine Pointer verwendet).

Variablen beziehen sich immer auf ein Objekt.
Beim Erstellen einer Variable muss ihr konsequenterweise direkt ein Objekt zugewiesen werden.
Betrachten wir ein einfaches Beispiel.
\begin{lstlisting}
a = 'Hallo' # Erstelle a; a referenziert das Objekt 'Hallo'.
b = 'Welt'  # Erstelle b; b referenziert das Objekt 'Welt'.
a = b       # a bezieht sich nun auf das Objekt "Welt".
b = 3.141   # b bezieht sich nun auf das Objekt 3.141
\end{lstlisting}
In der ersten Zeile erstellen wir eine Variable \lpy{a}, die sich auf das Objekt \lpy{'Hallo'} bezieht.
Das Objekt \lpy{'Hallo'} hat beispielsweise die Identität \lpy{1000}.
In der zweiten Zeile erstellen wir eine Variable \lpy{b} die sich auf das Objekt \lpy{'Welt'} bezieht.
Das Objekt \lpy{'Welt'} hat beispielsweise die Identität \lpy{2000}.
In der dritten Zeile weisen wir \lpy{a} die Referenz auf das Objekt zu, auf das sich \lpy{b} bezieht.
In der vierten Zeile weisen wir \lpy{b} die Referenz auf das Objekt \lpy{3.141} zu.
Das Objekt \lpy{3.141} hat beispielsweise die Identität \lpy{3000}.
Danach bezieht sich \lpy{a} auf \lpy{'Welt'} und \lpy{b} auf \lpy{3.141}.

\subsection{Speichermanagement}
Die meisten Objekte können genau solange verwendet werden, wie es Variablen gibt, die auf sie referenzieren.
(Einige Ausnahmen sind beispielsweise Zahlen und das Objekt \lpy{None}.)
Objekte, die nicht mehr verwendet werden können, werden automatisch dem ``Garbage Collector'' übergeben.
Dieser entfernt die Objekte nach eigenem Bedarf aus dem Speicher.
Das ist in den meisten Fällen bequemer als in \CC, wo man seinen Speicher selbst verwalten muss.
In der Zeit wo der \CC Programmierer seinen Speicher aufräumt, trinken wir lieber GLUMP%
\footnote{GLUMP Alkoholcola vereint das erweckende Prickeln von koffeinhaltiger Cola mit der betäubenden Wirkung von Alkohol.}.


\newpage

\section{Crashkurs}
\label{section:crashkurs}
Wir beginnen mit einem Überblick der grundlegende Bausteine.
In den nachfolgenden Kapiteln besprechen wir einige der Bausteine nochmal im Detail.


\subsection{Kommentare}
\label{section:crashkurs:kommentare}
Kommentare in \Python beginnen mit \lpy{#} und enden am Ende der Zeile.
Sie verhalten sich also ganz genau wie Kommentare in \CC, die mit \lcpp{//} beginnen%
\footnote{Die Kommentare \lcpp{//} wurden in \CNeunundneunzig eingeführt, siehe \cite{C99}.}.
Gut platzierte Kommentare dürfen in keinem Programm fehlen.
Ein schlecht kommentiertes Programm ist ein Anzeichen dafür, dass es von einem schlechten Programmierer geschrieben wurde.


\subsection{Print}
\label{section:crashkurs:print}
So gut wie alle Objekte können durch die \Python Funktion \lpy{print} auf der Standardausgabe ausgegeben werden.
Das liegt daran, dass die meisten Objekte Klassen sind und die Funktion \lpy{__str__} implementieren (auch wenn wir das jetzt noch nicht verstehen;
siehe Abschnitt \ref{section:klassen:spezielle_funktionen}).
Die Funktion \lpy{print} ist für \PythonZwei und \PythonDrei verschieden.
\begin{lstlisting}
print 'Hallo', 'Welt'   # Python2
print ('Hallo', 'Welt') # Python3
\end{lstlisting}
In diesem Skript nutzen wir die von \PythonDrei bereitgestellte Version von \lpy{print}.
Will man in \PythonZwei die \lpy{print}-Funktion aus \PythonDrei verwenden, kann man folgende Zeile am Anfang seines Skript einfügen.
\begin{lstlisting}
from __future__ import print_function # Nutze Python3-print in Python2
\end{lstlisting}


\subsection{Typ und Identität}
\label{section:crashkurs:typ_und_id}
Um die Identiät eines Objekts \lpy{x} zu bestimmen, nutzen wir die Funktion \lpy{id(x)}.
Sie gibt einen String zurück, den wir mit \lpy{print} ausdrucken können.
Ganz ähnlich bekommen wir den Typ eines Objekts mit \lpy{type(x)}.


\subsection{Zahlen}
\label{section:crashkurs:zahlen}
In \Python gibt es die Zahlentypen \lpy{int}, \lpy{long}, \lpy{float} und \lpy{complex}.
Sie verhalten sich (fast) genauso wie die aus \CC bekannten Typen \lcpp{int}, \lcpp{long}, \lcpp{float} und \lcpp{complex}%
\footnote{Der Typ \lpy{complex} wurde in \CNeunundneunzig eingeführt, siehe \cite{C99}.}.
Man erstellt ein Objekt vom Typ \lpy{int}, durch Angabe einer ganzen Zahl oder durch Aufrufen der Funktion \lpy{int}.
Ähnlich erstellt man ein Objekt vom Typ \lpy{float}, durch Angabe einer Kommazahl oder durch Aufrufen der Funktion \lpy{float}.
\begin{lstlisting}
print( 2 )        # 2    typ: int
print( int(2.1) ) # 2    typ: int
print( 2.0 )      # 2.0  typ: float
print( float(2) ) # 2.0  typ: float
\end{lstlisting}
Man kann zwei verschiedene oder gleiche Zahlentypen mit mathematischen Operatoren verbinden.
Das Ergebnis hat den ``genaueren'' Typ.
Des Weiteren beschreibt \lpy{//} die ganzzahlige Division ohne Rest.
\begin{lstlisting}
print( 2 + 3.3 )  # 5.3  typ: int +  float -> float
print( 5 // 2 )   # 2    typ: int // int   -> int
print( 5.3 // 2 ) # 2.0  typ: int // float -> float
\end{lstlisting}
Die Division von Ganzzahlen unterscheidet sich in \PythonZwei und \PythonDrei.
\begin{lstlisting}
4 /  2 # Python2: 2 vom Typ int;   Python3: 2.0 vom Typ float;
4 // 2 # Python2: 2 vom Typ int;   Python3: 2   vom Typ int;
\end{lstlisting}


\subsection{Strings}
\label{section:crashkurs:strings}
Strings die eine Zeile lang sind, werden entweder mit \lpy{'} oder mit \lstinline[style=PyInline]|"| begonnen und beendet.
Soll ein String mehre Zeilen lang sein, kann man ihn mit \lstinline[style=PyInline]|"""| beginnen und beenden.
Genau wie in \CC gibt es gewisse Zeichen, die nicht direkt geschrieben werden können und mit \lstinline[language=C++,style=CPP]|\| beginnen müssen (zum Beispiel werden neue Zeilen durch \lstinline[language=C++,style=CPP]|\n| ausgedrückt).
Folgender Code erstellt den String ``Hallo[Neue Zeile]Welt''.
\begin{lstlisting}
print( 'Hallo\nWelt' ) # Hallo[neue Zeile]Welt
\end{lstlisting}
Es gibt eine große Auswahl an Funktionen, um aus einem gegebenen String einen neuen zu erstellen.
Beispielsweise verknüpft man zwei Strings mit \lpy{+} und die Funktion \lpy{lower} ersetzt alle Großbuchstaben durch Kleinbuchstaben.
\begin{lstlisting}
print( 'Hallo' + ' ' + 'Welt' ) # Hallo Welt
print( 'Hallo Welt'.lower() )   # hallo welt
\end{lstlisting}


\subsection{Bedingungen}
\label{section:crashkurs:bedingungen}
In \Python gibt es genau zwei Objekte \lpy{True} und \lpy{False} von Typ \lpy{bool}.
(Fast) alle Typen können mit der Funktion \lpy{bool} in einen der beiden Objekte umgewandelt werden.
Den Kontrollfluss des Programms steuert man im einfachsten Fall wir folgt.
\begin{lstlisting}
if bedingung:
  ausdruck
  ...
\end{lstlisting}
Das heißt, wir brauchen zunächst eine Expression \lpy{bedingung}, die als \lpy{True} oder \lpy{False} interpretiert werden kann.
Wertet sie als \lpy{True} aus, so wird ein Block von Ausdrücken ausgeführt.
Anders als in \CC werden in \Python Blöcke nicht durch Klammern, sondern durch konsistente Einrückung definiert.
\begin{lstlisting}
x = 101//3+46//5
if x == 42:
  print('Die Antwort')
\end{lstlisting}
Da gut eingerückter Code (also Code, in dem jeder Block konsistent eingerückt ist) deutlich lesbarer sind als nicht eingerückter
Code, ist er in jedem Fall -- unabhängig von der Sprache -- ein Zeichen für eine gute Programmiererin. 

Da es jedoch trotzdem immer noch genug Leute gibt, die hässlichen Code produzieren, wird man in \Python dazu gezwungen, auf 
konsistente Einrückung zu achten: zu einem Block gehören genau die nachfolgenden Zeilen, die genauso weit eingerückt sind wie die 
erste Zeile des Blocks. Bei nicht konsistenter Einrückung gibt es Syntaxfehler. Dies macht es sehr schwer, hässlichen Code in
\Python zu schreiben.

Aus eigener Erfahrung können wir sagen, dass man sich als erfahrene \CC Programmiererin recht schnell an diese Konvention gewöhnen kann, 
wenn man auch schon vorher auf sauber geschriebene Programme geachtet hat.

Optional besteht eine Bedingung aus keinem oder mehreren \lpy{elif bedingung:}\footnote{sprich ``else if''} und keinem oder einem \lpy{else:}
Die Konstrukte \lpy{else:} und \lpy{elif bedingung:} in \Python verhalten sich wie die Konstrukte \lcpp{else} und \lcpp{else if} in \CC.
\begin{lstlisting}
x = 101//3+46//5               # x = ?
if x == 42:                    # Falls x den Wert 42 hat:
  print('Die Antwort')         # Drucke 'Die Antwort'
elif x == 84:                  # Sonst, falls x den Wert 84 hat:
  print('Zweimal die Antwort') # Drucke 'Zweimal die Antwort'
elif x == 106:                 # Sonst, falls x den Wert 106 hat:
  print('Dreimal die Antwort') # Drucke 'Dreimal die Antwort'
else:                          # In allen anderen Faellen:
  print('Versuchs nochmal')    # Drucke 'Versuchs nochmal'
\end{lstlisting}


\subsection{Listen}
\label{section:crashkurs:listen}
In \Python gibt es verschiedene ``Containertypen'' und hier besprechen wir den Typ \lpy{list}.
Eine Liste \lpy{thor} ist eine linear geordnete Menge von Variablen, auf die man mit \lpy{thor[0], thor[1], ..., thor[i]} zugreift.
Um auf Elemente von hinten zuzugreifen, sind negative Werte für \lpy{i} erlaubt.
\begin{lstlisting}
thor = [2, 'Hallo', 33.3, 'Welt', [] ] # Erzeugt eine Liste
print( thor[1], thor[-1] )             # Gibt folgendes aus: Hallo []
\end{lstlisting}
Die Länge einer Liste ist \lpy{len(thor)}.
Listen sind mutable, das heißt wir können sie verändern.
Mit \lpy{thor.append(x)} erweitert man die Liste \lpy{thor} um die Variable \lpy{x} und mit \lpy{thor.remove(x)} entfernt man die erste Variable mit dem Wert \lpy{x}.
Ob ein Objekt mit Wert \lpy{x} in \lpy{thor} enthalten ist, testet man mit \lpy{x in thor}.
Mit \lpy{thor+asgard} verknüpft man die beiden Listen \lpy{thor} und \lpy{asgard}.


\subsection{Schleifen}
\label{section:crashkurs:schleifen}
In \Python gibt es zwei Arten von Schleifen.
Die \lpy{while}-Schleife führt einen Codeblock aus, solange die vor der Ausführung des Blocks geprüfte Bedingung als \lpy{True} auswertet.
\begin{lstlisting}
while bedingung:
  ausdruck
  ...
\end{lstlisting}
Da uns dieses Konzept aus \CC sehr vertraut ist, brauchen wir hier kein Beispiel.

Die \lpy{for}-Schleife führt einen Codeblock für jedes Element einer gegebenen Sequenz (siehe Abschnitt \ref{section:std_data_types:sequenzen}) aus.
Eine Sequenz kann zum Beispiel eine Liste oder ein String sein.
\begin{lstlisting}
for i in s: # der Ausdruck s muss als Sequenz interpretiert werden koennen
  ausdruck
  ...
\end{lstlisting}
Dabei referenziert die Laufvariable \lpy{i} jedes Element der Sequenz \lpy{s} nacheinander einmal. Durch Verwendung von \lpy{i} innerhalb 
des folgenden Codeblocks kann 
dann auf den Wert des jeweiligen Objekts zugegriffen werden. Nach Ausführung des Codeblocks der \lpy{for}-Schleife referenziert 
\lpy{i} dann das nächste Element von \lpy{s}, bis das Ende der Sequenz erreicht ist. 

Dies wird durch das folgende Beispiel veranschaulicht:
\begin{lstlisting}
s = 'Hallo Welt' # Die Sequenz 'Hallo Welt'
for i in s:      # Nimm alle i = H,a,l,l,o, ,W,e,l,t
  print(i)       # Drucke i aus
\end{lstlisting}
Hier ist unsere Sequenz \lpy{s} ein String mit dem Wert \lpy{Hallo Welt}. In der \lpy{for}-Schleife nimmt die Laufvariable nacheinander
die Werte der einzelnen Zeichen von \lpy{s}, also \lpy{H,a,l,} \lpy{l,o, ,W,e,l} und \lpy{t}. Im zur Schleife gehörenden Block werden 
diese Zeichen dann einzeln ausgegeben.

Wenn der Codeblock für eine aufsteigende Folge von Zahlen \lpy{from, from+1, ..., to-1} ausgeführt werden soll, nimmt man die Sequenz \lpy{range(from, to)}.
Wichtig ist, dass wir das halboffene Interval betrachten, also nur bis \lpy{to-1} und nicht bis \lpy{to} gehen.
Damit verhält sich der \Python-Code \lpy{for i in range(from, to):} ganz genau wie der \CC-Code \lstinline[language=C++,style=CPPinline]|for(i = from; i < to; ++i) { ... }|.
Hier ein Beispiel
\begin{lstlisting}
sum = 0              # Setze sum auf 0
for i in range(1,5): # Nimm alle i = 1, 2, 3, 4
  sum += 3*i         # Addiere 3*i zu sum
print( sum )         # Drucke sum aus
\end{lstlisting}

Will man eine absteigende Folge ganzer Zahlen, oder allgemeiner eine Folge ganzer Zahlen mit Schrittweite \lpy{step > 0} oder \lpy{step < 0} haben,
nimmt man die Sequenz \lpy{range(from,to,step)}.
Also können wir obiges Beispiel wie folgt umformulieren.
\begin{lstlisting}
sum = 0                  # Setze sum auf 0
for i in range(3,15, 3): # Nimm alle i = 3, 6, 9, 12
  sum += i               # Addiere i zu sum
print( sum )             # Drucke sum aus
\end{lstlisting}


\subsection{Funktionen}
\label{section:crashkurs:funktionen}
Meistens will man sich wiederholende Codeblöcke auslagern und natürlich kann man auch in \Python Funktionen definieren.
Jede Funktionsdefinition erstellt ein Objekt, welches der Funktion im Speicher entspricht.
Außerdem wird eine Variable definiert, welche auf die Funktion im Speicher referenziert.
\begin{lstlisting}
def funktionsname(parameter):
  ausdruck
  ...
\end{lstlisting}
Die Variable ist hier \lpy{funktionsname} und zeigt auf das Objekt, welches der definierten Funktion entspricht.

Anders als in \CC hat eine Funktion keine Signatur und keinen definierbaren Rückgabetyp.
Mit \lpy{return} können kein, ein oder mehrere Objekte zurückgegeben werden.
Wird kein Objekt zurückgegeben, so ist die Rückgabe automatisch \lpy{None}.
Wird ein Objekt \lpy{x} zurückgegeben, so ist der Rückgabetyp automatisch \lpy{type(x)}.
Werden mehrere Objekte zurückgegeben, so werden diese automatisch in einem \lpy{tuple} zusammengefasst, der Rückgabetyp ist also \lpy{tuple}.
Ein Tuple ist eine Liste, die nicht geändert werden kann (siehe Abschnitt \ref{section:std_data_types:sequenzen}).
\begin{lstlisting}
def f(i):
  if i == 1:
    return 1           # Rueckgabetyp: int,   Rueckgabewert: 1
  elif i == 2:
    return 'Hallo', 44 # Rueckgabetyp: tuple, Rueckgabewert: ('Hallo', 44)

print( type( f ) )                 # <class 'function'>
print( type( f(1) ) )              # <class 'int'>
print( type( f(2) ) )              # <class 'tuple'>
print( type( f('Gurkenwasser') ) ) # <class 'NoneType'>
\end{lstlisting}
Also ist das von \lpy{f(3)} oder auch \lpy{f('Gurkenwasser')} zurückgegebene Objekt \lpy{None}.

Beim Funktionsausruf übergibt man die Parameter entweder in der Reihenfolge wie sie in der Funktionsdefinition spezifiziert sind, oder man übergibt \lpy{parametername=ausdruck}.
Außerdem kann man den Argumenten einer Funktion Standardwerte übergeben.
Beim Funktionsausruf muss man die Argumente mit Standardwerten nicht angeben, kann es aber.
\begin{lstlisting}
def plus(a,b=1):
  return a+b

plus(4,5)      # = 9
plus(b=5, a=4) # = 9
plus(3)        # = 4
\end{lstlisting}


\subsection{Module}
\label{section:crashkurs:module}
Was \Python wirklich mächtig macht, ist nicht die Sprache an sich, sondern die schier unendliche Fülle an Paketen die \Python-Programmierer bereit stellen.
Ein Modul ist so etwas wie eine ``shared library'' in \CC.
Man kann sie importieren und dann nutzen, die Implementierung haben andere bereits übernommen (siehe Abbildung \ref{figure:xkcd_python}).
\begin{figure}[ht]
  \centering
  \includegraphics[width=0.5\textwidth]{pictures/xkcd_python.png}
  \caption{\label{figure:xkcd_python}``Python'' by Randall Munroe \cite{Munroe_python}}
\end{figure}

Ein Modul bindet man mit \lpy{import modulname} in sein Programm ein.
Alle Objekttypen (Klassen, Funktionen und andere Objekte) die \lpy{modulname} bereitstellt, können durch \lpy{modulname.oname} erreicht werden.
\begin{lstlisting}
import math
print( math.sin(math.pi/3.0) )
\end{lstlisting}
Wenn man aus \lpy{modulname} bestimmte Objekte einbinden möchte, nutzt man folgende Codezeile.
\begin{lstlisting}
from modulname import oname1, oname2, ...
\end{lstlisting}
Jetzt kann man auf die Objekte direkt (also ohne das Präfix \lpy{modulname.}) zugreifen.
Wir hatten bereits gesehen, wie man die \PythonDrei Printfunktion in \PythonZwei einbinden kann.
\begin{lstlisting}
from __future__ import print_function
\end{lstlisting}
Hier noch ein Beispiel.
\begin{lstlisting}
from math import sin, cos, pi
print( sin(pi/3.0), cos(pi/3.0) )
\end{lstlisting}

An dieser Stelle machen wir noch auf Abschnitt \ref{section:module:empfohlene_module} indem wir eine Hand voll \Python-Pakete erwähnen, die wir oft benutzen.


\subsection{Workflow}
\label{section:crashkurs:workflow}
Wir arbeiten recht erfolgreich mit folgendem Workflow.

Beginne ein Projekt in \Python.
Solange du noch nicht fertig bist:
Wenn eine Funktion zu aufwendig zu implementieren ist, suche nach einer bereits vorhandene \Python Bibliothek (es gibt bestimmt eine).
Wenn eine Funktion für dein Projekt zu langsam ist, schreibe sie in \CC und binde sie als Modul ein.
Wie genau letzteres geht, lernen wir noch.


\newpage

\section{Crashkurs}
\label{section:crashkurs}
Wir beginnen mit einem Überblick der grundlegende Bausteine.
In den nachfolgenden Kapiteln besprechen wir einige der Bausteine nochmal im Detail.


\subsection{Kommentare}
\label{section:crashkurs:kommentare}
Kommentare in \Python beginnen mit \lpy{#} und enden am Ende der Zeile.
Sie verhalten sich also ganz genau wie Kommentare in \CC, die mit \lcpp{//} beginnen%
\footnote{Die Kommentare \lcpp{//} wurden in \CNeunundneunzig eingeführt, siehe \cite{C99}.}.
Gut platzierte Kommentare dürfen in keinem Programm fehlen.
Ein schlecht kommentiertes Programm ist ein Anzeichen dafür, dass es von einem schlechten Programmierer geschrieben wurde.


\subsection{Print}
\label{section:crashkurs:print}
So gut wie alle Objekte können durch die \Python Funktion \lpy{print} auf der Standardausgabe ausgegeben werden.
Das liegt daran, dass die meisten Objekte Klassen sind und die Funktion \lpy{__str__} implementieren (auch wenn wir das jetzt noch nicht verstehen;
siehe Abschnitt \ref{section:klassen:spezielle_funktionen}).
Die Funktion \lpy{print} ist für \PythonZwei und \PythonDrei verschieden.
\begin{lstlisting}
print 'Hallo', 'Welt'   # Python2
print ('Hallo', 'Welt') # Python3
\end{lstlisting}
In diesem Skript nutzen wir die von \PythonDrei bereitgestellte Version von \lpy{print}.
Will man in \PythonZwei die \lpy{print}-Funktion aus \PythonDrei verwenden, kann man folgende Zeile am Anfang seines Skript einfügen.
\begin{lstlisting}
from __future__ import print_function # Nutze Python3-print in Python2
\end{lstlisting}


\subsection{Typ und Identität}
\label{section:crashkurs:typ_und_id}
Um die Identiät eines Objekts \lpy{x} zu bestimmen, nutzen wir die Funktion \lpy{id(x)}.
Sie gibt einen String zurück, den wir mit \lpy{print} ausdrucken können.
Ganz ähnlich bekommen wir den Typ eines Objekts mit \lpy{type(x)}.


\subsection{Zahlen}
\label{section:crashkurs:zahlen}
In \Python gibt es die Zahlentypen \lpy{int}, \lpy{long}, \lpy{float} und \lpy{complex}.
Sie verhalten sich (fast) genauso wie die aus \CC bekannten Typen \lcpp{int}, \lcpp{long}, \lcpp{float} und \lcpp{complex}%
\footnote{Der Typ \lpy{complex} wurde in \CNeunundneunzig eingeführt, siehe \cite{C99}.}.
Man erstellt ein Objekt vom Typ \lpy{int}, durch Angabe einer ganzen Zahl oder durch Aufrufen der Funktion \lpy{int}.
Ähnlich erstellt man ein Objekt vom Typ \lpy{float}, durch Angabe einer Kommazahl oder durch Aufrufen der Funktion \lpy{float}.
\begin{lstlisting}
print( 2 )        # 2    typ: int
print( int(2.1) ) # 2    typ: int
print( 2.0 )      # 2.0  typ: float
print( float(2) ) # 2.0  typ: float
\end{lstlisting}
Man kann zwei verschiedene oder gleiche Zahlentypen mit mathematischen Operatoren verbinden.
Das Ergebnis hat den ``genaueren'' Typ.
Des Weiteren beschreibt \lpy{//} die ganzzahlige Division ohne Rest.
\begin{lstlisting}
print( 2 + 3.3 )  # 5.3  typ: int +  float -> float
print( 5 // 2 )   # 2    typ: int // int   -> int
print( 5.3 // 2 ) # 2.0  typ: int // float -> float
\end{lstlisting}
Die Division von Ganzzahlen unterscheidet sich in \PythonZwei und \PythonDrei.
\begin{lstlisting}
4 /  2 # Python2: 2 vom Typ int;   Python3: 2.0 vom Typ float;
4 // 2 # Python2: 2 vom Typ int;   Python3: 2   vom Typ int;
\end{lstlisting}


\subsection{Strings}
\label{section:crashkurs:strings}
Strings die eine Zeile lang sind, werden entweder mit \lpy{'} oder mit \lstinline[style=PyInline]|"| begonnen und beendet.
Soll ein String mehre Zeilen lang sein, kann man ihn mit \lstinline[style=PyInline]|"""| beginnen und beenden.
Genau wie in \CC gibt es gewisse Zeichen, die nicht direkt geschrieben werden können und mit \lstinline[language=C++,style=CPP]|\| beginnen müssen (zum Beispiel werden neue Zeilen durch \lstinline[language=C++,style=CPP]|\n| ausgedrückt).
Folgender Code erstellt den String ``Hallo[Neue Zeile]Welt''.
\begin{lstlisting}
print( 'Hallo\nWelt' ) # Hallo[neue Zeile]Welt
\end{lstlisting}
Es gibt eine große Auswahl an Funktionen, um aus einem gegebenen String einen neuen zu erstellen.
Beispielsweise verknüpft man zwei Strings mit \lpy{+} und die Funktion \lpy{lower} ersetzt alle Großbuchstaben durch Kleinbuchstaben.
\begin{lstlisting}
print( 'Hallo' + ' ' + 'Welt' ) # Hallo Welt
print( 'Hallo Welt'.lower() )   # hallo welt
\end{lstlisting}


\subsection{Bedingungen}
\label{section:crashkurs:bedingungen}
In \Python gibt es genau zwei Objekte \lpy{True} und \lpy{False} von Typ \lpy{bool}.
(Fast) alle Typen können mit der Funktion \lpy{bool} in einen der beiden Objekte umgewandelt werden.
Den Kontrollfluss des Programms steuert man im einfachsten Fall wir folgt.
\begin{lstlisting}
if bedingung:
  ausdruck
  ...
\end{lstlisting}
Das heißt, wir brauchen zunächst eine Expression \lpy{bedingung}, die als \lpy{True} oder \lpy{False} interpretiert werden kann.
Wertet sie als \lpy{True} aus, so wird ein Block von Ausdrücken ausgeführt.
Anders als in \CC werden in \Python Blöcke nicht durch Klammern, sondern durch konsistente Einrückung definiert.
\begin{lstlisting}
x = 101//3+46//5
if x == 42:
  print('Die Antwort')
\end{lstlisting}
Da gut eingerückter Code (also Code, in dem jeder Block konsistent eingerückt ist) deutlich lesbarer sind als nicht eingerückter
Code, ist er in jedem Fall -- unabhängig von der Sprache -- ein Zeichen für eine gute Programmiererin. 

Da es jedoch trotzdem immer noch genug Leute gibt, die hässlichen Code produzieren, wird man in \Python dazu gezwungen, auf 
konsistente Einrückung zu achten: zu einem Block gehören genau die nachfolgenden Zeilen, die genauso weit eingerückt sind wie die 
erste Zeile des Blocks. Bei nicht konsistenter Einrückung gibt es Syntaxfehler. Dies macht es sehr schwer, hässlichen Code in
\Python zu schreiben.

Aus eigener Erfahrung können wir sagen, dass man sich als erfahrene \CC Programmiererin recht schnell an diese Konvention gewöhnen kann, 
wenn man auch schon vorher auf sauber geschriebene Programme geachtet hat.

Optional besteht eine Bedingung aus keinem oder mehreren \lpy{elif bedingung:}\footnote{sprich ``else if''} und keinem oder einem \lpy{else:}
Die Konstrukte \lpy{else:} und \lpy{elif bedingung:} in \Python verhalten sich wie die Konstrukte \lcpp{else} und \lcpp{else if} in \CC.
\begin{lstlisting}
x = 101//3+46//5               # x = ?
if x == 42:                    # Falls x den Wert 42 hat:
  print('Die Antwort')         # Drucke 'Die Antwort'
elif x == 84:                  # Sonst, falls x den Wert 84 hat:
  print('Zweimal die Antwort') # Drucke 'Zweimal die Antwort'
elif x == 106:                 # Sonst, falls x den Wert 106 hat:
  print('Dreimal die Antwort') # Drucke 'Dreimal die Antwort'
else:                          # In allen anderen Faellen:
  print('Versuchs nochmal')    # Drucke 'Versuchs nochmal'
\end{lstlisting}


\subsection{Listen}
\label{section:crashkurs:listen}
In \Python gibt es verschiedene ``Containertypen'' und hier besprechen wir den Typ \lpy{list}.
Eine Liste \lpy{thor} ist eine linear geordnete Menge von Variablen, auf die man mit \lpy{thor[0], thor[1], ..., thor[i]} zugreift.
Um auf Elemente von hinten zuzugreifen, sind negative Werte für \lpy{i} erlaubt.
\begin{lstlisting}
thor = [2, 'Hallo', 33.3, 'Welt', [] ] # Erzeugt eine Liste
print( thor[1], thor[-1] )             # Gibt folgendes aus: Hallo []
\end{lstlisting}
Die Länge einer Liste ist \lpy{len(thor)}.
Listen sind mutable, das heißt wir können sie verändern.
Mit \lpy{thor.append(x)} erweitert man die Liste \lpy{thor} um die Variable \lpy{x} und mit \lpy{thor.remove(x)} entfernt man die erste Variable mit dem Wert \lpy{x}.
Ob ein Objekt mit Wert \lpy{x} in \lpy{thor} enthalten ist, testet man mit \lpy{x in thor}.
Mit \lpy{thor+asgard} verknüpft man die beiden Listen \lpy{thor} und \lpy{asgard}.


\subsection{Schleifen}
\label{section:crashkurs:schleifen}
In \Python gibt es zwei Arten von Schleifen.
Die \lpy{while}-Schleife führt einen Codeblock aus, solange die vor der Ausführung des Blocks geprüfte Bedingung als \lpy{True} auswertet.
\begin{lstlisting}
while bedingung:
  ausdruck
  ...
\end{lstlisting}
Da uns dieses Konzept aus \CC sehr vertraut ist, brauchen wir hier kein Beispiel.

Die \lpy{for}-Schleife führt einen Codeblock für jedes Element einer gegebenen Sequenz (siehe Abschnitt \ref{section:std_data_types:sequenzen}) aus.
Eine Sequenz kann zum Beispiel eine Liste oder ein String sein.
\begin{lstlisting}
for i in s: # der Ausdruck s muss als Sequenz interpretiert werden koennen
  ausdruck
  ...
\end{lstlisting}
Dabei referenziert die Laufvariable \lpy{i} jedes Element der Sequenz \lpy{s} nacheinander einmal. Durch Verwendung von \lpy{i} innerhalb 
des folgenden Codeblocks kann 
dann auf den Wert des jeweiligen Objekts zugegriffen werden. Nach Ausführung des Codeblocks der \lpy{for}-Schleife referenziert 
\lpy{i} dann das nächste Element von \lpy{s}, bis das Ende der Sequenz erreicht ist. 

Dies wird durch das folgende Beispiel veranschaulicht:
\begin{lstlisting}
s = 'Hallo Welt' # Die Sequenz 'Hallo Welt'
for i in s:      # Nimm alle i = H,a,l,l,o, ,W,e,l,t
  print(i)       # Drucke i aus
\end{lstlisting}
Hier ist unsere Sequenz \lpy{s} ein String mit dem Wert \lpy{Hallo Welt}. In der \lpy{for}-Schleife nimmt die Laufvariable nacheinander
die Werte der einzelnen Zeichen von \lpy{s}, also \lpy{H,a,l,} \lpy{l,o, ,W,e,l} und \lpy{t}. Im zur Schleife gehörenden Block werden 
diese Zeichen dann einzeln ausgegeben.

Wenn der Codeblock für eine aufsteigende Folge von Zahlen \lpy{from, from+1, ..., to-1} ausgeführt werden soll, nimmt man die Sequenz \lpy{range(from, to)}.
Wichtig ist, dass wir das halboffene Interval betrachten, also nur bis \lpy{to-1} und nicht bis \lpy{to} gehen.
Damit verhält sich der \Python-Code \lpy{for i in range(from, to):} ganz genau wie der \CC-Code \lstinline[language=C++,style=CPPinline]|for(i = from; i < to; ++i) { ... }|.
Hier ein Beispiel
\begin{lstlisting}
sum = 0              # Setze sum auf 0
for i in range(1,5): # Nimm alle i = 1, 2, 3, 4
  sum += 3*i         # Addiere 3*i zu sum
print( sum )         # Drucke sum aus
\end{lstlisting}

Will man eine absteigende Folge ganzer Zahlen, oder allgemeiner eine Folge ganzer Zahlen mit Schrittweite \lpy{step > 0} oder \lpy{step < 0} haben,
nimmt man die Sequenz \lpy{range(from,to,step)}.
Also können wir obiges Beispiel wie folgt umformulieren.
\begin{lstlisting}
sum = 0                  # Setze sum auf 0
for i in range(3,15, 3): # Nimm alle i = 3, 6, 9, 12
  sum += i               # Addiere i zu sum
print( sum )             # Drucke sum aus
\end{lstlisting}


\subsection{Funktionen}
\label{section:crashkurs:funktionen}
Meistens will man sich wiederholende Codeblöcke auslagern und natürlich kann man auch in \Python Funktionen definieren.
Jede Funktionsdefinition erstellt ein Objekt, welches der Funktion im Speicher entspricht.
Außerdem wird eine Variable definiert, welche auf die Funktion im Speicher referenziert.
\begin{lstlisting}
def funktionsname(parameter):
  ausdruck
  ...
\end{lstlisting}
Die Variable ist hier \lpy{funktionsname} und zeigt auf das Objekt, welches der definierten Funktion entspricht.

Anders als in \CC hat eine Funktion keine Signatur und keinen definierbaren Rückgabetyp.
Mit \lpy{return} können kein, ein oder mehrere Objekte zurückgegeben werden.
Wird kein Objekt zurückgegeben, so ist die Rückgabe automatisch \lpy{None}.
Wird ein Objekt \lpy{x} zurückgegeben, so ist der Rückgabetyp automatisch \lpy{type(x)}.
Werden mehrere Objekte zurückgegeben, so werden diese automatisch in einem \lpy{tuple} zusammengefasst, der Rückgabetyp ist also \lpy{tuple}.
Ein Tuple ist eine Liste, die nicht geändert werden kann (siehe Abschnitt \ref{section:std_data_types:sequenzen}).
\begin{lstlisting}
def f(i):
  if i == 1:
    return 1           # Rueckgabetyp: int,   Rueckgabewert: 1
  elif i == 2:
    return 'Hallo', 44 # Rueckgabetyp: tuple, Rueckgabewert: ('Hallo', 44)

print( type( f ) )                 # <class 'function'>
print( type( f(1) ) )              # <class 'int'>
print( type( f(2) ) )              # <class 'tuple'>
print( type( f('Gurkenwasser') ) ) # <class 'NoneType'>
\end{lstlisting}
Also ist das von \lpy{f(3)} oder auch \lpy{f('Gurkenwasser')} zurückgegebene Objekt \lpy{None}.

Beim Funktionsausruf übergibt man die Parameter entweder in der Reihenfolge wie sie in der Funktionsdefinition spezifiziert sind, oder man übergibt \lpy{parametername=ausdruck}.
Außerdem kann man den Argumenten einer Funktion Standardwerte übergeben.
Beim Funktionsausruf muss man die Argumente mit Standardwerten nicht angeben, kann es aber.
\begin{lstlisting}
def plus(a,b=1):
  return a+b

plus(4,5)      # = 9
plus(b=5, a=4) # = 9
plus(3)        # = 4
\end{lstlisting}


\subsection{Module}
\label{section:crashkurs:module}
Was \Python wirklich mächtig macht, ist nicht die Sprache an sich, sondern die schier unendliche Fülle an Paketen die \Python-Programmierer bereit stellen.
Ein Modul ist so etwas wie eine ``shared library'' in \CC.
Man kann sie importieren und dann nutzen, die Implementierung haben andere bereits übernommen (siehe Abbildung \ref{figure:xkcd_python}).
\begin{figure}[ht]
  \centering
  \includegraphics[width=0.5\textwidth]{pictures/xkcd_python.png}
  \caption{\label{figure:xkcd_python}``Python'' by Randall Munroe \cite{Munroe_python}}
\end{figure}

Ein Modul bindet man mit \lpy{import modulname} in sein Programm ein.
Alle Objekttypen (Klassen, Funktionen und andere Objekte) die \lpy{modulname} bereitstellt, können durch \lpy{modulname.oname} erreicht werden.
\begin{lstlisting}
import math
print( math.sin(math.pi/3.0) )
\end{lstlisting}
Wenn man aus \lpy{modulname} bestimmte Objekte einbinden möchte, nutzt man folgende Codezeile.
\begin{lstlisting}
from modulname import oname1, oname2, ...
\end{lstlisting}
Jetzt kann man auf die Objekte direkt (also ohne das Präfix \lpy{modulname.}) zugreifen.
Wir hatten bereits gesehen, wie man die \PythonDrei Printfunktion in \PythonZwei einbinden kann.
\begin{lstlisting}
from __future__ import print_function
\end{lstlisting}
Hier noch ein Beispiel.
\begin{lstlisting}
from math import sin, cos, pi
print( sin(pi/3.0), cos(pi/3.0) )
\end{lstlisting}

An dieser Stelle machen wir noch auf Abschnitt \ref{section:module:empfohlene_module} indem wir eine Hand voll \Python-Pakete erwähnen, die wir oft benutzen.


\subsection{Workflow}
\label{section:crashkurs:workflow}
Wir arbeiten recht erfolgreich mit folgendem Workflow.

Beginne ein Projekt in \Python.
Solange du noch nicht fertig bist:
Wenn eine Funktion zu aufwendig zu implementieren ist, suche nach einer bereits vorhandene \Python Bibliothek (es gibt bestimmt eine).
Wenn eine Funktion für dein Projekt zu langsam ist, schreibe sie in \CC und binde sie als Modul ein.
Wie genau letzteres geht, lernen wir noch.


\newpage

\section{Crashkurs}
\label{section:crashkurs}
Wir beginnen mit einem Überblick der grundlegende Bausteine.
In den nachfolgenden Kapiteln besprechen wir einige der Bausteine nochmal im Detail.


\subsection{Kommentare}
\label{section:crashkurs:kommentare}
Kommentare in \Python beginnen mit \lpy{#} und enden am Ende der Zeile.
Sie verhalten sich also ganz genau wie Kommentare in \CC, die mit \lcpp{//} beginnen%
\footnote{Die Kommentare \lcpp{//} wurden in \CNeunundneunzig eingeführt, siehe \cite{C99}.}.
Gut platzierte Kommentare dürfen in keinem Programm fehlen.
Ein schlecht kommentiertes Programm ist ein Anzeichen dafür, dass es von einem schlechten Programmierer geschrieben wurde.


\subsection{Print}
\label{section:crashkurs:print}
So gut wie alle Objekte können durch die \Python Funktion \lpy{print} auf der Standardausgabe ausgegeben werden.
Das liegt daran, dass die meisten Objekte Klassen sind und die Funktion \lpy{__str__} implementieren (auch wenn wir das jetzt noch nicht verstehen;
siehe Abschnitt \ref{section:klassen:spezielle_funktionen}).
Die Funktion \lpy{print} ist für \PythonZwei und \PythonDrei verschieden.
\begin{lstlisting}
print 'Hallo', 'Welt'   # Python2
print ('Hallo', 'Welt') # Python3
\end{lstlisting}
In diesem Skript nutzen wir die von \PythonDrei bereitgestellte Version von \lpy{print}.
Will man in \PythonZwei die \lpy{print}-Funktion aus \PythonDrei verwenden, kann man folgende Zeile am Anfang seines Skript einfügen.
\begin{lstlisting}
from __future__ import print_function # Nutze Python3-print in Python2
\end{lstlisting}


\subsection{Typ und Identität}
\label{section:crashkurs:typ_und_id}
Um die Identiät eines Objekts \lpy{x} zu bestimmen, nutzen wir die Funktion \lpy{id(x)}.
Sie gibt einen String zurück, den wir mit \lpy{print} ausdrucken können.
Ganz ähnlich bekommen wir den Typ eines Objekts mit \lpy{type(x)}.


\subsection{Zahlen}
\label{section:crashkurs:zahlen}
In \Python gibt es die Zahlentypen \lpy{int}, \lpy{long}, \lpy{float} und \lpy{complex}.
Sie verhalten sich (fast) genauso wie die aus \CC bekannten Typen \lcpp{int}, \lcpp{long}, \lcpp{float} und \lcpp{complex}%
\footnote{Der Typ \lpy{complex} wurde in \CNeunundneunzig eingeführt, siehe \cite{C99}.}.
Man erstellt ein Objekt vom Typ \lpy{int}, durch Angabe einer ganzen Zahl oder durch Aufrufen der Funktion \lpy{int}.
Ähnlich erstellt man ein Objekt vom Typ \lpy{float}, durch Angabe einer Kommazahl oder durch Aufrufen der Funktion \lpy{float}.
\begin{lstlisting}
print( 2 )        # 2    typ: int
print( int(2.1) ) # 2    typ: int
print( 2.0 )      # 2.0  typ: float
print( float(2) ) # 2.0  typ: float
\end{lstlisting}
Man kann zwei verschiedene oder gleiche Zahlentypen mit mathematischen Operatoren verbinden.
Das Ergebnis hat den ``genaueren'' Typ.
Des Weiteren beschreibt \lpy{//} die ganzzahlige Division ohne Rest.
\begin{lstlisting}
print( 2 + 3.3 )  # 5.3  typ: int +  float -> float
print( 5 // 2 )   # 2    typ: int // int   -> int
print( 5.3 // 2 ) # 2.0  typ: int // float -> float
\end{lstlisting}
Die Division von Ganzzahlen unterscheidet sich in \PythonZwei und \PythonDrei.
\begin{lstlisting}
4 /  2 # Python2: 2 vom Typ int;   Python3: 2.0 vom Typ float;
4 // 2 # Python2: 2 vom Typ int;   Python3: 2   vom Typ int;
\end{lstlisting}


\subsection{Strings}
\label{section:crashkurs:strings}
Strings die eine Zeile lang sind, werden entweder mit \lpy{'} oder mit \lstinline[style=PyInline]|"| begonnen und beendet.
Soll ein String mehre Zeilen lang sein, kann man ihn mit \lstinline[style=PyInline]|"""| beginnen und beenden.
Genau wie in \CC gibt es gewisse Zeichen, die nicht direkt geschrieben werden können und mit \lstinline[language=C++,style=CPP]|\| beginnen müssen (zum Beispiel werden neue Zeilen durch \lstinline[language=C++,style=CPP]|\n| ausgedrückt).
Folgender Code erstellt den String ``Hallo[Neue Zeile]Welt''.
\begin{lstlisting}
print( 'Hallo\nWelt' ) # Hallo[neue Zeile]Welt
\end{lstlisting}
Es gibt eine große Auswahl an Funktionen, um aus einem gegebenen String einen neuen zu erstellen.
Beispielsweise verknüpft man zwei Strings mit \lpy{+} und die Funktion \lpy{lower} ersetzt alle Großbuchstaben durch Kleinbuchstaben.
\begin{lstlisting}
print( 'Hallo' + ' ' + 'Welt' ) # Hallo Welt
print( 'Hallo Welt'.lower() )   # hallo welt
\end{lstlisting}


\subsection{Bedingungen}
\label{section:crashkurs:bedingungen}
In \Python gibt es genau zwei Objekte \lpy{True} und \lpy{False} von Typ \lpy{bool}.
(Fast) alle Typen können mit der Funktion \lpy{bool} in einen der beiden Objekte umgewandelt werden.
Den Kontrollfluss des Programms steuert man im einfachsten Fall wir folgt.
\begin{lstlisting}
if bedingung:
  ausdruck
  ...
\end{lstlisting}
Das heißt, wir brauchen zunächst eine Expression \lpy{bedingung}, die als \lpy{True} oder \lpy{False} interpretiert werden kann.
Wertet sie als \lpy{True} aus, so wird ein Block von Ausdrücken ausgeführt.
Anders als in \CC werden in \Python Blöcke nicht durch Klammern, sondern durch konsistente Einrückung definiert.
\begin{lstlisting}
x = 101//3+46//5
if x == 42:
  print('Die Antwort')
\end{lstlisting}
Da gut eingerückter Code (also Code, in dem jeder Block konsistent eingerückt ist) deutlich lesbarer sind als nicht eingerückter
Code, ist er in jedem Fall -- unabhängig von der Sprache -- ein Zeichen für eine gute Programmiererin. 

Da es jedoch trotzdem immer noch genug Leute gibt, die hässlichen Code produzieren, wird man in \Python dazu gezwungen, auf 
konsistente Einrückung zu achten: zu einem Block gehören genau die nachfolgenden Zeilen, die genauso weit eingerückt sind wie die 
erste Zeile des Blocks. Bei nicht konsistenter Einrückung gibt es Syntaxfehler. Dies macht es sehr schwer, hässlichen Code in
\Python zu schreiben.

Aus eigener Erfahrung können wir sagen, dass man sich als erfahrene \CC Programmiererin recht schnell an diese Konvention gewöhnen kann, 
wenn man auch schon vorher auf sauber geschriebene Programme geachtet hat.

Optional besteht eine Bedingung aus keinem oder mehreren \lpy{elif bedingung:}\footnote{sprich ``else if''} und keinem oder einem \lpy{else:}
Die Konstrukte \lpy{else:} und \lpy{elif bedingung:} in \Python verhalten sich wie die Konstrukte \lcpp{else} und \lcpp{else if} in \CC.
\begin{lstlisting}
x = 101//3+46//5               # x = ?
if x == 42:                    # Falls x den Wert 42 hat:
  print('Die Antwort')         # Drucke 'Die Antwort'
elif x == 84:                  # Sonst, falls x den Wert 84 hat:
  print('Zweimal die Antwort') # Drucke 'Zweimal die Antwort'
elif x == 106:                 # Sonst, falls x den Wert 106 hat:
  print('Dreimal die Antwort') # Drucke 'Dreimal die Antwort'
else:                          # In allen anderen Faellen:
  print('Versuchs nochmal')    # Drucke 'Versuchs nochmal'
\end{lstlisting}


\subsection{Listen}
\label{section:crashkurs:listen}
In \Python gibt es verschiedene ``Containertypen'' und hier besprechen wir den Typ \lpy{list}.
Eine Liste \lpy{thor} ist eine linear geordnete Menge von Variablen, auf die man mit \lpy{thor[0], thor[1], ..., thor[i]} zugreift.
Um auf Elemente von hinten zuzugreifen, sind negative Werte für \lpy{i} erlaubt.
\begin{lstlisting}
thor = [2, 'Hallo', 33.3, 'Welt', [] ] # Erzeugt eine Liste
print( thor[1], thor[-1] )             # Gibt folgendes aus: Hallo []
\end{lstlisting}
Die Länge einer Liste ist \lpy{len(thor)}.
Listen sind mutable, das heißt wir können sie verändern.
Mit \lpy{thor.append(x)} erweitert man die Liste \lpy{thor} um die Variable \lpy{x} und mit \lpy{thor.remove(x)} entfernt man die erste Variable mit dem Wert \lpy{x}.
Ob ein Objekt mit Wert \lpy{x} in \lpy{thor} enthalten ist, testet man mit \lpy{x in thor}.
Mit \lpy{thor+asgard} verknüpft man die beiden Listen \lpy{thor} und \lpy{asgard}.


\subsection{Schleifen}
\label{section:crashkurs:schleifen}
In \Python gibt es zwei Arten von Schleifen.
Die \lpy{while}-Schleife führt einen Codeblock aus, solange die vor der Ausführung des Blocks geprüfte Bedingung als \lpy{True} auswertet.
\begin{lstlisting}
while bedingung:
  ausdruck
  ...
\end{lstlisting}
Da uns dieses Konzept aus \CC sehr vertraut ist, brauchen wir hier kein Beispiel.

Die \lpy{for}-Schleife führt einen Codeblock für jedes Element einer gegebenen Sequenz (siehe Abschnitt \ref{section:std_data_types:sequenzen}) aus.
Eine Sequenz kann zum Beispiel eine Liste oder ein String sein.
\begin{lstlisting}
for i in s: # der Ausdruck s muss als Sequenz interpretiert werden koennen
  ausdruck
  ...
\end{lstlisting}
Dabei referenziert die Laufvariable \lpy{i} jedes Element der Sequenz \lpy{s} nacheinander einmal. Durch Verwendung von \lpy{i} innerhalb 
des folgenden Codeblocks kann 
dann auf den Wert des jeweiligen Objekts zugegriffen werden. Nach Ausführung des Codeblocks der \lpy{for}-Schleife referenziert 
\lpy{i} dann das nächste Element von \lpy{s}, bis das Ende der Sequenz erreicht ist. 

Dies wird durch das folgende Beispiel veranschaulicht:
\begin{lstlisting}
s = 'Hallo Welt' # Die Sequenz 'Hallo Welt'
for i in s:      # Nimm alle i = H,a,l,l,o, ,W,e,l,t
  print(i)       # Drucke i aus
\end{lstlisting}
Hier ist unsere Sequenz \lpy{s} ein String mit dem Wert \lpy{Hallo Welt}. In der \lpy{for}-Schleife nimmt die Laufvariable nacheinander
die Werte der einzelnen Zeichen von \lpy{s}, also \lpy{H,a,l,} \lpy{l,o, ,W,e,l} und \lpy{t}. Im zur Schleife gehörenden Block werden 
diese Zeichen dann einzeln ausgegeben.

Wenn der Codeblock für eine aufsteigende Folge von Zahlen \lpy{from, from+1, ..., to-1} ausgeführt werden soll, nimmt man die Sequenz \lpy{range(from, to)}.
Wichtig ist, dass wir das halboffene Interval betrachten, also nur bis \lpy{to-1} und nicht bis \lpy{to} gehen.
Damit verhält sich der \Python-Code \lpy{for i in range(from, to):} ganz genau wie der \CC-Code \lstinline[language=C++,style=CPPinline]|for(i = from; i < to; ++i) { ... }|.
Hier ein Beispiel
\begin{lstlisting}
sum = 0              # Setze sum auf 0
for i in range(1,5): # Nimm alle i = 1, 2, 3, 4
  sum += 3*i         # Addiere 3*i zu sum
print( sum )         # Drucke sum aus
\end{lstlisting}

Will man eine absteigende Folge ganzer Zahlen, oder allgemeiner eine Folge ganzer Zahlen mit Schrittweite \lpy{step > 0} oder \lpy{step < 0} haben,
nimmt man die Sequenz \lpy{range(from,to,step)}.
Also können wir obiges Beispiel wie folgt umformulieren.
\begin{lstlisting}
sum = 0                  # Setze sum auf 0
for i in range(3,15, 3): # Nimm alle i = 3, 6, 9, 12
  sum += i               # Addiere i zu sum
print( sum )             # Drucke sum aus
\end{lstlisting}


\subsection{Funktionen}
\label{section:crashkurs:funktionen}
Meistens will man sich wiederholende Codeblöcke auslagern und natürlich kann man auch in \Python Funktionen definieren.
Jede Funktionsdefinition erstellt ein Objekt, welches der Funktion im Speicher entspricht.
Außerdem wird eine Variable definiert, welche auf die Funktion im Speicher referenziert.
\begin{lstlisting}
def funktionsname(parameter):
  ausdruck
  ...
\end{lstlisting}
Die Variable ist hier \lpy{funktionsname} und zeigt auf das Objekt, welches der definierten Funktion entspricht.

Anders als in \CC hat eine Funktion keine Signatur und keinen definierbaren Rückgabetyp.
Mit \lpy{return} können kein, ein oder mehrere Objekte zurückgegeben werden.
Wird kein Objekt zurückgegeben, so ist die Rückgabe automatisch \lpy{None}.
Wird ein Objekt \lpy{x} zurückgegeben, so ist der Rückgabetyp automatisch \lpy{type(x)}.
Werden mehrere Objekte zurückgegeben, so werden diese automatisch in einem \lpy{tuple} zusammengefasst, der Rückgabetyp ist also \lpy{tuple}.
Ein Tuple ist eine Liste, die nicht geändert werden kann (siehe Abschnitt \ref{section:std_data_types:sequenzen}).
\begin{lstlisting}
def f(i):
  if i == 1:
    return 1           # Rueckgabetyp: int,   Rueckgabewert: 1
  elif i == 2:
    return 'Hallo', 44 # Rueckgabetyp: tuple, Rueckgabewert: ('Hallo', 44)

print( type( f ) )                 # <class 'function'>
print( type( f(1) ) )              # <class 'int'>
print( type( f(2) ) )              # <class 'tuple'>
print( type( f('Gurkenwasser') ) ) # <class 'NoneType'>
\end{lstlisting}
Also ist das von \lpy{f(3)} oder auch \lpy{f('Gurkenwasser')} zurückgegebene Objekt \lpy{None}.

Beim Funktionsausruf übergibt man die Parameter entweder in der Reihenfolge wie sie in der Funktionsdefinition spezifiziert sind, oder man übergibt \lpy{parametername=ausdruck}.
Außerdem kann man den Argumenten einer Funktion Standardwerte übergeben.
Beim Funktionsausruf muss man die Argumente mit Standardwerten nicht angeben, kann es aber.
\begin{lstlisting}
def plus(a,b=1):
  return a+b

plus(4,5)      # = 9
plus(b=5, a=4) # = 9
plus(3)        # = 4
\end{lstlisting}


\subsection{Module}
\label{section:crashkurs:module}
Was \Python wirklich mächtig macht, ist nicht die Sprache an sich, sondern die schier unendliche Fülle an Paketen die \Python-Programmierer bereit stellen.
Ein Modul ist so etwas wie eine ``shared library'' in \CC.
Man kann sie importieren und dann nutzen, die Implementierung haben andere bereits übernommen (siehe Abbildung \ref{figure:xkcd_python}).
\begin{figure}[ht]
  \centering
  \includegraphics[width=0.5\textwidth]{pictures/xkcd_python.png}
  \caption{\label{figure:xkcd_python}``Python'' by Randall Munroe \cite{Munroe_python}}
\end{figure}

Ein Modul bindet man mit \lpy{import modulname} in sein Programm ein.
Alle Objekttypen (Klassen, Funktionen und andere Objekte) die \lpy{modulname} bereitstellt, können durch \lpy{modulname.oname} erreicht werden.
\begin{lstlisting}
import math
print( math.sin(math.pi/3.0) )
\end{lstlisting}
Wenn man aus \lpy{modulname} bestimmte Objekte einbinden möchte, nutzt man folgende Codezeile.
\begin{lstlisting}
from modulname import oname1, oname2, ...
\end{lstlisting}
Jetzt kann man auf die Objekte direkt (also ohne das Präfix \lpy{modulname.}) zugreifen.
Wir hatten bereits gesehen, wie man die \PythonDrei Printfunktion in \PythonZwei einbinden kann.
\begin{lstlisting}
from __future__ import print_function
\end{lstlisting}
Hier noch ein Beispiel.
\begin{lstlisting}
from math import sin, cos, pi
print( sin(pi/3.0), cos(pi/3.0) )
\end{lstlisting}

An dieser Stelle machen wir noch auf Abschnitt \ref{section:module:empfohlene_module} indem wir eine Hand voll \Python-Pakete erwähnen, die wir oft benutzen.


\subsection{Workflow}
\label{section:crashkurs:workflow}
Wir arbeiten recht erfolgreich mit folgendem Workflow.

Beginne ein Projekt in \Python.
Solange du noch nicht fertig bist:
Wenn eine Funktion zu aufwendig zu implementieren ist, suche nach einer bereits vorhandene \Python Bibliothek (es gibt bestimmt eine).
Wenn eine Funktion für dein Projekt zu langsam ist, schreibe sie in \CC und binde sie als Modul ein.
Wie genau letzteres geht, lernen wir noch.


\newpage

\section{Crashkurs}
\label{section:crashkurs}
Wir beginnen mit einem Überblick der grundlegende Bausteine.
In den nachfolgenden Kapiteln besprechen wir einige der Bausteine nochmal im Detail.


\subsection{Kommentare}
\label{section:crashkurs:kommentare}
Kommentare in \Python beginnen mit \lpy{#} und enden am Ende der Zeile.
Sie verhalten sich also ganz genau wie Kommentare in \CC, die mit \lcpp{//} beginnen%
\footnote{Die Kommentare \lcpp{//} wurden in \CNeunundneunzig eingeführt, siehe \cite{C99}.}.
Gut platzierte Kommentare dürfen in keinem Programm fehlen.
Ein schlecht kommentiertes Programm ist ein Anzeichen dafür, dass es von einem schlechten Programmierer geschrieben wurde.


\subsection{Print}
\label{section:crashkurs:print}
So gut wie alle Objekte können durch die \Python Funktion \lpy{print} auf der Standardausgabe ausgegeben werden.
Das liegt daran, dass die meisten Objekte Klassen sind und die Funktion \lpy{__str__} implementieren (auch wenn wir das jetzt noch nicht verstehen;
siehe Abschnitt \ref{section:klassen:spezielle_funktionen}).
Die Funktion \lpy{print} ist für \PythonZwei und \PythonDrei verschieden.
\begin{lstlisting}
print 'Hallo', 'Welt'   # Python2
print ('Hallo', 'Welt') # Python3
\end{lstlisting}
In diesem Skript nutzen wir die von \PythonDrei bereitgestellte Version von \lpy{print}.
Will man in \PythonZwei die \lpy{print}-Funktion aus \PythonDrei verwenden, kann man folgende Zeile am Anfang seines Skript einfügen.
\begin{lstlisting}
from __future__ import print_function # Nutze Python3-print in Python2
\end{lstlisting}


\subsection{Typ und Identität}
\label{section:crashkurs:typ_und_id}
Um die Identiät eines Objekts \lpy{x} zu bestimmen, nutzen wir die Funktion \lpy{id(x)}.
Sie gibt einen String zurück, den wir mit \lpy{print} ausdrucken können.
Ganz ähnlich bekommen wir den Typ eines Objekts mit \lpy{type(x)}.


\subsection{Zahlen}
\label{section:crashkurs:zahlen}
In \Python gibt es die Zahlentypen \lpy{int}, \lpy{long}, \lpy{float} und \lpy{complex}.
Sie verhalten sich (fast) genauso wie die aus \CC bekannten Typen \lcpp{int}, \lcpp{long}, \lcpp{float} und \lcpp{complex}%
\footnote{Der Typ \lpy{complex} wurde in \CNeunundneunzig eingeführt, siehe \cite{C99}.}.
Man erstellt ein Objekt vom Typ \lpy{int}, durch Angabe einer ganzen Zahl oder durch Aufrufen der Funktion \lpy{int}.
Ähnlich erstellt man ein Objekt vom Typ \lpy{float}, durch Angabe einer Kommazahl oder durch Aufrufen der Funktion \lpy{float}.
\begin{lstlisting}
print( 2 )        # 2    typ: int
print( int(2.1) ) # 2    typ: int
print( 2.0 )      # 2.0  typ: float
print( float(2) ) # 2.0  typ: float
\end{lstlisting}
Man kann zwei verschiedene oder gleiche Zahlentypen mit mathematischen Operatoren verbinden.
Das Ergebnis hat den ``genaueren'' Typ.
Des Weiteren beschreibt \lpy{//} die ganzzahlige Division ohne Rest.
\begin{lstlisting}
print( 2 + 3.3 )  # 5.3  typ: int +  float -> float
print( 5 // 2 )   # 2    typ: int // int   -> int
print( 5.3 // 2 ) # 2.0  typ: int // float -> float
\end{lstlisting}
Die Division von Ganzzahlen unterscheidet sich in \PythonZwei und \PythonDrei.
\begin{lstlisting}
4 /  2 # Python2: 2 vom Typ int;   Python3: 2.0 vom Typ float;
4 // 2 # Python2: 2 vom Typ int;   Python3: 2   vom Typ int;
\end{lstlisting}


\subsection{Strings}
\label{section:crashkurs:strings}
Strings die eine Zeile lang sind, werden entweder mit \lpy{'} oder mit \lstinline[style=PyInline]|"| begonnen und beendet.
Soll ein String mehre Zeilen lang sein, kann man ihn mit \lstinline[style=PyInline]|"""| beginnen und beenden.
Genau wie in \CC gibt es gewisse Zeichen, die nicht direkt geschrieben werden können und mit \lstinline[language=C++,style=CPP]|\| beginnen müssen (zum Beispiel werden neue Zeilen durch \lstinline[language=C++,style=CPP]|\n| ausgedrückt).
Folgender Code erstellt den String ``Hallo[Neue Zeile]Welt''.
\begin{lstlisting}
print( 'Hallo\nWelt' ) # Hallo[neue Zeile]Welt
\end{lstlisting}
Es gibt eine große Auswahl an Funktionen, um aus einem gegebenen String einen neuen zu erstellen.
Beispielsweise verknüpft man zwei Strings mit \lpy{+} und die Funktion \lpy{lower} ersetzt alle Großbuchstaben durch Kleinbuchstaben.
\begin{lstlisting}
print( 'Hallo' + ' ' + 'Welt' ) # Hallo Welt
print( 'Hallo Welt'.lower() )   # hallo welt
\end{lstlisting}


\subsection{Bedingungen}
\label{section:crashkurs:bedingungen}
In \Python gibt es genau zwei Objekte \lpy{True} und \lpy{False} von Typ \lpy{bool}.
(Fast) alle Typen können mit der Funktion \lpy{bool} in einen der beiden Objekte umgewandelt werden.
Den Kontrollfluss des Programms steuert man im einfachsten Fall wir folgt.
\begin{lstlisting}
if bedingung:
  ausdruck
  ...
\end{lstlisting}
Das heißt, wir brauchen zunächst eine Expression \lpy{bedingung}, die als \lpy{True} oder \lpy{False} interpretiert werden kann.
Wertet sie als \lpy{True} aus, so wird ein Block von Ausdrücken ausgeführt.
Anders als in \CC werden in \Python Blöcke nicht durch Klammern, sondern durch konsistente Einrückung definiert.
\begin{lstlisting}
x = 101//3+46//5
if x == 42:
  print('Die Antwort')
\end{lstlisting}
Da gut eingerückter Code (also Code, in dem jeder Block konsistent eingerückt ist) deutlich lesbarer sind als nicht eingerückter
Code, ist er in jedem Fall -- unabhängig von der Sprache -- ein Zeichen für eine gute Programmiererin. 

Da es jedoch trotzdem immer noch genug Leute gibt, die hässlichen Code produzieren, wird man in \Python dazu gezwungen, auf 
konsistente Einrückung zu achten: zu einem Block gehören genau die nachfolgenden Zeilen, die genauso weit eingerückt sind wie die 
erste Zeile des Blocks. Bei nicht konsistenter Einrückung gibt es Syntaxfehler. Dies macht es sehr schwer, hässlichen Code in
\Python zu schreiben.

Aus eigener Erfahrung können wir sagen, dass man sich als erfahrene \CC Programmiererin recht schnell an diese Konvention gewöhnen kann, 
wenn man auch schon vorher auf sauber geschriebene Programme geachtet hat.

Optional besteht eine Bedingung aus keinem oder mehreren \lpy{elif bedingung:}\footnote{sprich ``else if''} und keinem oder einem \lpy{else:}
Die Konstrukte \lpy{else:} und \lpy{elif bedingung:} in \Python verhalten sich wie die Konstrukte \lcpp{else} und \lcpp{else if} in \CC.
\begin{lstlisting}
x = 101//3+46//5               # x = ?
if x == 42:                    # Falls x den Wert 42 hat:
  print('Die Antwort')         # Drucke 'Die Antwort'
elif x == 84:                  # Sonst, falls x den Wert 84 hat:
  print('Zweimal die Antwort') # Drucke 'Zweimal die Antwort'
elif x == 106:                 # Sonst, falls x den Wert 106 hat:
  print('Dreimal die Antwort') # Drucke 'Dreimal die Antwort'
else:                          # In allen anderen Faellen:
  print('Versuchs nochmal')    # Drucke 'Versuchs nochmal'
\end{lstlisting}


\subsection{Listen}
\label{section:crashkurs:listen}
In \Python gibt es verschiedene ``Containertypen'' und hier besprechen wir den Typ \lpy{list}.
Eine Liste \lpy{thor} ist eine linear geordnete Menge von Variablen, auf die man mit \lpy{thor[0], thor[1], ..., thor[i]} zugreift.
Um auf Elemente von hinten zuzugreifen, sind negative Werte für \lpy{i} erlaubt.
\begin{lstlisting}
thor = [2, 'Hallo', 33.3, 'Welt', [] ] # Erzeugt eine Liste
print( thor[1], thor[-1] )             # Gibt folgendes aus: Hallo []
\end{lstlisting}
Die Länge einer Liste ist \lpy{len(thor)}.
Listen sind mutable, das heißt wir können sie verändern.
Mit \lpy{thor.append(x)} erweitert man die Liste \lpy{thor} um die Variable \lpy{x} und mit \lpy{thor.remove(x)} entfernt man die erste Variable mit dem Wert \lpy{x}.
Ob ein Objekt mit Wert \lpy{x} in \lpy{thor} enthalten ist, testet man mit \lpy{x in thor}.
Mit \lpy{thor+asgard} verknüpft man die beiden Listen \lpy{thor} und \lpy{asgard}.


\subsection{Schleifen}
\label{section:crashkurs:schleifen}
In \Python gibt es zwei Arten von Schleifen.
Die \lpy{while}-Schleife führt einen Codeblock aus, solange die vor der Ausführung des Blocks geprüfte Bedingung als \lpy{True} auswertet.
\begin{lstlisting}
while bedingung:
  ausdruck
  ...
\end{lstlisting}
Da uns dieses Konzept aus \CC sehr vertraut ist, brauchen wir hier kein Beispiel.

Die \lpy{for}-Schleife führt einen Codeblock für jedes Element einer gegebenen Sequenz (siehe Abschnitt \ref{section:std_data_types:sequenzen}) aus.
Eine Sequenz kann zum Beispiel eine Liste oder ein String sein.
\begin{lstlisting}
for i in s: # der Ausdruck s muss als Sequenz interpretiert werden koennen
  ausdruck
  ...
\end{lstlisting}
Dabei referenziert die Laufvariable \lpy{i} jedes Element der Sequenz \lpy{s} nacheinander einmal. Durch Verwendung von \lpy{i} innerhalb 
des folgenden Codeblocks kann 
dann auf den Wert des jeweiligen Objekts zugegriffen werden. Nach Ausführung des Codeblocks der \lpy{for}-Schleife referenziert 
\lpy{i} dann das nächste Element von \lpy{s}, bis das Ende der Sequenz erreicht ist. 

Dies wird durch das folgende Beispiel veranschaulicht:
\begin{lstlisting}
s = 'Hallo Welt' # Die Sequenz 'Hallo Welt'
for i in s:      # Nimm alle i = H,a,l,l,o, ,W,e,l,t
  print(i)       # Drucke i aus
\end{lstlisting}
Hier ist unsere Sequenz \lpy{s} ein String mit dem Wert \lpy{Hallo Welt}. In der \lpy{for}-Schleife nimmt die Laufvariable nacheinander
die Werte der einzelnen Zeichen von \lpy{s}, also \lpy{H,a,l,} \lpy{l,o, ,W,e,l} und \lpy{t}. Im zur Schleife gehörenden Block werden 
diese Zeichen dann einzeln ausgegeben.

Wenn der Codeblock für eine aufsteigende Folge von Zahlen \lpy{from, from+1, ..., to-1} ausgeführt werden soll, nimmt man die Sequenz \lpy{range(from, to)}.
Wichtig ist, dass wir das halboffene Interval betrachten, also nur bis \lpy{to-1} und nicht bis \lpy{to} gehen.
Damit verhält sich der \Python-Code \lpy{for i in range(from, to):} ganz genau wie der \CC-Code \lstinline[language=C++,style=CPPinline]|for(i = from; i < to; ++i) { ... }|.
Hier ein Beispiel
\begin{lstlisting}
sum = 0              # Setze sum auf 0
for i in range(1,5): # Nimm alle i = 1, 2, 3, 4
  sum += 3*i         # Addiere 3*i zu sum
print( sum )         # Drucke sum aus
\end{lstlisting}

Will man eine absteigende Folge ganzer Zahlen, oder allgemeiner eine Folge ganzer Zahlen mit Schrittweite \lpy{step > 0} oder \lpy{step < 0} haben,
nimmt man die Sequenz \lpy{range(from,to,step)}.
Also können wir obiges Beispiel wie folgt umformulieren.
\begin{lstlisting}
sum = 0                  # Setze sum auf 0
for i in range(3,15, 3): # Nimm alle i = 3, 6, 9, 12
  sum += i               # Addiere i zu sum
print( sum )             # Drucke sum aus
\end{lstlisting}


\subsection{Funktionen}
\label{section:crashkurs:funktionen}
Meistens will man sich wiederholende Codeblöcke auslagern und natürlich kann man auch in \Python Funktionen definieren.
Jede Funktionsdefinition erstellt ein Objekt, welches der Funktion im Speicher entspricht.
Außerdem wird eine Variable definiert, welche auf die Funktion im Speicher referenziert.
\begin{lstlisting}
def funktionsname(parameter):
  ausdruck
  ...
\end{lstlisting}
Die Variable ist hier \lpy{funktionsname} und zeigt auf das Objekt, welches der definierten Funktion entspricht.

Anders als in \CC hat eine Funktion keine Signatur und keinen definierbaren Rückgabetyp.
Mit \lpy{return} können kein, ein oder mehrere Objekte zurückgegeben werden.
Wird kein Objekt zurückgegeben, so ist die Rückgabe automatisch \lpy{None}.
Wird ein Objekt \lpy{x} zurückgegeben, so ist der Rückgabetyp automatisch \lpy{type(x)}.
Werden mehrere Objekte zurückgegeben, so werden diese automatisch in einem \lpy{tuple} zusammengefasst, der Rückgabetyp ist also \lpy{tuple}.
Ein Tuple ist eine Liste, die nicht geändert werden kann (siehe Abschnitt \ref{section:std_data_types:sequenzen}).
\begin{lstlisting}
def f(i):
  if i == 1:
    return 1           # Rueckgabetyp: int,   Rueckgabewert: 1
  elif i == 2:
    return 'Hallo', 44 # Rueckgabetyp: tuple, Rueckgabewert: ('Hallo', 44)

print( type( f ) )                 # <class 'function'>
print( type( f(1) ) )              # <class 'int'>
print( type( f(2) ) )              # <class 'tuple'>
print( type( f('Gurkenwasser') ) ) # <class 'NoneType'>
\end{lstlisting}
Also ist das von \lpy{f(3)} oder auch \lpy{f('Gurkenwasser')} zurückgegebene Objekt \lpy{None}.

Beim Funktionsausruf übergibt man die Parameter entweder in der Reihenfolge wie sie in der Funktionsdefinition spezifiziert sind, oder man übergibt \lpy{parametername=ausdruck}.
Außerdem kann man den Argumenten einer Funktion Standardwerte übergeben.
Beim Funktionsausruf muss man die Argumente mit Standardwerten nicht angeben, kann es aber.
\begin{lstlisting}
def plus(a,b=1):
  return a+b

plus(4,5)      # = 9
plus(b=5, a=4) # = 9
plus(3)        # = 4
\end{lstlisting}


\subsection{Module}
\label{section:crashkurs:module}
Was \Python wirklich mächtig macht, ist nicht die Sprache an sich, sondern die schier unendliche Fülle an Paketen die \Python-Programmierer bereit stellen.
Ein Modul ist so etwas wie eine ``shared library'' in \CC.
Man kann sie importieren und dann nutzen, die Implementierung haben andere bereits übernommen (siehe Abbildung \ref{figure:xkcd_python}).
\begin{figure}[ht]
  \centering
  \includegraphics[width=0.5\textwidth]{pictures/xkcd_python.png}
  \caption{\label{figure:xkcd_python}``Python'' by Randall Munroe \cite{Munroe_python}}
\end{figure}

Ein Modul bindet man mit \lpy{import modulname} in sein Programm ein.
Alle Objekttypen (Klassen, Funktionen und andere Objekte) die \lpy{modulname} bereitstellt, können durch \lpy{modulname.oname} erreicht werden.
\begin{lstlisting}
import math
print( math.sin(math.pi/3.0) )
\end{lstlisting}
Wenn man aus \lpy{modulname} bestimmte Objekte einbinden möchte, nutzt man folgende Codezeile.
\begin{lstlisting}
from modulname import oname1, oname2, ...
\end{lstlisting}
Jetzt kann man auf die Objekte direkt (also ohne das Präfix \lpy{modulname.}) zugreifen.
Wir hatten bereits gesehen, wie man die \PythonDrei Printfunktion in \PythonZwei einbinden kann.
\begin{lstlisting}
from __future__ import print_function
\end{lstlisting}
Hier noch ein Beispiel.
\begin{lstlisting}
from math import sin, cos, pi
print( sin(pi/3.0), cos(pi/3.0) )
\end{lstlisting}

An dieser Stelle machen wir noch auf Abschnitt \ref{section:module:empfohlene_module} indem wir eine Hand voll \Python-Pakete erwähnen, die wir oft benutzen.


\subsection{Workflow}
\label{section:crashkurs:workflow}
Wir arbeiten recht erfolgreich mit folgendem Workflow.

Beginne ein Projekt in \Python.
Solange du noch nicht fertig bist:
Wenn eine Funktion zu aufwendig zu implementieren ist, suche nach einer bereits vorhandene \Python Bibliothek (es gibt bestimmt eine).
Wenn eine Funktion für dein Projekt zu langsam ist, schreibe sie in \CC und binde sie als Modul ein.
Wie genau letzteres geht, lernen wir noch.


\newpage

\section{Crashkurs}
\label{section:crashkurs}
Wir beginnen mit einem Überblick der grundlegende Bausteine.
In den nachfolgenden Kapiteln besprechen wir einige der Bausteine nochmal im Detail.


\subsection{Kommentare}
\label{section:crashkurs:kommentare}
Kommentare in \Python beginnen mit \lpy{#} und enden am Ende der Zeile.
Sie verhalten sich also ganz genau wie Kommentare in \CC, die mit \lcpp{//} beginnen%
\footnote{Die Kommentare \lcpp{//} wurden in \CNeunundneunzig eingeführt, siehe \cite{C99}.}.
Gut platzierte Kommentare dürfen in keinem Programm fehlen.
Ein schlecht kommentiertes Programm ist ein Anzeichen dafür, dass es von einem schlechten Programmierer geschrieben wurde.


\subsection{Print}
\label{section:crashkurs:print}
So gut wie alle Objekte können durch die \Python Funktion \lpy{print} auf der Standardausgabe ausgegeben werden.
Das liegt daran, dass die meisten Objekte Klassen sind und die Funktion \lpy{__str__} implementieren (auch wenn wir das jetzt noch nicht verstehen;
siehe Abschnitt \ref{section:klassen:spezielle_funktionen}).
Die Funktion \lpy{print} ist für \PythonZwei und \PythonDrei verschieden.
\begin{lstlisting}
print 'Hallo', 'Welt'   # Python2
print ('Hallo', 'Welt') # Python3
\end{lstlisting}
In diesem Skript nutzen wir die von \PythonDrei bereitgestellte Version von \lpy{print}.
Will man in \PythonZwei die \lpy{print}-Funktion aus \PythonDrei verwenden, kann man folgende Zeile am Anfang seines Skript einfügen.
\begin{lstlisting}
from __future__ import print_function # Nutze Python3-print in Python2
\end{lstlisting}


\subsection{Typ und Identität}
\label{section:crashkurs:typ_und_id}
Um die Identiät eines Objekts \lpy{x} zu bestimmen, nutzen wir die Funktion \lpy{id(x)}.
Sie gibt einen String zurück, den wir mit \lpy{print} ausdrucken können.
Ganz ähnlich bekommen wir den Typ eines Objekts mit \lpy{type(x)}.


\subsection{Zahlen}
\label{section:crashkurs:zahlen}
In \Python gibt es die Zahlentypen \lpy{int}, \lpy{long}, \lpy{float} und \lpy{complex}.
Sie verhalten sich (fast) genauso wie die aus \CC bekannten Typen \lcpp{int}, \lcpp{long}, \lcpp{float} und \lcpp{complex}%
\footnote{Der Typ \lpy{complex} wurde in \CNeunundneunzig eingeführt, siehe \cite{C99}.}.
Man erstellt ein Objekt vom Typ \lpy{int}, durch Angabe einer ganzen Zahl oder durch Aufrufen der Funktion \lpy{int}.
Ähnlich erstellt man ein Objekt vom Typ \lpy{float}, durch Angabe einer Kommazahl oder durch Aufrufen der Funktion \lpy{float}.
\begin{lstlisting}
print( 2 )        # 2    typ: int
print( int(2.1) ) # 2    typ: int
print( 2.0 )      # 2.0  typ: float
print( float(2) ) # 2.0  typ: float
\end{lstlisting}
Man kann zwei verschiedene oder gleiche Zahlentypen mit mathematischen Operatoren verbinden.
Das Ergebnis hat den ``genaueren'' Typ.
Des Weiteren beschreibt \lpy{//} die ganzzahlige Division ohne Rest.
\begin{lstlisting}
print( 2 + 3.3 )  # 5.3  typ: int +  float -> float
print( 5 // 2 )   # 2    typ: int // int   -> int
print( 5.3 // 2 ) # 2.0  typ: int // float -> float
\end{lstlisting}
Die Division von Ganzzahlen unterscheidet sich in \PythonZwei und \PythonDrei.
\begin{lstlisting}
4 /  2 # Python2: 2 vom Typ int;   Python3: 2.0 vom Typ float;
4 // 2 # Python2: 2 vom Typ int;   Python3: 2   vom Typ int;
\end{lstlisting}


\subsection{Strings}
\label{section:crashkurs:strings}
Strings die eine Zeile lang sind, werden entweder mit \lpy{'} oder mit \lstinline[style=PyInline]|"| begonnen und beendet.
Soll ein String mehre Zeilen lang sein, kann man ihn mit \lstinline[style=PyInline]|"""| beginnen und beenden.
Genau wie in \CC gibt es gewisse Zeichen, die nicht direkt geschrieben werden können und mit \lstinline[language=C++,style=CPP]|\| beginnen müssen (zum Beispiel werden neue Zeilen durch \lstinline[language=C++,style=CPP]|\n| ausgedrückt).
Folgender Code erstellt den String ``Hallo[Neue Zeile]Welt''.
\begin{lstlisting}
print( 'Hallo\nWelt' ) # Hallo[neue Zeile]Welt
\end{lstlisting}
Es gibt eine große Auswahl an Funktionen, um aus einem gegebenen String einen neuen zu erstellen.
Beispielsweise verknüpft man zwei Strings mit \lpy{+} und die Funktion \lpy{lower} ersetzt alle Großbuchstaben durch Kleinbuchstaben.
\begin{lstlisting}
print( 'Hallo' + ' ' + 'Welt' ) # Hallo Welt
print( 'Hallo Welt'.lower() )   # hallo welt
\end{lstlisting}


\subsection{Bedingungen}
\label{section:crashkurs:bedingungen}
In \Python gibt es genau zwei Objekte \lpy{True} und \lpy{False} von Typ \lpy{bool}.
(Fast) alle Typen können mit der Funktion \lpy{bool} in einen der beiden Objekte umgewandelt werden.
Den Kontrollfluss des Programms steuert man im einfachsten Fall wir folgt.
\begin{lstlisting}
if bedingung:
  ausdruck
  ...
\end{lstlisting}
Das heißt, wir brauchen zunächst eine Expression \lpy{bedingung}, die als \lpy{True} oder \lpy{False} interpretiert werden kann.
Wertet sie als \lpy{True} aus, so wird ein Block von Ausdrücken ausgeführt.
Anders als in \CC werden in \Python Blöcke nicht durch Klammern, sondern durch konsistente Einrückung definiert.
\begin{lstlisting}
x = 101//3+46//5
if x == 42:
  print('Die Antwort')
\end{lstlisting}
Da gut eingerückter Code (also Code, in dem jeder Block konsistent eingerückt ist) deutlich lesbarer sind als nicht eingerückter
Code, ist er in jedem Fall -- unabhängig von der Sprache -- ein Zeichen für eine gute Programmiererin. 

Da es jedoch trotzdem immer noch genug Leute gibt, die hässlichen Code produzieren, wird man in \Python dazu gezwungen, auf 
konsistente Einrückung zu achten: zu einem Block gehören genau die nachfolgenden Zeilen, die genauso weit eingerückt sind wie die 
erste Zeile des Blocks. Bei nicht konsistenter Einrückung gibt es Syntaxfehler. Dies macht es sehr schwer, hässlichen Code in
\Python zu schreiben.

Aus eigener Erfahrung können wir sagen, dass man sich als erfahrene \CC Programmiererin recht schnell an diese Konvention gewöhnen kann, 
wenn man auch schon vorher auf sauber geschriebene Programme geachtet hat.

Optional besteht eine Bedingung aus keinem oder mehreren \lpy{elif bedingung:}\footnote{sprich ``else if''} und keinem oder einem \lpy{else:}
Die Konstrukte \lpy{else:} und \lpy{elif bedingung:} in \Python verhalten sich wie die Konstrukte \lcpp{else} und \lcpp{else if} in \CC.
\begin{lstlisting}
x = 101//3+46//5               # x = ?
if x == 42:                    # Falls x den Wert 42 hat:
  print('Die Antwort')         # Drucke 'Die Antwort'
elif x == 84:                  # Sonst, falls x den Wert 84 hat:
  print('Zweimal die Antwort') # Drucke 'Zweimal die Antwort'
elif x == 106:                 # Sonst, falls x den Wert 106 hat:
  print('Dreimal die Antwort') # Drucke 'Dreimal die Antwort'
else:                          # In allen anderen Faellen:
  print('Versuchs nochmal')    # Drucke 'Versuchs nochmal'
\end{lstlisting}


\subsection{Listen}
\label{section:crashkurs:listen}
In \Python gibt es verschiedene ``Containertypen'' und hier besprechen wir den Typ \lpy{list}.
Eine Liste \lpy{thor} ist eine linear geordnete Menge von Variablen, auf die man mit \lpy{thor[0], thor[1], ..., thor[i]} zugreift.
Um auf Elemente von hinten zuzugreifen, sind negative Werte für \lpy{i} erlaubt.
\begin{lstlisting}
thor = [2, 'Hallo', 33.3, 'Welt', [] ] # Erzeugt eine Liste
print( thor[1], thor[-1] )             # Gibt folgendes aus: Hallo []
\end{lstlisting}
Die Länge einer Liste ist \lpy{len(thor)}.
Listen sind mutable, das heißt wir können sie verändern.
Mit \lpy{thor.append(x)} erweitert man die Liste \lpy{thor} um die Variable \lpy{x} und mit \lpy{thor.remove(x)} entfernt man die erste Variable mit dem Wert \lpy{x}.
Ob ein Objekt mit Wert \lpy{x} in \lpy{thor} enthalten ist, testet man mit \lpy{x in thor}.
Mit \lpy{thor+asgard} verknüpft man die beiden Listen \lpy{thor} und \lpy{asgard}.


\subsection{Schleifen}
\label{section:crashkurs:schleifen}
In \Python gibt es zwei Arten von Schleifen.
Die \lpy{while}-Schleife führt einen Codeblock aus, solange die vor der Ausführung des Blocks geprüfte Bedingung als \lpy{True} auswertet.
\begin{lstlisting}
while bedingung:
  ausdruck
  ...
\end{lstlisting}
Da uns dieses Konzept aus \CC sehr vertraut ist, brauchen wir hier kein Beispiel.

Die \lpy{for}-Schleife führt einen Codeblock für jedes Element einer gegebenen Sequenz (siehe Abschnitt \ref{section:std_data_types:sequenzen}) aus.
Eine Sequenz kann zum Beispiel eine Liste oder ein String sein.
\begin{lstlisting}
for i in s: # der Ausdruck s muss als Sequenz interpretiert werden koennen
  ausdruck
  ...
\end{lstlisting}
Dabei referenziert die Laufvariable \lpy{i} jedes Element der Sequenz \lpy{s} nacheinander einmal. Durch Verwendung von \lpy{i} innerhalb 
des folgenden Codeblocks kann 
dann auf den Wert des jeweiligen Objekts zugegriffen werden. Nach Ausführung des Codeblocks der \lpy{for}-Schleife referenziert 
\lpy{i} dann das nächste Element von \lpy{s}, bis das Ende der Sequenz erreicht ist. 

Dies wird durch das folgende Beispiel veranschaulicht:
\begin{lstlisting}
s = 'Hallo Welt' # Die Sequenz 'Hallo Welt'
for i in s:      # Nimm alle i = H,a,l,l,o, ,W,e,l,t
  print(i)       # Drucke i aus
\end{lstlisting}
Hier ist unsere Sequenz \lpy{s} ein String mit dem Wert \lpy{Hallo Welt}. In der \lpy{for}-Schleife nimmt die Laufvariable nacheinander
die Werte der einzelnen Zeichen von \lpy{s}, also \lpy{H,a,l,} \lpy{l,o, ,W,e,l} und \lpy{t}. Im zur Schleife gehörenden Block werden 
diese Zeichen dann einzeln ausgegeben.

Wenn der Codeblock für eine aufsteigende Folge von Zahlen \lpy{from, from+1, ..., to-1} ausgeführt werden soll, nimmt man die Sequenz \lpy{range(from, to)}.
Wichtig ist, dass wir das halboffene Interval betrachten, also nur bis \lpy{to-1} und nicht bis \lpy{to} gehen.
Damit verhält sich der \Python-Code \lpy{for i in range(from, to):} ganz genau wie der \CC-Code \lstinline[language=C++,style=CPPinline]|for(i = from; i < to; ++i) { ... }|.
Hier ein Beispiel
\begin{lstlisting}
sum = 0              # Setze sum auf 0
for i in range(1,5): # Nimm alle i = 1, 2, 3, 4
  sum += 3*i         # Addiere 3*i zu sum
print( sum )         # Drucke sum aus
\end{lstlisting}

Will man eine absteigende Folge ganzer Zahlen, oder allgemeiner eine Folge ganzer Zahlen mit Schrittweite \lpy{step > 0} oder \lpy{step < 0} haben,
nimmt man die Sequenz \lpy{range(from,to,step)}.
Also können wir obiges Beispiel wie folgt umformulieren.
\begin{lstlisting}
sum = 0                  # Setze sum auf 0
for i in range(3,15, 3): # Nimm alle i = 3, 6, 9, 12
  sum += i               # Addiere i zu sum
print( sum )             # Drucke sum aus
\end{lstlisting}


\subsection{Funktionen}
\label{section:crashkurs:funktionen}
Meistens will man sich wiederholende Codeblöcke auslagern und natürlich kann man auch in \Python Funktionen definieren.
Jede Funktionsdefinition erstellt ein Objekt, welches der Funktion im Speicher entspricht.
Außerdem wird eine Variable definiert, welche auf die Funktion im Speicher referenziert.
\begin{lstlisting}
def funktionsname(parameter):
  ausdruck
  ...
\end{lstlisting}
Die Variable ist hier \lpy{funktionsname} und zeigt auf das Objekt, welches der definierten Funktion entspricht.

Anders als in \CC hat eine Funktion keine Signatur und keinen definierbaren Rückgabetyp.
Mit \lpy{return} können kein, ein oder mehrere Objekte zurückgegeben werden.
Wird kein Objekt zurückgegeben, so ist die Rückgabe automatisch \lpy{None}.
Wird ein Objekt \lpy{x} zurückgegeben, so ist der Rückgabetyp automatisch \lpy{type(x)}.
Werden mehrere Objekte zurückgegeben, so werden diese automatisch in einem \lpy{tuple} zusammengefasst, der Rückgabetyp ist also \lpy{tuple}.
Ein Tuple ist eine Liste, die nicht geändert werden kann (siehe Abschnitt \ref{section:std_data_types:sequenzen}).
\begin{lstlisting}
def f(i):
  if i == 1:
    return 1           # Rueckgabetyp: int,   Rueckgabewert: 1
  elif i == 2:
    return 'Hallo', 44 # Rueckgabetyp: tuple, Rueckgabewert: ('Hallo', 44)

print( type( f ) )                 # <class 'function'>
print( type( f(1) ) )              # <class 'int'>
print( type( f(2) ) )              # <class 'tuple'>
print( type( f('Gurkenwasser') ) ) # <class 'NoneType'>
\end{lstlisting}
Also ist das von \lpy{f(3)} oder auch \lpy{f('Gurkenwasser')} zurückgegebene Objekt \lpy{None}.

Beim Funktionsausruf übergibt man die Parameter entweder in der Reihenfolge wie sie in der Funktionsdefinition spezifiziert sind, oder man übergibt \lpy{parametername=ausdruck}.
Außerdem kann man den Argumenten einer Funktion Standardwerte übergeben.
Beim Funktionsausruf muss man die Argumente mit Standardwerten nicht angeben, kann es aber.
\begin{lstlisting}
def plus(a,b=1):
  return a+b

plus(4,5)      # = 9
plus(b=5, a=4) # = 9
plus(3)        # = 4
\end{lstlisting}


\subsection{Module}
\label{section:crashkurs:module}
Was \Python wirklich mächtig macht, ist nicht die Sprache an sich, sondern die schier unendliche Fülle an Paketen die \Python-Programmierer bereit stellen.
Ein Modul ist so etwas wie eine ``shared library'' in \CC.
Man kann sie importieren und dann nutzen, die Implementierung haben andere bereits übernommen (siehe Abbildung \ref{figure:xkcd_python}).
\begin{figure}[ht]
  \centering
  \includegraphics[width=0.5\textwidth]{pictures/xkcd_python.png}
  \caption{\label{figure:xkcd_python}``Python'' by Randall Munroe \cite{Munroe_python}}
\end{figure}

Ein Modul bindet man mit \lpy{import modulname} in sein Programm ein.
Alle Objekttypen (Klassen, Funktionen und andere Objekte) die \lpy{modulname} bereitstellt, können durch \lpy{modulname.oname} erreicht werden.
\begin{lstlisting}
import math
print( math.sin(math.pi/3.0) )
\end{lstlisting}
Wenn man aus \lpy{modulname} bestimmte Objekte einbinden möchte, nutzt man folgende Codezeile.
\begin{lstlisting}
from modulname import oname1, oname2, ...
\end{lstlisting}
Jetzt kann man auf die Objekte direkt (also ohne das Präfix \lpy{modulname.}) zugreifen.
Wir hatten bereits gesehen, wie man die \PythonDrei Printfunktion in \PythonZwei einbinden kann.
\begin{lstlisting}
from __future__ import print_function
\end{lstlisting}
Hier noch ein Beispiel.
\begin{lstlisting}
from math import sin, cos, pi
print( sin(pi/3.0), cos(pi/3.0) )
\end{lstlisting}

An dieser Stelle machen wir noch auf Abschnitt \ref{section:module:empfohlene_module} indem wir eine Hand voll \Python-Pakete erwähnen, die wir oft benutzen.


\subsection{Workflow}
\label{section:crashkurs:workflow}
Wir arbeiten recht erfolgreich mit folgendem Workflow.

Beginne ein Projekt in \Python.
Solange du noch nicht fertig bist:
Wenn eine Funktion zu aufwendig zu implementieren ist, suche nach einer bereits vorhandene \Python Bibliothek (es gibt bestimmt eine).
Wenn eine Funktion für dein Projekt zu langsam ist, schreibe sie in \CC und binde sie als Modul ein.
Wie genau letzteres geht, lernen wir noch.


\newpage

\section{Ausnahmen}
\label{section:ausnahmen}
Gut geschriebene Bibliotheken und Programme stürzen nicht ab, wenn sie falsch bedient werden oder
wenn eine andere außergewöhnliche Situation eintritt.
Beispielsweise soll unsere \lcpp{joelixblas} Bibliothek nicht abstürzen, wenn der Benutzer eine Funktion mit unzulässigen Werten aufruft oder
wenn der Benutzer mehr Speicher verlangt, als das System uns bieten kann.

In \Python trennt man den ``gewöhnlichen Programmfluss'' und die ``außer\-ge\-wöhn\-lich\-en Programmfluss'' wie folgt:
Wenn der normale, sequenzielle Programmfluss durch ein außergewöhnliches Vorkommnis unterbrochen werden muss, nennt man das eine \emph{Ausnahme} oder \emph{Exception}.
Sobald eine Ausnahme \emph{ausgelöst} wird, muss sie \emph{behandelt} werden.

Typische Ausnahmen sind Fehler wie Syntaxfehler, Zugriffsfehler oder unzureichend viel Speicher.
Es gibt in \Python auch Ausnahmen, die nicht durch Fehler ausgelöst werden.
Außerdem kann man eigene Ausnahmen definieren.

Wird eine Ausnahme nicht behandelt, bricht der Programmfluss der gerade laufenden Funktion ab und die Ausnahme wird an die aufrufende Funktion weitergegeben.
Die Ausnahme wird dann entweder dort behandelt, oder das Weiterreichen wird fortgesetzt bis die Ausnahme entweder irgendwann behandelt wird oder das Programm mit der besagten Ausnahme abbricht.

Die folgende Ausnahme hat man bestimmt schon einige Male gesehen, wenn man Code direkt im \Python-Interpreter schreibt:
\begin{lstlisting}
a,b = 2,5
if a == b
  print( "A ist ja wirklich B")

# Der Code liefert:
#   File "...", line ...
#     if a == b
#            ^
# SyntaxError: invalid syntax
\end{lstlisting}
Hier wird die zweite Zeile vom \Python-Interpreter gelesen.
Dieser stellt einen Syntaxfehler fest und löst eine Ausnahme aus, die den Programmfluss an dieser Stelle unterbricht.
Schlussendlich wird das Programm beendet.

\subsection{Ausnahmen behandeln}
\label{section:ausnahmen:ausnahmen_behandeln}
Bevor wir erklären, wie eine Ausnahme behandelt wird, gehen wir kurz darauf ein, was eine Ausnahme ist:
Eine Ausnahme ist ein Objekt und somit hat sie einen festen Typ.
Der \Python-Standard definiert eine Hand voll Ausnahmetypen, wie zum Beispiel
den Ausnahmetyp \lpy{ZeroDivisionError} (der bei einer Division durch Null ausgelöst wird) oder
den Ausnahmetyp \lpy{SyntaxError} (den der \Python-Interpreter bei einem Syntaxfehler auslöst).
Der Wert einer Ausnahme \lpy{ausnahme} vom Ausnahmetyp \lpy{Ausnahmetyp} ist (für uns und nur hier) ein String und jede Ausnahme kann als String interpretiert werden.
Eine Liste der wichtigsten, bereits definierten Ausnahmetypen sind in Abschnitt \ref{section:ausnahmen:definierte_und_eigene_ausnahmen} zu finden.

Ein Programmabschnitt in dem Ausnahmen ausgelöst werden können, die wir (im Fall das Fälle) behandeln wollen,
wird der Programmabschnitt in ein \lpy{try-except} Konstrukt eingefasst.
Nach \lpy{try:} folgt unser (eingerückter) Programmabschnitt.
Dann werden die möglichen Ausnahmen behandelt.
Möchte oder muss man eine Ausnahme vom Ausnahmetyp \lpy{Ausnahmetyp} behandeln, geschieht das mit \lpy{except Ausnahmetyp:} gefolgt von der auszuführenden Ausnahmebehandlung.
Hier kann man auch mehrere Ausnahmentypen zusammenfassen mit \lpy{except (Ausnahmetyp_1, Ausnahmetyp_2, ...):}.
Alle anderen Ausnahmentypen sammelt man mit \lpy{except:}.
Das sieht dann zum Beispiel so aus:
\begin{lstlisting}
# Programmfluss

try:
  # Abschnitt der Ausnahmen ausloesen kann, die wir behandeln wollen
except Ausnahmetyp_1:
  # Ausnahmetyp 1 behandeln
except (Ausnahmetyp_2, Ausnahmetyp_3):
  # Ausnahmetyp 2 behandeln
# Hier noch mehr Ausnahmentypen die man behandeln moechte
except:
  # Alle anderen Ausnahmen behandeln

# Hier geht der normale Programmfluss weiter
\end{lstlisting}

Um auf eine Ausnahme des Ausnahmetyps \lpy{Ausnahmetyp} zugreifen zu können nutzt man \lpy{except Ausnahmetyp as ausnahme}.
Hier ein Beispiel, das die Ausnahme ``Division durch Null'' behandelt.
\begin{lstlisting}
try:
  a = 1/0
except ZeroDivisionError as ausnahme:
  print('Ausnahme:', ausnahme)
\end{lstlisting}
Man kann auch mehrere Ausnahmetypen mit \lpy{as} benennen.
\begin{lstlisting}
try:
  a = 1/0
except (ZeroDivisionError, ValueError) as ausnahme:
  print('Ausnahme vom Typ:', type(ausnahme), 'mit Wert:', ausnahme )
\end{lstlisting}

Es gibt Situationen, da möchte man einen Programmabschnitt ausführen der Ausnahmen auslösen kann und diese dann folgendermaßen behandeln.
Wird eine (behandelnare) Ausnahme ausgelöst, so soll sie behandelt werden.
Wird jedoch keine Ausnahme ausgelöst, dann (und nur dann) soll ein weiterer Programmabschnitt ausgeführt werden.
Das funktioniert mit der \lpy{try-except-else} Konstruktion:
\begin{lstlisting}
# Programmfluss

try:
  # Abschnitt A
except ...: # Zu behandelnden Ausnahmetyp festlegen
  # Ausnahmen behandeln
... # Weitere Ausnahmebehandlungen
else: # Wird ausgefuehrt genau dann wenn Absch. A keine Ausnahme ausloest
  # Abschnitt B

# Hier geht der normale Programmfluss weiter
\end{lstlisting}
Der obige Code verhält sich genau wie der nachfolgende Code:
\begin{lstlisting}
try:
  ausnahme_aufgetreten = False
  # Abschnitt A
except ...: # Zu behandelnden Ausnahmetyp festlegen
  ausnahme_aufgetreten = True
  # Ausnahmen behandeln
... # Weitere Ausnahmebehandlungen, die ausnahme_aufgetreten=True setzen
if ausnahme_aufgetreten == False:
  # Abschnitt B
\end{lstlisting}

An dieser Stelle kann man bereits verstehen, warum Ausnahmen ein gutes Konzept sind.
Durch die Aufteilung in einen von \lpy{try} eingeleiteten Block schreibt man den auszuführenden Programmcode und
teilt die Ausnahmebehandlung in die von \lpy{except} eingeleiteten Blöcken ein.
Das führt zu wesentlich übersichtlicherem Code.

Hier noch ein Beispiel:
\begin{lstlisting}[escapechar=|]
import math
def ganzzahlige_wurzel( x ):
  """Diese Funktion zieht die ganzzahlige Wurzel."""
  y = 0 # Wir definieren eine Variable y, die wir am Ende zurueckgeben,
        # unabhaengig davon, ob eine Ausnahme behandelt werden muss oder
        # nicht
  try:  # Versuche die ganzzahlige Wurzel zu siehen
    y = int(math.sqrt(x)) |\label{zeile:sqrt_loest_fehler_aus}|
  except TypeError as ausnahme:  # Ausnahme: x hat den falschen Typ
    print('Falscher Typ:', ausnahme)
  except ValueError as ausnahme: # Ausnahme: x ist negativ.
    print('Falscher Wert:', ausnahme)
  except:                        # Ausnahme: andere Ausnahme
    print('Anderer, komischer Fehler...')
  return y |\label{zeile:y_ist_evtl_nicht_definiert}|

ganzzahlige_wurzel('Suppe') # Druckt: 'Falscher Typ: a float is required'
ganzzahlige_wurzel(-3)      # Druckt: 'Falscher Wert: math domain error'
ganzzahlige_wurzel(6)       # Druckt nix.
\end{lstlisting}
Man beachte, dass die Zeile \lpy{y=0} nicht vergessen werden darf, denn sonst kann es passieren, dass \lpy{y} in Zeile~\ref{zeile:y_ist_evtl_nicht_definiert} nicht definiert ist.
Falls \lpy{math.sqrt(x)} eine Ausnahme auslöst, wird der Programmfluss in Zeile~\ref{zeile:sqrt_loest_fehler_aus} unterbrochen und die Ausnahme behandelt.
Dass heißt, beim Auslösen einer Ausnahme wird \lpy{y} in dieser Zeile weder definiert noch auf ein Objekt gesetzt und kann insbesondere in Zeile~\ref{zeile:y_ist_evtl_nicht_definiert} nicht zurückgegeben werden.

\subsection{Ausnahmen auslösen und weitergeben}
\label{section:ausnahmen:ausnahmen_ausloesen}
Wir wollen nun verstehen, wie man Ausnahmen auslöst und wie Ausnahmen weitergegeben werden.
Beides geschieht mit \lpy{raise}.

Man löst eine Ausnahme vom Typ \lpy{Ausnahmetyp} mit dem beschreibenden String \lpy{ausnahmestring} durch folgendes Statement aus:
\begin{lstlisting}
raise Ausnahmetyp(ausnahmestring)
\end{lstlisting}
Wenn man sein Programm sehr trotzig abbrechen möchte kann man das also so tun:
\begin{lstlisting}
raise RuntimeError("Mir ist jetzt alles egal!")
\end{lstlisting}
Im folgenden Beispiel definieren wir eine Funktion, die nur mit Strings und Ganzzahlen umgehen möchte:
\begin{lstlisting}
def ich_mag_nur_strings_und_ganzzahlen( x ):
  """Diese Funktion mag nur Strings und Ganzzahlen."""
  if not (type(x) is int or type(x) is str):
    raise ValueError("Ich mag nur Strings und Ganzzahlen")
  print("Ich mag dich: '{}'".format(x))
\end{lstlisting}

Nun klären wir die Frage:
\begin{center}
  Wem wird eine Ausnahme zum Behandeln eigentlich übergeben?
\end{center}
Zuerst führen wir den sogenannten \emph{Call Tree} eines Programms ein.
Der Call Tree ist bei sequenziellen Programmen das ohne ausgelöse Ausnahmen auskommt immer ein gewurzelter Baum.
Die Wurzel $v$ ist die main-Funktion (oder genauer das main-Modul).
Wird eine Funktion \lpy{f} aufgerufen, definiert das eine Kante mit einem neuen Knoten in unserem Baum, den wir hier der Einfachheit halber $v(f)$ nennen.
Die in \lpy{f} aufgerufenen Funktionen, sagen wir \lpy{g}, \lpy{h} oder vielleicht sogar \lpy{f}, definieren dann neue Kanten zu neuen Knoten, sagen wir $v(g,f)$, $v(h,f)$, $v(f,f)$.
Wird eine Funktion \lpy{k} mehrere Male hintereinander aufgerufen, erstellen wir für jeden Aufruf eine neue Kante mit einem neuen Endknoten.

Per Konstruktion entspricht jede Kante einem Funktionsaufruf.
Braucht man $k$ Kanten um von der Wurzel $v$ zu einem anderen Knoten $w$ zu kommen, bedeutet dass wir $k$ ineinander verschachtelte Funktionsaufrufe benötigt haben.

Also entsteht zur Programmlaufzeit ein gewurzelter Baum.
Zu einem festen gewählen Zeitpunkt, während das Programm läuft, gibt es immer einen Knoten, der zuletzt erstellt wurde.
Diesen Knoten nennen wir ``aktiv''.
Der aktive Knoten entspricht dem Programmabschnitt, in dem wir uns (zur festgelegten Laufzeit) befinden.

Nun können wir leicht verstehen, wem eine Ausnahme zum Behandeln übergeben wird.
Dazu nehmen wir uns den Call Tree das laufende Programm zur Hilfe.
Wird eine Ausnahme in einem Programmabschnitt ausgelöst, so entspricht dieser Programmabschnitt dem aktiven Knoten.
Die Ausnahme kann dann vom aktiven Knoten behandelt werden, falls der Programmabschnitt im einem \lpy{try-except} Konstrukt liegt.
Außerdem muss die Ausnahme in ihrem \lpy{except} Abschnitt behandelt werden.
Ist mindestens eins von beiden nicht der Fall, wird die Ausnahme an den Knoten über dem aktiven Knoten zum Behandeln weitergeben.
Dies wird so lange fortgeführt, bis die Ausnahme behandelt wurde oder bis sie schlussendlich auch in der Wurzel nicht behandelt wurde.
Im letzteren Fall erklärt der \Python-Interpreter wo die Ausnahme aufgetreten ist und beendet das Programm.

Man kann sogar Ausnahmen behandeln und nach der Behanlung die Ausnahme mit \lpy{raise} an den darüberliegenden Knoten weitergeben.
Das besprechen wir an dieser Stelle aber nicht ausführlicher.

Schauen wir uns das ganze mal anhand eines Beispiels an:
\begin{lstlisting}
def drucke_float_aus(zahl): # Druckt float aus oder loest Ausnahme aus
  if type(zahl) is not float:
    raise TypeError('Ich will float und sonst nichts.')
  else:
    print('{}, ich liebe dich'.format(zahl))

def eins_durch_null(): # Teilt durch Null
  return (1.0/0.0)

def behandle_ausnahmen_nicht(): # Behandelt Ausnahmen nicht
  eins_durch_null()             # Ausnahme wird nicht behandelt

def behandle_ausnahmen():
  try:
    drucke_float_aus('Hallo!')
  except:
    print('Ausnahme an Stelle 1 wurde ausgeloest')
  
  try:
    drucke_float_aus(eins_durch_null())
  except:
    print('Ausnahme an Stelle 2 wurde ausgeloest')
  
  try:
    behandle_ausnahmen_nicht()
  except:
    print('Ausnahme an Stelle 3 wurde ausgeloest')
\end{lstlisting}
Beim Aufruf der Funktion \lpy{behandle_ausnahmen()} versuchen wir zunächst, die Funktion \lpy{drucke_float_aus('Hallo!')} auszuführen.
Diese löst eine Ausnahme aus, die wir in \lpy{behandle_ausnahmen()} abfangen.
In der Ausnahmebehandlung drucken wir:
\begin{center}
\lpy{'Ausnahme an Stelle 1 wurde ausgeloest'} 
\end{center}
Nun versuchen wir \lpy{drucke_float_aus(eins_durch_null())} auszuführen.
Dabei wird zuerst die innere Funktion, also \lpy{eins_durch_null()} ausgeführt.
Diese löst eine Ausnahme aus.
Insbesondere gibt die Funktion \lpy{eins_durch_null()} nichts zurück denn die normale Ausführung wird unterbrochen und wir machen sofort mit der Fehlerbehandlung weiter.
In der Ausnahmebehandlung drucken wir:
\begin{center}
\lpy{'Ausnahme an Stelle 2 wurde ausgeloest'}
\end{center}
Nun versuchen wir \lpy{behandle_ausnahmen_nicht()} auszuführen.
Diese Funktion ruft \lpy{eins_durch_null()} aus.
In unserem Call Tree haben wir momentan also einen Weg von \lpy{behandle_ausnahmen} über \lpy{behandle_ausnahmen_nicht} zu \lpy{eins_durch_null}.
Die Funktion \lpy{eins_durch_null} löst eine Ausnahme aus.
Diese wird (im Call Tree) an die Funktion \lpy{behandle_ausnahmen_nicht} weitergegeben und dort nicht behandelt.
Also wird sie weitergegeben an \lpy{behandle_ausnahmen}.
Dort wird sie behandelt.
In der Ausnahmebehandlung drucken wir:
\begin{center}
\lpy{'Ausnahme an Stelle 3 wurde ausgeloest'}
\end{center}

\subsection{Bereits definierte und eigens definierte Ausnahmen}
\label{section:ausnahmen:definierte_und_eigene_ausnahmen}

Wir listen nachfolgend einige Ausnahmentypen auf.
Diese Liste ist nicht vollständig und wir verweisen den interessierten Leser auf \cite[Library: Exceptions]{Python3}.
Außerdem sind einige der Ausnahmen voneinander abgeleitet.
Da wir in diesem Kurs ``abgeleitete Klassen'' nicht behandelt haben, gehen wir hier weiter nicht darauf ein und verweisen nocheinmal auf \cite[Library: Exceptions]{Python3}.
\begin{lstlisting}
Exception          # Allgemeine Ausnahme.

FloatingPointError # Gleitkommafehler
OverflowError      # Overflowfehler
ZeroDivisionError  # Du hast durch Null geteilt
ImportError        # Fehler beim Importieren
IndexError         # Falscher Index beim Sequenzzugriff
KeyError           # Falscher Key beim Verzeichniszugriff
MemoryError        # Wir haben nicht genug Speicher
FileExistsError    # Datei existiert (beim erstellen einer neuen Datei)
FileNotFoundError  # Datei nicht gefunden (beim oeffnen einer Datei)
IsADirectoryError  # Dateioperation auf Ordner angewendet
NotADirectoryError # Ordneroperation auf Datei angewendet
PermissionError    # Unzureichende Zugriffsrechte (bei Dateien / Ordnern)
RuntimeError       # Laufzeitfehler (wird vom Programmierer ausgeloest)
NotImplementedError # Funktion ist nicht implementiert
SyntaxError        # Syntaxfehler
IndentationError   # Syntaxfehler: Falsch eingerueckt
SystemError        # Komischer Systemfehler
TypeError          # Falscher Typ
ValueError         # Falscher Wert


Warning            # Allgemeine Warnung

DeprecationWarning # Warnung: Veraltete Funktion / Klasse wird verwendet
ImportWarning      # Warnung beim Importieren
\end{lstlisting}

Da wir in diesem Kurs  ``abgeleitete Klassen'' nicht behandelt haben, erklären wir hier nur, wie man eigene Ausnahmen erstellt, aber nicht wie das im Detail funktioniert.
Eine eigenen Ausnahmetyp erstellt man so:
\begin{lstlisting}
class meine_ausnahme(Exception): pass
\end{lstlisting}
Dann kann man im \lpy{try-except} Konstrukt seinen eigenen Ausnahmetyp verwenden.
Hier ein Beispiel:
\begin{lstlisting}
import math

class NichtQuadratisch(Exception) : pass
class KeineReellenLoesungen(Exception) : pass

def loese_quad_gl(a,b,c):
  if a == 0:
    ausnbeschr = 'Nicht quadratisch a={}, b={}, c={}'.format(a,b,c)
    ausnahme = NichtQuadratisch(ausnbeschr)
    raise ausnahme
  if b**2 - 4.0*a*c < 0:
    ausnbeschr = 'Keine reellen Loesungen a={}, b={}, c={}'.format(a,b,c)
    ausnahme = KeineReellenLoesungen(ausnbeschr)
    raise ausnahme
  x1 = ( -b + math.sqrt( b**2 - 4.0*a*c) ) / (2.0*a)
  x2 = ( -b - math.sqrt( b**2 - 4.0*a*c) ) / (2.0*a)
  return x1, x2

try:
  loese_quad_gl(1,0,-1) # (1,0), (-1,0)
  loese_quad_gl(1,0,1)  # Ausnahme: Keine reellen Loesungen a=1, b=0, c=1
  loese_quad_gl(0,0,1)  # Ausnahme: Nicht quadratisch a=0, b=0, c=0
except (NichtQuadratisch, KeineReellenLoesungen) as ausnahme:
  print('Meine Ausnahme', ausnahme)
\end{lstlisting}

Als letztes erklären wir in diesem Abschnitt wie man in einem \lpy{try-except} Konstrukt die ``restlichen Ausnahmentypen'', also solche die mit \lpy{except:} behandelt werden, auch mit \lpy{as} benennen kann.
Warum das funktioniet erklären wir an dieser Stelle nicht, da man ``abgeleitete Klassen'' für die Erklärung braucht, die haben wir in diesem Kurs aber nicht behandeln können.
Um also ``die restlichen Ausnahmen'' mit \lpy{as} zu benennen verwendet man die Zeile \lpy{except Exception as ausnahme:}, auch wenn \lpy{Exception} nicht alle sondern nur alle sinnvollen Ausnahmen zusammenfasst.
Zum Beispiel ist die Ausnahme \lpy{SystemExit} nicht vom Typ \lpy{Exception}.
Die Ausnahmebehandlung soll so aussehen:
\begin{lstlisting}
try:
  # Programmabschnitt der Ausnahmen ausloesen kann
except Ausnahmetyp_1 as ausnahme:
  # Ausnahmetyp 1 behandeln
...
except Exception as ausnahme:
  # Alle anderen sinnvollen Ausnahmen behandeln
except:
  raise # Lass das mal lieber den Python Interpreter handhaben.
# Hier geht der normale Programmfluss weiter
\end{lstlisting}



\newpage

\section{Crashkurs}
\label{section:crashkurs}
Wir beginnen mit einem Überblick der grundlegende Bausteine.
In den nachfolgenden Kapiteln besprechen wir einige der Bausteine nochmal im Detail.


\subsection{Kommentare}
\label{section:crashkurs:kommentare}
Kommentare in \Python beginnen mit \lpy{#} und enden am Ende der Zeile.
Sie verhalten sich also ganz genau wie Kommentare in \CC, die mit \lcpp{//} beginnen%
\footnote{Die Kommentare \lcpp{//} wurden in \CNeunundneunzig eingeführt, siehe \cite{C99}.}.
Gut platzierte Kommentare dürfen in keinem Programm fehlen.
Ein schlecht kommentiertes Programm ist ein Anzeichen dafür, dass es von einem schlechten Programmierer geschrieben wurde.


\subsection{Print}
\label{section:crashkurs:print}
So gut wie alle Objekte können durch die \Python Funktion \lpy{print} auf der Standardausgabe ausgegeben werden.
Das liegt daran, dass die meisten Objekte Klassen sind und die Funktion \lpy{__str__} implementieren (auch wenn wir das jetzt noch nicht verstehen;
siehe Abschnitt \ref{section:klassen:spezielle_funktionen}).
Die Funktion \lpy{print} ist für \PythonZwei und \PythonDrei verschieden.
\begin{lstlisting}
print 'Hallo', 'Welt'   # Python2
print ('Hallo', 'Welt') # Python3
\end{lstlisting}
In diesem Skript nutzen wir die von \PythonDrei bereitgestellte Version von \lpy{print}.
Will man in \PythonZwei die \lpy{print}-Funktion aus \PythonDrei verwenden, kann man folgende Zeile am Anfang seines Skript einfügen.
\begin{lstlisting}
from __future__ import print_function # Nutze Python3-print in Python2
\end{lstlisting}


\subsection{Typ und Identität}
\label{section:crashkurs:typ_und_id}
Um die Identiät eines Objekts \lpy{x} zu bestimmen, nutzen wir die Funktion \lpy{id(x)}.
Sie gibt einen String zurück, den wir mit \lpy{print} ausdrucken können.
Ganz ähnlich bekommen wir den Typ eines Objekts mit \lpy{type(x)}.


\subsection{Zahlen}
\label{section:crashkurs:zahlen}
In \Python gibt es die Zahlentypen \lpy{int}, \lpy{long}, \lpy{float} und \lpy{complex}.
Sie verhalten sich (fast) genauso wie die aus \CC bekannten Typen \lcpp{int}, \lcpp{long}, \lcpp{float} und \lcpp{complex}%
\footnote{Der Typ \lpy{complex} wurde in \CNeunundneunzig eingeführt, siehe \cite{C99}.}.
Man erstellt ein Objekt vom Typ \lpy{int}, durch Angabe einer ganzen Zahl oder durch Aufrufen der Funktion \lpy{int}.
Ähnlich erstellt man ein Objekt vom Typ \lpy{float}, durch Angabe einer Kommazahl oder durch Aufrufen der Funktion \lpy{float}.
\begin{lstlisting}
print( 2 )        # 2    typ: int
print( int(2.1) ) # 2    typ: int
print( 2.0 )      # 2.0  typ: float
print( float(2) ) # 2.0  typ: float
\end{lstlisting}
Man kann zwei verschiedene oder gleiche Zahlentypen mit mathematischen Operatoren verbinden.
Das Ergebnis hat den ``genaueren'' Typ.
Des Weiteren beschreibt \lpy{//} die ganzzahlige Division ohne Rest.
\begin{lstlisting}
print( 2 + 3.3 )  # 5.3  typ: int +  float -> float
print( 5 // 2 )   # 2    typ: int // int   -> int
print( 5.3 // 2 ) # 2.0  typ: int // float -> float
\end{lstlisting}
Die Division von Ganzzahlen unterscheidet sich in \PythonZwei und \PythonDrei.
\begin{lstlisting}
4 /  2 # Python2: 2 vom Typ int;   Python3: 2.0 vom Typ float;
4 // 2 # Python2: 2 vom Typ int;   Python3: 2   vom Typ int;
\end{lstlisting}


\subsection{Strings}
\label{section:crashkurs:strings}
Strings die eine Zeile lang sind, werden entweder mit \lpy{'} oder mit \lstinline[style=PyInline]|"| begonnen und beendet.
Soll ein String mehre Zeilen lang sein, kann man ihn mit \lstinline[style=PyInline]|"""| beginnen und beenden.
Genau wie in \CC gibt es gewisse Zeichen, die nicht direkt geschrieben werden können und mit \lstinline[language=C++,style=CPP]|\| beginnen müssen (zum Beispiel werden neue Zeilen durch \lstinline[language=C++,style=CPP]|\n| ausgedrückt).
Folgender Code erstellt den String ``Hallo[Neue Zeile]Welt''.
\begin{lstlisting}
print( 'Hallo\nWelt' ) # Hallo[neue Zeile]Welt
\end{lstlisting}
Es gibt eine große Auswahl an Funktionen, um aus einem gegebenen String einen neuen zu erstellen.
Beispielsweise verknüpft man zwei Strings mit \lpy{+} und die Funktion \lpy{lower} ersetzt alle Großbuchstaben durch Kleinbuchstaben.
\begin{lstlisting}
print( 'Hallo' + ' ' + 'Welt' ) # Hallo Welt
print( 'Hallo Welt'.lower() )   # hallo welt
\end{lstlisting}


\subsection{Bedingungen}
\label{section:crashkurs:bedingungen}
In \Python gibt es genau zwei Objekte \lpy{True} und \lpy{False} von Typ \lpy{bool}.
(Fast) alle Typen können mit der Funktion \lpy{bool} in einen der beiden Objekte umgewandelt werden.
Den Kontrollfluss des Programms steuert man im einfachsten Fall wir folgt.
\begin{lstlisting}
if bedingung:
  ausdruck
  ...
\end{lstlisting}
Das heißt, wir brauchen zunächst eine Expression \lpy{bedingung}, die als \lpy{True} oder \lpy{False} interpretiert werden kann.
Wertet sie als \lpy{True} aus, so wird ein Block von Ausdrücken ausgeführt.
Anders als in \CC werden in \Python Blöcke nicht durch Klammern, sondern durch konsistente Einrückung definiert.
\begin{lstlisting}
x = 101//3+46//5
if x == 42:
  print('Die Antwort')
\end{lstlisting}
Da gut eingerückter Code (also Code, in dem jeder Block konsistent eingerückt ist) deutlich lesbarer sind als nicht eingerückter
Code, ist er in jedem Fall -- unabhängig von der Sprache -- ein Zeichen für eine gute Programmiererin. 

Da es jedoch trotzdem immer noch genug Leute gibt, die hässlichen Code produzieren, wird man in \Python dazu gezwungen, auf 
konsistente Einrückung zu achten: zu einem Block gehören genau die nachfolgenden Zeilen, die genauso weit eingerückt sind wie die 
erste Zeile des Blocks. Bei nicht konsistenter Einrückung gibt es Syntaxfehler. Dies macht es sehr schwer, hässlichen Code in
\Python zu schreiben.

Aus eigener Erfahrung können wir sagen, dass man sich als erfahrene \CC Programmiererin recht schnell an diese Konvention gewöhnen kann, 
wenn man auch schon vorher auf sauber geschriebene Programme geachtet hat.

Optional besteht eine Bedingung aus keinem oder mehreren \lpy{elif bedingung:}\footnote{sprich ``else if''} und keinem oder einem \lpy{else:}
Die Konstrukte \lpy{else:} und \lpy{elif bedingung:} in \Python verhalten sich wie die Konstrukte \lcpp{else} und \lcpp{else if} in \CC.
\begin{lstlisting}
x = 101//3+46//5               # x = ?
if x == 42:                    # Falls x den Wert 42 hat:
  print('Die Antwort')         # Drucke 'Die Antwort'
elif x == 84:                  # Sonst, falls x den Wert 84 hat:
  print('Zweimal die Antwort') # Drucke 'Zweimal die Antwort'
elif x == 106:                 # Sonst, falls x den Wert 106 hat:
  print('Dreimal die Antwort') # Drucke 'Dreimal die Antwort'
else:                          # In allen anderen Faellen:
  print('Versuchs nochmal')    # Drucke 'Versuchs nochmal'
\end{lstlisting}


\subsection{Listen}
\label{section:crashkurs:listen}
In \Python gibt es verschiedene ``Containertypen'' und hier besprechen wir den Typ \lpy{list}.
Eine Liste \lpy{thor} ist eine linear geordnete Menge von Variablen, auf die man mit \lpy{thor[0], thor[1], ..., thor[i]} zugreift.
Um auf Elemente von hinten zuzugreifen, sind negative Werte für \lpy{i} erlaubt.
\begin{lstlisting}
thor = [2, 'Hallo', 33.3, 'Welt', [] ] # Erzeugt eine Liste
print( thor[1], thor[-1] )             # Gibt folgendes aus: Hallo []
\end{lstlisting}
Die Länge einer Liste ist \lpy{len(thor)}.
Listen sind mutable, das heißt wir können sie verändern.
Mit \lpy{thor.append(x)} erweitert man die Liste \lpy{thor} um die Variable \lpy{x} und mit \lpy{thor.remove(x)} entfernt man die erste Variable mit dem Wert \lpy{x}.
Ob ein Objekt mit Wert \lpy{x} in \lpy{thor} enthalten ist, testet man mit \lpy{x in thor}.
Mit \lpy{thor+asgard} verknüpft man die beiden Listen \lpy{thor} und \lpy{asgard}.


\subsection{Schleifen}
\label{section:crashkurs:schleifen}
In \Python gibt es zwei Arten von Schleifen.
Die \lpy{while}-Schleife führt einen Codeblock aus, solange die vor der Ausführung des Blocks geprüfte Bedingung als \lpy{True} auswertet.
\begin{lstlisting}
while bedingung:
  ausdruck
  ...
\end{lstlisting}
Da uns dieses Konzept aus \CC sehr vertraut ist, brauchen wir hier kein Beispiel.

Die \lpy{for}-Schleife führt einen Codeblock für jedes Element einer gegebenen Sequenz (siehe Abschnitt \ref{section:std_data_types:sequenzen}) aus.
Eine Sequenz kann zum Beispiel eine Liste oder ein String sein.
\begin{lstlisting}
for i in s: # der Ausdruck s muss als Sequenz interpretiert werden koennen
  ausdruck
  ...
\end{lstlisting}
Dabei referenziert die Laufvariable \lpy{i} jedes Element der Sequenz \lpy{s} nacheinander einmal. Durch Verwendung von \lpy{i} innerhalb 
des folgenden Codeblocks kann 
dann auf den Wert des jeweiligen Objekts zugegriffen werden. Nach Ausführung des Codeblocks der \lpy{for}-Schleife referenziert 
\lpy{i} dann das nächste Element von \lpy{s}, bis das Ende der Sequenz erreicht ist. 

Dies wird durch das folgende Beispiel veranschaulicht:
\begin{lstlisting}
s = 'Hallo Welt' # Die Sequenz 'Hallo Welt'
for i in s:      # Nimm alle i = H,a,l,l,o, ,W,e,l,t
  print(i)       # Drucke i aus
\end{lstlisting}
Hier ist unsere Sequenz \lpy{s} ein String mit dem Wert \lpy{Hallo Welt}. In der \lpy{for}-Schleife nimmt die Laufvariable nacheinander
die Werte der einzelnen Zeichen von \lpy{s}, also \lpy{H,a,l,} \lpy{l,o, ,W,e,l} und \lpy{t}. Im zur Schleife gehörenden Block werden 
diese Zeichen dann einzeln ausgegeben.

Wenn der Codeblock für eine aufsteigende Folge von Zahlen \lpy{from, from+1, ..., to-1} ausgeführt werden soll, nimmt man die Sequenz \lpy{range(from, to)}.
Wichtig ist, dass wir das halboffene Interval betrachten, also nur bis \lpy{to-1} und nicht bis \lpy{to} gehen.
Damit verhält sich der \Python-Code \lpy{for i in range(from, to):} ganz genau wie der \CC-Code \lstinline[language=C++,style=CPPinline]|for(i = from; i < to; ++i) { ... }|.
Hier ein Beispiel
\begin{lstlisting}
sum = 0              # Setze sum auf 0
for i in range(1,5): # Nimm alle i = 1, 2, 3, 4
  sum += 3*i         # Addiere 3*i zu sum
print( sum )         # Drucke sum aus
\end{lstlisting}

Will man eine absteigende Folge ganzer Zahlen, oder allgemeiner eine Folge ganzer Zahlen mit Schrittweite \lpy{step > 0} oder \lpy{step < 0} haben,
nimmt man die Sequenz \lpy{range(from,to,step)}.
Also können wir obiges Beispiel wie folgt umformulieren.
\begin{lstlisting}
sum = 0                  # Setze sum auf 0
for i in range(3,15, 3): # Nimm alle i = 3, 6, 9, 12
  sum += i               # Addiere i zu sum
print( sum )             # Drucke sum aus
\end{lstlisting}


\subsection{Funktionen}
\label{section:crashkurs:funktionen}
Meistens will man sich wiederholende Codeblöcke auslagern und natürlich kann man auch in \Python Funktionen definieren.
Jede Funktionsdefinition erstellt ein Objekt, welches der Funktion im Speicher entspricht.
Außerdem wird eine Variable definiert, welche auf die Funktion im Speicher referenziert.
\begin{lstlisting}
def funktionsname(parameter):
  ausdruck
  ...
\end{lstlisting}
Die Variable ist hier \lpy{funktionsname} und zeigt auf das Objekt, welches der definierten Funktion entspricht.

Anders als in \CC hat eine Funktion keine Signatur und keinen definierbaren Rückgabetyp.
Mit \lpy{return} können kein, ein oder mehrere Objekte zurückgegeben werden.
Wird kein Objekt zurückgegeben, so ist die Rückgabe automatisch \lpy{None}.
Wird ein Objekt \lpy{x} zurückgegeben, so ist der Rückgabetyp automatisch \lpy{type(x)}.
Werden mehrere Objekte zurückgegeben, so werden diese automatisch in einem \lpy{tuple} zusammengefasst, der Rückgabetyp ist also \lpy{tuple}.
Ein Tuple ist eine Liste, die nicht geändert werden kann (siehe Abschnitt \ref{section:std_data_types:sequenzen}).
\begin{lstlisting}
def f(i):
  if i == 1:
    return 1           # Rueckgabetyp: int,   Rueckgabewert: 1
  elif i == 2:
    return 'Hallo', 44 # Rueckgabetyp: tuple, Rueckgabewert: ('Hallo', 44)

print( type( f ) )                 # <class 'function'>
print( type( f(1) ) )              # <class 'int'>
print( type( f(2) ) )              # <class 'tuple'>
print( type( f('Gurkenwasser') ) ) # <class 'NoneType'>
\end{lstlisting}
Also ist das von \lpy{f(3)} oder auch \lpy{f('Gurkenwasser')} zurückgegebene Objekt \lpy{None}.

Beim Funktionsausruf übergibt man die Parameter entweder in der Reihenfolge wie sie in der Funktionsdefinition spezifiziert sind, oder man übergibt \lpy{parametername=ausdruck}.
Außerdem kann man den Argumenten einer Funktion Standardwerte übergeben.
Beim Funktionsausruf muss man die Argumente mit Standardwerten nicht angeben, kann es aber.
\begin{lstlisting}
def plus(a,b=1):
  return a+b

plus(4,5)      # = 9
plus(b=5, a=4) # = 9
plus(3)        # = 4
\end{lstlisting}


\subsection{Module}
\label{section:crashkurs:module}
Was \Python wirklich mächtig macht, ist nicht die Sprache an sich, sondern die schier unendliche Fülle an Paketen die \Python-Programmierer bereit stellen.
Ein Modul ist so etwas wie eine ``shared library'' in \CC.
Man kann sie importieren und dann nutzen, die Implementierung haben andere bereits übernommen (siehe Abbildung \ref{figure:xkcd_python}).
\begin{figure}[ht]
  \centering
  \includegraphics[width=0.5\textwidth]{pictures/xkcd_python.png}
  \caption{\label{figure:xkcd_python}``Python'' by Randall Munroe \cite{Munroe_python}}
\end{figure}

Ein Modul bindet man mit \lpy{import modulname} in sein Programm ein.
Alle Objekttypen (Klassen, Funktionen und andere Objekte) die \lpy{modulname} bereitstellt, können durch \lpy{modulname.oname} erreicht werden.
\begin{lstlisting}
import math
print( math.sin(math.pi/3.0) )
\end{lstlisting}
Wenn man aus \lpy{modulname} bestimmte Objekte einbinden möchte, nutzt man folgende Codezeile.
\begin{lstlisting}
from modulname import oname1, oname2, ...
\end{lstlisting}
Jetzt kann man auf die Objekte direkt (also ohne das Präfix \lpy{modulname.}) zugreifen.
Wir hatten bereits gesehen, wie man die \PythonDrei Printfunktion in \PythonZwei einbinden kann.
\begin{lstlisting}
from __future__ import print_function
\end{lstlisting}
Hier noch ein Beispiel.
\begin{lstlisting}
from math import sin, cos, pi
print( sin(pi/3.0), cos(pi/3.0) )
\end{lstlisting}

An dieser Stelle machen wir noch auf Abschnitt \ref{section:module:empfohlene_module} indem wir eine Hand voll \Python-Pakete erwähnen, die wir oft benutzen.


\subsection{Workflow}
\label{section:crashkurs:workflow}
Wir arbeiten recht erfolgreich mit folgendem Workflow.

Beginne ein Projekt in \Python.
Solange du noch nicht fertig bist:
Wenn eine Funktion zu aufwendig zu implementieren ist, suche nach einer bereits vorhandene \Python Bibliothek (es gibt bestimmt eine).
Wenn eine Funktion für dein Projekt zu langsam ist, schreibe sie in \CC und binde sie als Modul ein.
Wie genau letzteres geht, lernen wir noch.


\newpage

\section{Crashkurs}
\label{section:crashkurs}
Wir beginnen mit einem Überblick der grundlegende Bausteine.
In den nachfolgenden Kapiteln besprechen wir einige der Bausteine nochmal im Detail.


\subsection{Kommentare}
\label{section:crashkurs:kommentare}
Kommentare in \Python beginnen mit \lpy{#} und enden am Ende der Zeile.
Sie verhalten sich also ganz genau wie Kommentare in \CC, die mit \lcpp{//} beginnen%
\footnote{Die Kommentare \lcpp{//} wurden in \CNeunundneunzig eingeführt, siehe \cite{C99}.}.
Gut platzierte Kommentare dürfen in keinem Programm fehlen.
Ein schlecht kommentiertes Programm ist ein Anzeichen dafür, dass es von einem schlechten Programmierer geschrieben wurde.


\subsection{Print}
\label{section:crashkurs:print}
So gut wie alle Objekte können durch die \Python Funktion \lpy{print} auf der Standardausgabe ausgegeben werden.
Das liegt daran, dass die meisten Objekte Klassen sind und die Funktion \lpy{__str__} implementieren (auch wenn wir das jetzt noch nicht verstehen;
siehe Abschnitt \ref{section:klassen:spezielle_funktionen}).
Die Funktion \lpy{print} ist für \PythonZwei und \PythonDrei verschieden.
\begin{lstlisting}
print 'Hallo', 'Welt'   # Python2
print ('Hallo', 'Welt') # Python3
\end{lstlisting}
In diesem Skript nutzen wir die von \PythonDrei bereitgestellte Version von \lpy{print}.
Will man in \PythonZwei die \lpy{print}-Funktion aus \PythonDrei verwenden, kann man folgende Zeile am Anfang seines Skript einfügen.
\begin{lstlisting}
from __future__ import print_function # Nutze Python3-print in Python2
\end{lstlisting}


\subsection{Typ und Identität}
\label{section:crashkurs:typ_und_id}
Um die Identiät eines Objekts \lpy{x} zu bestimmen, nutzen wir die Funktion \lpy{id(x)}.
Sie gibt einen String zurück, den wir mit \lpy{print} ausdrucken können.
Ganz ähnlich bekommen wir den Typ eines Objekts mit \lpy{type(x)}.


\subsection{Zahlen}
\label{section:crashkurs:zahlen}
In \Python gibt es die Zahlentypen \lpy{int}, \lpy{long}, \lpy{float} und \lpy{complex}.
Sie verhalten sich (fast) genauso wie die aus \CC bekannten Typen \lcpp{int}, \lcpp{long}, \lcpp{float} und \lcpp{complex}%
\footnote{Der Typ \lpy{complex} wurde in \CNeunundneunzig eingeführt, siehe \cite{C99}.}.
Man erstellt ein Objekt vom Typ \lpy{int}, durch Angabe einer ganzen Zahl oder durch Aufrufen der Funktion \lpy{int}.
Ähnlich erstellt man ein Objekt vom Typ \lpy{float}, durch Angabe einer Kommazahl oder durch Aufrufen der Funktion \lpy{float}.
\begin{lstlisting}
print( 2 )        # 2    typ: int
print( int(2.1) ) # 2    typ: int
print( 2.0 )      # 2.0  typ: float
print( float(2) ) # 2.0  typ: float
\end{lstlisting}
Man kann zwei verschiedene oder gleiche Zahlentypen mit mathematischen Operatoren verbinden.
Das Ergebnis hat den ``genaueren'' Typ.
Des Weiteren beschreibt \lpy{//} die ganzzahlige Division ohne Rest.
\begin{lstlisting}
print( 2 + 3.3 )  # 5.3  typ: int +  float -> float
print( 5 // 2 )   # 2    typ: int // int   -> int
print( 5.3 // 2 ) # 2.0  typ: int // float -> float
\end{lstlisting}
Die Division von Ganzzahlen unterscheidet sich in \PythonZwei und \PythonDrei.
\begin{lstlisting}
4 /  2 # Python2: 2 vom Typ int;   Python3: 2.0 vom Typ float;
4 // 2 # Python2: 2 vom Typ int;   Python3: 2   vom Typ int;
\end{lstlisting}


\subsection{Strings}
\label{section:crashkurs:strings}
Strings die eine Zeile lang sind, werden entweder mit \lpy{'} oder mit \lstinline[style=PyInline]|"| begonnen und beendet.
Soll ein String mehre Zeilen lang sein, kann man ihn mit \lstinline[style=PyInline]|"""| beginnen und beenden.
Genau wie in \CC gibt es gewisse Zeichen, die nicht direkt geschrieben werden können und mit \lstinline[language=C++,style=CPP]|\| beginnen müssen (zum Beispiel werden neue Zeilen durch \lstinline[language=C++,style=CPP]|\n| ausgedrückt).
Folgender Code erstellt den String ``Hallo[Neue Zeile]Welt''.
\begin{lstlisting}
print( 'Hallo\nWelt' ) # Hallo[neue Zeile]Welt
\end{lstlisting}
Es gibt eine große Auswahl an Funktionen, um aus einem gegebenen String einen neuen zu erstellen.
Beispielsweise verknüpft man zwei Strings mit \lpy{+} und die Funktion \lpy{lower} ersetzt alle Großbuchstaben durch Kleinbuchstaben.
\begin{lstlisting}
print( 'Hallo' + ' ' + 'Welt' ) # Hallo Welt
print( 'Hallo Welt'.lower() )   # hallo welt
\end{lstlisting}


\subsection{Bedingungen}
\label{section:crashkurs:bedingungen}
In \Python gibt es genau zwei Objekte \lpy{True} und \lpy{False} von Typ \lpy{bool}.
(Fast) alle Typen können mit der Funktion \lpy{bool} in einen der beiden Objekte umgewandelt werden.
Den Kontrollfluss des Programms steuert man im einfachsten Fall wir folgt.
\begin{lstlisting}
if bedingung:
  ausdruck
  ...
\end{lstlisting}
Das heißt, wir brauchen zunächst eine Expression \lpy{bedingung}, die als \lpy{True} oder \lpy{False} interpretiert werden kann.
Wertet sie als \lpy{True} aus, so wird ein Block von Ausdrücken ausgeführt.
Anders als in \CC werden in \Python Blöcke nicht durch Klammern, sondern durch konsistente Einrückung definiert.
\begin{lstlisting}
x = 101//3+46//5
if x == 42:
  print('Die Antwort')
\end{lstlisting}
Da gut eingerückter Code (also Code, in dem jeder Block konsistent eingerückt ist) deutlich lesbarer sind als nicht eingerückter
Code, ist er in jedem Fall -- unabhängig von der Sprache -- ein Zeichen für eine gute Programmiererin. 

Da es jedoch trotzdem immer noch genug Leute gibt, die hässlichen Code produzieren, wird man in \Python dazu gezwungen, auf 
konsistente Einrückung zu achten: zu einem Block gehören genau die nachfolgenden Zeilen, die genauso weit eingerückt sind wie die 
erste Zeile des Blocks. Bei nicht konsistenter Einrückung gibt es Syntaxfehler. Dies macht es sehr schwer, hässlichen Code in
\Python zu schreiben.

Aus eigener Erfahrung können wir sagen, dass man sich als erfahrene \CC Programmiererin recht schnell an diese Konvention gewöhnen kann, 
wenn man auch schon vorher auf sauber geschriebene Programme geachtet hat.

Optional besteht eine Bedingung aus keinem oder mehreren \lpy{elif bedingung:}\footnote{sprich ``else if''} und keinem oder einem \lpy{else:}
Die Konstrukte \lpy{else:} und \lpy{elif bedingung:} in \Python verhalten sich wie die Konstrukte \lcpp{else} und \lcpp{else if} in \CC.
\begin{lstlisting}
x = 101//3+46//5               # x = ?
if x == 42:                    # Falls x den Wert 42 hat:
  print('Die Antwort')         # Drucke 'Die Antwort'
elif x == 84:                  # Sonst, falls x den Wert 84 hat:
  print('Zweimal die Antwort') # Drucke 'Zweimal die Antwort'
elif x == 106:                 # Sonst, falls x den Wert 106 hat:
  print('Dreimal die Antwort') # Drucke 'Dreimal die Antwort'
else:                          # In allen anderen Faellen:
  print('Versuchs nochmal')    # Drucke 'Versuchs nochmal'
\end{lstlisting}


\subsection{Listen}
\label{section:crashkurs:listen}
In \Python gibt es verschiedene ``Containertypen'' und hier besprechen wir den Typ \lpy{list}.
Eine Liste \lpy{thor} ist eine linear geordnete Menge von Variablen, auf die man mit \lpy{thor[0], thor[1], ..., thor[i]} zugreift.
Um auf Elemente von hinten zuzugreifen, sind negative Werte für \lpy{i} erlaubt.
\begin{lstlisting}
thor = [2, 'Hallo', 33.3, 'Welt', [] ] # Erzeugt eine Liste
print( thor[1], thor[-1] )             # Gibt folgendes aus: Hallo []
\end{lstlisting}
Die Länge einer Liste ist \lpy{len(thor)}.
Listen sind mutable, das heißt wir können sie verändern.
Mit \lpy{thor.append(x)} erweitert man die Liste \lpy{thor} um die Variable \lpy{x} und mit \lpy{thor.remove(x)} entfernt man die erste Variable mit dem Wert \lpy{x}.
Ob ein Objekt mit Wert \lpy{x} in \lpy{thor} enthalten ist, testet man mit \lpy{x in thor}.
Mit \lpy{thor+asgard} verknüpft man die beiden Listen \lpy{thor} und \lpy{asgard}.


\subsection{Schleifen}
\label{section:crashkurs:schleifen}
In \Python gibt es zwei Arten von Schleifen.
Die \lpy{while}-Schleife führt einen Codeblock aus, solange die vor der Ausführung des Blocks geprüfte Bedingung als \lpy{True} auswertet.
\begin{lstlisting}
while bedingung:
  ausdruck
  ...
\end{lstlisting}
Da uns dieses Konzept aus \CC sehr vertraut ist, brauchen wir hier kein Beispiel.

Die \lpy{for}-Schleife führt einen Codeblock für jedes Element einer gegebenen Sequenz (siehe Abschnitt \ref{section:std_data_types:sequenzen}) aus.
Eine Sequenz kann zum Beispiel eine Liste oder ein String sein.
\begin{lstlisting}
for i in s: # der Ausdruck s muss als Sequenz interpretiert werden koennen
  ausdruck
  ...
\end{lstlisting}
Dabei referenziert die Laufvariable \lpy{i} jedes Element der Sequenz \lpy{s} nacheinander einmal. Durch Verwendung von \lpy{i} innerhalb 
des folgenden Codeblocks kann 
dann auf den Wert des jeweiligen Objekts zugegriffen werden. Nach Ausführung des Codeblocks der \lpy{for}-Schleife referenziert 
\lpy{i} dann das nächste Element von \lpy{s}, bis das Ende der Sequenz erreicht ist. 

Dies wird durch das folgende Beispiel veranschaulicht:
\begin{lstlisting}
s = 'Hallo Welt' # Die Sequenz 'Hallo Welt'
for i in s:      # Nimm alle i = H,a,l,l,o, ,W,e,l,t
  print(i)       # Drucke i aus
\end{lstlisting}
Hier ist unsere Sequenz \lpy{s} ein String mit dem Wert \lpy{Hallo Welt}. In der \lpy{for}-Schleife nimmt die Laufvariable nacheinander
die Werte der einzelnen Zeichen von \lpy{s}, also \lpy{H,a,l,} \lpy{l,o, ,W,e,l} und \lpy{t}. Im zur Schleife gehörenden Block werden 
diese Zeichen dann einzeln ausgegeben.

Wenn der Codeblock für eine aufsteigende Folge von Zahlen \lpy{from, from+1, ..., to-1} ausgeführt werden soll, nimmt man die Sequenz \lpy{range(from, to)}.
Wichtig ist, dass wir das halboffene Interval betrachten, also nur bis \lpy{to-1} und nicht bis \lpy{to} gehen.
Damit verhält sich der \Python-Code \lpy{for i in range(from, to):} ganz genau wie der \CC-Code \lstinline[language=C++,style=CPPinline]|for(i = from; i < to; ++i) { ... }|.
Hier ein Beispiel
\begin{lstlisting}
sum = 0              # Setze sum auf 0
for i in range(1,5): # Nimm alle i = 1, 2, 3, 4
  sum += 3*i         # Addiere 3*i zu sum
print( sum )         # Drucke sum aus
\end{lstlisting}

Will man eine absteigende Folge ganzer Zahlen, oder allgemeiner eine Folge ganzer Zahlen mit Schrittweite \lpy{step > 0} oder \lpy{step < 0} haben,
nimmt man die Sequenz \lpy{range(from,to,step)}.
Also können wir obiges Beispiel wie folgt umformulieren.
\begin{lstlisting}
sum = 0                  # Setze sum auf 0
for i in range(3,15, 3): # Nimm alle i = 3, 6, 9, 12
  sum += i               # Addiere i zu sum
print( sum )             # Drucke sum aus
\end{lstlisting}


\subsection{Funktionen}
\label{section:crashkurs:funktionen}
Meistens will man sich wiederholende Codeblöcke auslagern und natürlich kann man auch in \Python Funktionen definieren.
Jede Funktionsdefinition erstellt ein Objekt, welches der Funktion im Speicher entspricht.
Außerdem wird eine Variable definiert, welche auf die Funktion im Speicher referenziert.
\begin{lstlisting}
def funktionsname(parameter):
  ausdruck
  ...
\end{lstlisting}
Die Variable ist hier \lpy{funktionsname} und zeigt auf das Objekt, welches der definierten Funktion entspricht.

Anders als in \CC hat eine Funktion keine Signatur und keinen definierbaren Rückgabetyp.
Mit \lpy{return} können kein, ein oder mehrere Objekte zurückgegeben werden.
Wird kein Objekt zurückgegeben, so ist die Rückgabe automatisch \lpy{None}.
Wird ein Objekt \lpy{x} zurückgegeben, so ist der Rückgabetyp automatisch \lpy{type(x)}.
Werden mehrere Objekte zurückgegeben, so werden diese automatisch in einem \lpy{tuple} zusammengefasst, der Rückgabetyp ist also \lpy{tuple}.
Ein Tuple ist eine Liste, die nicht geändert werden kann (siehe Abschnitt \ref{section:std_data_types:sequenzen}).
\begin{lstlisting}
def f(i):
  if i == 1:
    return 1           # Rueckgabetyp: int,   Rueckgabewert: 1
  elif i == 2:
    return 'Hallo', 44 # Rueckgabetyp: tuple, Rueckgabewert: ('Hallo', 44)

print( type( f ) )                 # <class 'function'>
print( type( f(1) ) )              # <class 'int'>
print( type( f(2) ) )              # <class 'tuple'>
print( type( f('Gurkenwasser') ) ) # <class 'NoneType'>
\end{lstlisting}
Also ist das von \lpy{f(3)} oder auch \lpy{f('Gurkenwasser')} zurückgegebene Objekt \lpy{None}.

Beim Funktionsausruf übergibt man die Parameter entweder in der Reihenfolge wie sie in der Funktionsdefinition spezifiziert sind, oder man übergibt \lpy{parametername=ausdruck}.
Außerdem kann man den Argumenten einer Funktion Standardwerte übergeben.
Beim Funktionsausruf muss man die Argumente mit Standardwerten nicht angeben, kann es aber.
\begin{lstlisting}
def plus(a,b=1):
  return a+b

plus(4,5)      # = 9
plus(b=5, a=4) # = 9
plus(3)        # = 4
\end{lstlisting}


\subsection{Module}
\label{section:crashkurs:module}
Was \Python wirklich mächtig macht, ist nicht die Sprache an sich, sondern die schier unendliche Fülle an Paketen die \Python-Programmierer bereit stellen.
Ein Modul ist so etwas wie eine ``shared library'' in \CC.
Man kann sie importieren und dann nutzen, die Implementierung haben andere bereits übernommen (siehe Abbildung \ref{figure:xkcd_python}).
\begin{figure}[ht]
  \centering
  \includegraphics[width=0.5\textwidth]{pictures/xkcd_python.png}
  \caption{\label{figure:xkcd_python}``Python'' by Randall Munroe \cite{Munroe_python}}
\end{figure}

Ein Modul bindet man mit \lpy{import modulname} in sein Programm ein.
Alle Objekttypen (Klassen, Funktionen und andere Objekte) die \lpy{modulname} bereitstellt, können durch \lpy{modulname.oname} erreicht werden.
\begin{lstlisting}
import math
print( math.sin(math.pi/3.0) )
\end{lstlisting}
Wenn man aus \lpy{modulname} bestimmte Objekte einbinden möchte, nutzt man folgende Codezeile.
\begin{lstlisting}
from modulname import oname1, oname2, ...
\end{lstlisting}
Jetzt kann man auf die Objekte direkt (also ohne das Präfix \lpy{modulname.}) zugreifen.
Wir hatten bereits gesehen, wie man die \PythonDrei Printfunktion in \PythonZwei einbinden kann.
\begin{lstlisting}
from __future__ import print_function
\end{lstlisting}
Hier noch ein Beispiel.
\begin{lstlisting}
from math import sin, cos, pi
print( sin(pi/3.0), cos(pi/3.0) )
\end{lstlisting}

An dieser Stelle machen wir noch auf Abschnitt \ref{section:module:empfohlene_module} indem wir eine Hand voll \Python-Pakete erwähnen, die wir oft benutzen.


\subsection{Workflow}
\label{section:crashkurs:workflow}
Wir arbeiten recht erfolgreich mit folgendem Workflow.

Beginne ein Projekt in \Python.
Solange du noch nicht fertig bist:
Wenn eine Funktion zu aufwendig zu implementieren ist, suche nach einer bereits vorhandene \Python Bibliothek (es gibt bestimmt eine).
Wenn eine Funktion für dein Projekt zu langsam ist, schreibe sie in \CC und binde sie als Modul ein.
Wie genau letzteres geht, lernen wir noch.


\newpage

\appendix
\section{Crashkurs}
\label{section:crashkurs}
Wir beginnen mit einem Überblick der grundlegende Bausteine.
In den nachfolgenden Kapiteln besprechen wir einige der Bausteine nochmal im Detail.


\subsection{Kommentare}
\label{section:crashkurs:kommentare}
Kommentare in \Python beginnen mit \lpy{#} und enden am Ende der Zeile.
Sie verhalten sich also ganz genau wie Kommentare in \CC, die mit \lcpp{//} beginnen%
\footnote{Die Kommentare \lcpp{//} wurden in \CNeunundneunzig eingeführt, siehe \cite{C99}.}.
Gut platzierte Kommentare dürfen in keinem Programm fehlen.
Ein schlecht kommentiertes Programm ist ein Anzeichen dafür, dass es von einem schlechten Programmierer geschrieben wurde.


\subsection{Print}
\label{section:crashkurs:print}
So gut wie alle Objekte können durch die \Python Funktion \lpy{print} auf der Standardausgabe ausgegeben werden.
Das liegt daran, dass die meisten Objekte Klassen sind und die Funktion \lpy{__str__} implementieren (auch wenn wir das jetzt noch nicht verstehen;
siehe Abschnitt \ref{section:klassen:spezielle_funktionen}).
Die Funktion \lpy{print} ist für \PythonZwei und \PythonDrei verschieden.
\begin{lstlisting}
print 'Hallo', 'Welt'   # Python2
print ('Hallo', 'Welt') # Python3
\end{lstlisting}
In diesem Skript nutzen wir die von \PythonDrei bereitgestellte Version von \lpy{print}.
Will man in \PythonZwei die \lpy{print}-Funktion aus \PythonDrei verwenden, kann man folgende Zeile am Anfang seines Skript einfügen.
\begin{lstlisting}
from __future__ import print_function # Nutze Python3-print in Python2
\end{lstlisting}


\subsection{Typ und Identität}
\label{section:crashkurs:typ_und_id}
Um die Identiät eines Objekts \lpy{x} zu bestimmen, nutzen wir die Funktion \lpy{id(x)}.
Sie gibt einen String zurück, den wir mit \lpy{print} ausdrucken können.
Ganz ähnlich bekommen wir den Typ eines Objekts mit \lpy{type(x)}.


\subsection{Zahlen}
\label{section:crashkurs:zahlen}
In \Python gibt es die Zahlentypen \lpy{int}, \lpy{long}, \lpy{float} und \lpy{complex}.
Sie verhalten sich (fast) genauso wie die aus \CC bekannten Typen \lcpp{int}, \lcpp{long}, \lcpp{float} und \lcpp{complex}%
\footnote{Der Typ \lpy{complex} wurde in \CNeunundneunzig eingeführt, siehe \cite{C99}.}.
Man erstellt ein Objekt vom Typ \lpy{int}, durch Angabe einer ganzen Zahl oder durch Aufrufen der Funktion \lpy{int}.
Ähnlich erstellt man ein Objekt vom Typ \lpy{float}, durch Angabe einer Kommazahl oder durch Aufrufen der Funktion \lpy{float}.
\begin{lstlisting}
print( 2 )        # 2    typ: int
print( int(2.1) ) # 2    typ: int
print( 2.0 )      # 2.0  typ: float
print( float(2) ) # 2.0  typ: float
\end{lstlisting}
Man kann zwei verschiedene oder gleiche Zahlentypen mit mathematischen Operatoren verbinden.
Das Ergebnis hat den ``genaueren'' Typ.
Des Weiteren beschreibt \lpy{//} die ganzzahlige Division ohne Rest.
\begin{lstlisting}
print( 2 + 3.3 )  # 5.3  typ: int +  float -> float
print( 5 // 2 )   # 2    typ: int // int   -> int
print( 5.3 // 2 ) # 2.0  typ: int // float -> float
\end{lstlisting}
Die Division von Ganzzahlen unterscheidet sich in \PythonZwei und \PythonDrei.
\begin{lstlisting}
4 /  2 # Python2: 2 vom Typ int;   Python3: 2.0 vom Typ float;
4 // 2 # Python2: 2 vom Typ int;   Python3: 2   vom Typ int;
\end{lstlisting}


\subsection{Strings}
\label{section:crashkurs:strings}
Strings die eine Zeile lang sind, werden entweder mit \lpy{'} oder mit \lstinline[style=PyInline]|"| begonnen und beendet.
Soll ein String mehre Zeilen lang sein, kann man ihn mit \lstinline[style=PyInline]|"""| beginnen und beenden.
Genau wie in \CC gibt es gewisse Zeichen, die nicht direkt geschrieben werden können und mit \lstinline[language=C++,style=CPP]|\| beginnen müssen (zum Beispiel werden neue Zeilen durch \lstinline[language=C++,style=CPP]|\n| ausgedrückt).
Folgender Code erstellt den String ``Hallo[Neue Zeile]Welt''.
\begin{lstlisting}
print( 'Hallo\nWelt' ) # Hallo[neue Zeile]Welt
\end{lstlisting}
Es gibt eine große Auswahl an Funktionen, um aus einem gegebenen String einen neuen zu erstellen.
Beispielsweise verknüpft man zwei Strings mit \lpy{+} und die Funktion \lpy{lower} ersetzt alle Großbuchstaben durch Kleinbuchstaben.
\begin{lstlisting}
print( 'Hallo' + ' ' + 'Welt' ) # Hallo Welt
print( 'Hallo Welt'.lower() )   # hallo welt
\end{lstlisting}


\subsection{Bedingungen}
\label{section:crashkurs:bedingungen}
In \Python gibt es genau zwei Objekte \lpy{True} und \lpy{False} von Typ \lpy{bool}.
(Fast) alle Typen können mit der Funktion \lpy{bool} in einen der beiden Objekte umgewandelt werden.
Den Kontrollfluss des Programms steuert man im einfachsten Fall wir folgt.
\begin{lstlisting}
if bedingung:
  ausdruck
  ...
\end{lstlisting}
Das heißt, wir brauchen zunächst eine Expression \lpy{bedingung}, die als \lpy{True} oder \lpy{False} interpretiert werden kann.
Wertet sie als \lpy{True} aus, so wird ein Block von Ausdrücken ausgeführt.
Anders als in \CC werden in \Python Blöcke nicht durch Klammern, sondern durch konsistente Einrückung definiert.
\begin{lstlisting}
x = 101//3+46//5
if x == 42:
  print('Die Antwort')
\end{lstlisting}
Da gut eingerückter Code (also Code, in dem jeder Block konsistent eingerückt ist) deutlich lesbarer sind als nicht eingerückter
Code, ist er in jedem Fall -- unabhängig von der Sprache -- ein Zeichen für eine gute Programmiererin. 

Da es jedoch trotzdem immer noch genug Leute gibt, die hässlichen Code produzieren, wird man in \Python dazu gezwungen, auf 
konsistente Einrückung zu achten: zu einem Block gehören genau die nachfolgenden Zeilen, die genauso weit eingerückt sind wie die 
erste Zeile des Blocks. Bei nicht konsistenter Einrückung gibt es Syntaxfehler. Dies macht es sehr schwer, hässlichen Code in
\Python zu schreiben.

Aus eigener Erfahrung können wir sagen, dass man sich als erfahrene \CC Programmiererin recht schnell an diese Konvention gewöhnen kann, 
wenn man auch schon vorher auf sauber geschriebene Programme geachtet hat.

Optional besteht eine Bedingung aus keinem oder mehreren \lpy{elif bedingung:}\footnote{sprich ``else if''} und keinem oder einem \lpy{else:}
Die Konstrukte \lpy{else:} und \lpy{elif bedingung:} in \Python verhalten sich wie die Konstrukte \lcpp{else} und \lcpp{else if} in \CC.
\begin{lstlisting}
x = 101//3+46//5               # x = ?
if x == 42:                    # Falls x den Wert 42 hat:
  print('Die Antwort')         # Drucke 'Die Antwort'
elif x == 84:                  # Sonst, falls x den Wert 84 hat:
  print('Zweimal die Antwort') # Drucke 'Zweimal die Antwort'
elif x == 106:                 # Sonst, falls x den Wert 106 hat:
  print('Dreimal die Antwort') # Drucke 'Dreimal die Antwort'
else:                          # In allen anderen Faellen:
  print('Versuchs nochmal')    # Drucke 'Versuchs nochmal'
\end{lstlisting}


\subsection{Listen}
\label{section:crashkurs:listen}
In \Python gibt es verschiedene ``Containertypen'' und hier besprechen wir den Typ \lpy{list}.
Eine Liste \lpy{thor} ist eine linear geordnete Menge von Variablen, auf die man mit \lpy{thor[0], thor[1], ..., thor[i]} zugreift.
Um auf Elemente von hinten zuzugreifen, sind negative Werte für \lpy{i} erlaubt.
\begin{lstlisting}
thor = [2, 'Hallo', 33.3, 'Welt', [] ] # Erzeugt eine Liste
print( thor[1], thor[-1] )             # Gibt folgendes aus: Hallo []
\end{lstlisting}
Die Länge einer Liste ist \lpy{len(thor)}.
Listen sind mutable, das heißt wir können sie verändern.
Mit \lpy{thor.append(x)} erweitert man die Liste \lpy{thor} um die Variable \lpy{x} und mit \lpy{thor.remove(x)} entfernt man die erste Variable mit dem Wert \lpy{x}.
Ob ein Objekt mit Wert \lpy{x} in \lpy{thor} enthalten ist, testet man mit \lpy{x in thor}.
Mit \lpy{thor+asgard} verknüpft man die beiden Listen \lpy{thor} und \lpy{asgard}.


\subsection{Schleifen}
\label{section:crashkurs:schleifen}
In \Python gibt es zwei Arten von Schleifen.
Die \lpy{while}-Schleife führt einen Codeblock aus, solange die vor der Ausführung des Blocks geprüfte Bedingung als \lpy{True} auswertet.
\begin{lstlisting}
while bedingung:
  ausdruck
  ...
\end{lstlisting}
Da uns dieses Konzept aus \CC sehr vertraut ist, brauchen wir hier kein Beispiel.

Die \lpy{for}-Schleife führt einen Codeblock für jedes Element einer gegebenen Sequenz (siehe Abschnitt \ref{section:std_data_types:sequenzen}) aus.
Eine Sequenz kann zum Beispiel eine Liste oder ein String sein.
\begin{lstlisting}
for i in s: # der Ausdruck s muss als Sequenz interpretiert werden koennen
  ausdruck
  ...
\end{lstlisting}
Dabei referenziert die Laufvariable \lpy{i} jedes Element der Sequenz \lpy{s} nacheinander einmal. Durch Verwendung von \lpy{i} innerhalb 
des folgenden Codeblocks kann 
dann auf den Wert des jeweiligen Objekts zugegriffen werden. Nach Ausführung des Codeblocks der \lpy{for}-Schleife referenziert 
\lpy{i} dann das nächste Element von \lpy{s}, bis das Ende der Sequenz erreicht ist. 

Dies wird durch das folgende Beispiel veranschaulicht:
\begin{lstlisting}
s = 'Hallo Welt' # Die Sequenz 'Hallo Welt'
for i in s:      # Nimm alle i = H,a,l,l,o, ,W,e,l,t
  print(i)       # Drucke i aus
\end{lstlisting}
Hier ist unsere Sequenz \lpy{s} ein String mit dem Wert \lpy{Hallo Welt}. In der \lpy{for}-Schleife nimmt die Laufvariable nacheinander
die Werte der einzelnen Zeichen von \lpy{s}, also \lpy{H,a,l,} \lpy{l,o, ,W,e,l} und \lpy{t}. Im zur Schleife gehörenden Block werden 
diese Zeichen dann einzeln ausgegeben.

Wenn der Codeblock für eine aufsteigende Folge von Zahlen \lpy{from, from+1, ..., to-1} ausgeführt werden soll, nimmt man die Sequenz \lpy{range(from, to)}.
Wichtig ist, dass wir das halboffene Interval betrachten, also nur bis \lpy{to-1} und nicht bis \lpy{to} gehen.
Damit verhält sich der \Python-Code \lpy{for i in range(from, to):} ganz genau wie der \CC-Code \lstinline[language=C++,style=CPPinline]|for(i = from; i < to; ++i) { ... }|.
Hier ein Beispiel
\begin{lstlisting}
sum = 0              # Setze sum auf 0
for i in range(1,5): # Nimm alle i = 1, 2, 3, 4
  sum += 3*i         # Addiere 3*i zu sum
print( sum )         # Drucke sum aus
\end{lstlisting}

Will man eine absteigende Folge ganzer Zahlen, oder allgemeiner eine Folge ganzer Zahlen mit Schrittweite \lpy{step > 0} oder \lpy{step < 0} haben,
nimmt man die Sequenz \lpy{range(from,to,step)}.
Also können wir obiges Beispiel wie folgt umformulieren.
\begin{lstlisting}
sum = 0                  # Setze sum auf 0
for i in range(3,15, 3): # Nimm alle i = 3, 6, 9, 12
  sum += i               # Addiere i zu sum
print( sum )             # Drucke sum aus
\end{lstlisting}


\subsection{Funktionen}
\label{section:crashkurs:funktionen}
Meistens will man sich wiederholende Codeblöcke auslagern und natürlich kann man auch in \Python Funktionen definieren.
Jede Funktionsdefinition erstellt ein Objekt, welches der Funktion im Speicher entspricht.
Außerdem wird eine Variable definiert, welche auf die Funktion im Speicher referenziert.
\begin{lstlisting}
def funktionsname(parameter):
  ausdruck
  ...
\end{lstlisting}
Die Variable ist hier \lpy{funktionsname} und zeigt auf das Objekt, welches der definierten Funktion entspricht.

Anders als in \CC hat eine Funktion keine Signatur und keinen definierbaren Rückgabetyp.
Mit \lpy{return} können kein, ein oder mehrere Objekte zurückgegeben werden.
Wird kein Objekt zurückgegeben, so ist die Rückgabe automatisch \lpy{None}.
Wird ein Objekt \lpy{x} zurückgegeben, so ist der Rückgabetyp automatisch \lpy{type(x)}.
Werden mehrere Objekte zurückgegeben, so werden diese automatisch in einem \lpy{tuple} zusammengefasst, der Rückgabetyp ist also \lpy{tuple}.
Ein Tuple ist eine Liste, die nicht geändert werden kann (siehe Abschnitt \ref{section:std_data_types:sequenzen}).
\begin{lstlisting}
def f(i):
  if i == 1:
    return 1           # Rueckgabetyp: int,   Rueckgabewert: 1
  elif i == 2:
    return 'Hallo', 44 # Rueckgabetyp: tuple, Rueckgabewert: ('Hallo', 44)

print( type( f ) )                 # <class 'function'>
print( type( f(1) ) )              # <class 'int'>
print( type( f(2) ) )              # <class 'tuple'>
print( type( f('Gurkenwasser') ) ) # <class 'NoneType'>
\end{lstlisting}
Also ist das von \lpy{f(3)} oder auch \lpy{f('Gurkenwasser')} zurückgegebene Objekt \lpy{None}.

Beim Funktionsausruf übergibt man die Parameter entweder in der Reihenfolge wie sie in der Funktionsdefinition spezifiziert sind, oder man übergibt \lpy{parametername=ausdruck}.
Außerdem kann man den Argumenten einer Funktion Standardwerte übergeben.
Beim Funktionsausruf muss man die Argumente mit Standardwerten nicht angeben, kann es aber.
\begin{lstlisting}
def plus(a,b=1):
  return a+b

plus(4,5)      # = 9
plus(b=5, a=4) # = 9
plus(3)        # = 4
\end{lstlisting}


\subsection{Module}
\label{section:crashkurs:module}
Was \Python wirklich mächtig macht, ist nicht die Sprache an sich, sondern die schier unendliche Fülle an Paketen die \Python-Programmierer bereit stellen.
Ein Modul ist so etwas wie eine ``shared library'' in \CC.
Man kann sie importieren und dann nutzen, die Implementierung haben andere bereits übernommen (siehe Abbildung \ref{figure:xkcd_python}).
\begin{figure}[ht]
  \centering
  \includegraphics[width=0.5\textwidth]{pictures/xkcd_python.png}
  \caption{\label{figure:xkcd_python}``Python'' by Randall Munroe \cite{Munroe_python}}
\end{figure}

Ein Modul bindet man mit \lpy{import modulname} in sein Programm ein.
Alle Objekttypen (Klassen, Funktionen und andere Objekte) die \lpy{modulname} bereitstellt, können durch \lpy{modulname.oname} erreicht werden.
\begin{lstlisting}
import math
print( math.sin(math.pi/3.0) )
\end{lstlisting}
Wenn man aus \lpy{modulname} bestimmte Objekte einbinden möchte, nutzt man folgende Codezeile.
\begin{lstlisting}
from modulname import oname1, oname2, ...
\end{lstlisting}
Jetzt kann man auf die Objekte direkt (also ohne das Präfix \lpy{modulname.}) zugreifen.
Wir hatten bereits gesehen, wie man die \PythonDrei Printfunktion in \PythonZwei einbinden kann.
\begin{lstlisting}
from __future__ import print_function
\end{lstlisting}
Hier noch ein Beispiel.
\begin{lstlisting}
from math import sin, cos, pi
print( sin(pi/3.0), cos(pi/3.0) )
\end{lstlisting}

An dieser Stelle machen wir noch auf Abschnitt \ref{section:module:empfohlene_module} indem wir eine Hand voll \Python-Pakete erwähnen, die wir oft benutzen.


\subsection{Workflow}
\label{section:crashkurs:workflow}
Wir arbeiten recht erfolgreich mit folgendem Workflow.

Beginne ein Projekt in \Python.
Solange du noch nicht fertig bist:
Wenn eine Funktion zu aufwendig zu implementieren ist, suche nach einer bereits vorhandene \Python Bibliothek (es gibt bestimmt eine).
Wenn eine Funktion für dein Projekt zu langsam ist, schreibe sie in \CC und binde sie als Modul ein.
Wie genau letzteres geht, lernen wir noch.


\newpage

% Bibliography
% We use our own bibstyle which is essential alpha extended by the fields 'eprint', 'archivePrefix' and 'primaryClass'.
%This is recommended by the arXiv (see https://arxiv.org/hypertex/bibstyles/)
\phantomsection                           % Braucht man damit hyperref das Literaturvezeichniss findet.
\addcontentsline{toc}{section}{Literatur} % Fuege Literaturvezeichniss zum Inhaltsverzeichnis hinzu.
\bibliographystyle{alphaplus}
\bibliography{bib_joelix}

\end{document}
