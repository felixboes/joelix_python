\section{Installation}
\label{section:installation}
In diesem Abschnitt besprechen wir ganz kurz wie man \PythonDrei und einen von uns bevorzugen Editor installiert.

\subsection{Python3 installieren}
Nutzt man eine auf Debian basierende GNU/Linux Distribution (so wie Debian, Ubuntu oder Mint), installiert man \PythonDrei zum Beispiel so:
\begin{lstlisting}[language=bash]
sudo apt-get install python3
\end{lstlisting}

Nutzt man Mac OS X gibt es zwei Möglichkeiten.
Ab Mac OSX 10.8 ist \PythonZwei bereits installiert und das reicht für unsere Zwecke eigentlich aus.
Um \PythonDrei zu nutzen, lädt man \PythonDrei von der offiziellen Webseite \cite{PythonWebseite} herunter und installiert es.

Nutzt man Windows, lädt man \PythonDrei von der offiziellen Webseite \cite{PythonWebseite} herunter und installiert es.
Wichtig ist hierbei, dass man sich vom Installationsprogramm die PATH-Variable automatisch setzen lässt.

\subsection{PyCharm Community Edition installieren}
Als IDE (Integrated Developer Environment) verwenden wir PyCharm Community Edition 2016.3.
Sowohl bei GNU/Linux als auch bei Mac OS X und auch bei Windows lädt man die ``Community Edition'' von PyCharm Webseite \cite{pycharm} herunter (und nicht die ``Professional Edition'').
Diese wird dann entpackt beziehungsweise installiert.
