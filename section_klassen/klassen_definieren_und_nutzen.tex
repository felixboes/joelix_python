\subsection{Klassen definieren und nutzen}
\label{section:klassen:klassen_definieren_und_nutzen}
Eine Klasse definiert man \Python wie folgt:
\begin{lstlisting}
class klassenname:
  def __init__(self, weitere_parameter):
    self.membervariable_1 = ...
    self.membervariable_2 = ...
    ...
  
  def memberfunktion_1(self, weitere_parameter):
    ...
  
  def memberfunktion_2(self, weitere_parameter):
    ...
  
  ...
  
  def __spezielle_funktion_1__(self, weitere_parameter):
    ...
  
  ...
\end{lstlisting}
Der Klassenname kann so wie alle Variablenamen fast frei gewählt werden.
Variablen die zur Klasse gehören heißen ``Membervariablen''.
Mit dem Präfix \lpy{self.} greift man auf sie {\bfseries innerhalb der Klasse} zu (aber nicht von außen).
Funktionen die zur Klasse gehören heißen ``Memberfunktionen''.
Ihr erstes Argument muss bei ihrer {\bfseries Definition innerhalb der Klasse} \lpy{self} sein, beim Zugriff wird der erste Prameter \lpy{self} nicht angegeben.

Es gibt zwei Sorten von Memberfunktionen:
sogenannte ``spezielle Funktionen'' die mit \lpy{__} beginnen und enden und
``normale Funktionen'', die nicht mit \lpy{__} beginnen und enden.
Mehr zu spezielle Funktionen besprechen wir in Abschnitt \ref{section:klassen:spezielle_funktionen}

Die prominenteste spezielle Funktion ist der sogenannte ``Konstruktor'' oder ``Initialisierungsfunktion'' mit dem Namen \lpy{__init__}.
Sie wird beim Erstellen eines Objekts automatisch aufgerufen.
Außerdem kann man beim Erstellen eines Objekts gewisse Parameter übergeben.
Bei unserem Matrixbeispiel würde es Sinn machen, die Zeilen- und Spaltengröße zu übergeben.
In der Initialisierungsfunktion sollen alle Membervariablen definiert und (in Abhängigkeit von den übergebenen Parametern) in einen sinnvollen Ausganszustand gebracht werden.
Man darf in der Initialisierungsfunktion auch andere Memberfunktionen aufrufen.
Hat der Konstruktor der Klasse \lpy{klassenname} die Parameter \lpy{p_1, ..., p_k},
so erstellt man ein Objekt vom Typ \lpy{klassenname} zu den Werten \lpy{par_1, ..., par_k} wie folgt.
\begin{lstlisting}
# Erstellt ein Objekt vom Typ klassenname bzgl. par_1, ..., par_k.
# Die Variable var referenziert auf dieses Objekt.
var = klassenname(par_1, ..., par_k)
\end{lstlisting}
Man sagt ``das Objekt \lpy{obj} ist eine Instanz der Klasse \lpy{klassenname}'' wenn der Typ des Objekts \lpy{obj} die Klasse \lpy{klassenname} ist.

Zeigt eine Variable \lpy{var} auf ein Instanz von \lpy{klassenname}, so kann man auf die Membervariablen und Memberfunktionen der Instanz mithilfe des Präfixes \lpy{var.} zugreifen.
Zeigt beispielsweise die Variable \lpy{var} auf ein Objekt dass eine Memberfunktion mit Namen \lpy{python_ist_cool},
dann greifen wir auf diese Memberfunktion mit \lpy{var.python_ist_cool()} zu.
