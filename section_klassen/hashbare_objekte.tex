\subsection{Hashbare Objekte}
\label{section:klassen:hashbare_objekte}
Wir erklären hier, was hashbare Objekte sind und wie man selbstdefinierte Klassen hashbar macht.

Hashfunktionen und Hashtables wurden im Fortgeschrittenen Programmierkurs (siehe \cite{joelixC}) bereits erklärt, deshalb halten wir uns hier kurz.
Für einen festgelegten Typ \lpy{T} nennen wir Menge aller Objekte diesen Typs $Set_T$.
Die Menge der \Python-Ganzzahlen nennen wir hier $Int$.
Man sagt dass
\[
  hash \colon Set_T \to Int
\]
eine ``zulässige Hashfunktion'' ist, wenn für je zwei Objekte \lpy{x} und \lpy{y} aus $Set_T$ für die \lpy{x == y} als \lpy{True} auswertet auch \lpy{hash(x) == hash(y)} als \lpy{True} auswertet.
Desweiteren muss bei mutable Objekten gewähleistet sein, dass beim Ändern des Werts eines jeden Objekts \lpy{x} die Ausdücke \lpy{x == y} und \lpy{hash(x) == hash(y)} unabhängig von der Änderung sind.
Man sagt, dass ein Objekt oder ein Typ \lpy{T} ``hashbar'' ist, wenn eine zulässige Hashfunktion für $Set_T$ definiert ist.
Der \Python-Standard hält für die dort definierten immutable Typen (also Objekte vom Typ \lpy{int}, \lpy{float}, \lpy{complex}, \lpy{tuple}, \lpy{str}, \lpy{bool} und \lpy{none},
aber nicht \lpy{list} und \lpy{dict}) zulässige Hashfunktionen bereit.
Demnach sind diese Typen hashbar.

Um eine selbstgeschriebene Klasse \lpy{K} hashbar zu machen, müssen wir die beiden spezielle Funktion \lpy{__eq__} und \lpy{__hash__} implementieren.
Dabei muss (wir oben gefordert) für je zwei Instanzen \lpy{x} und \lpy{y} von \lpy{K} für die \lpy{x == y} als \lpy{True} auswertet, der Wert von \lpy{x.__hash__()} und \lpy{y.__hash__()} übereinstimmen.
Außerdem verlangen wir, dass, falls das Objekt (durch Memberfunktionen) verändert werden kann, so soll \lpy{__eq__} und \lpy{__hash__} unabhängig von dieser Änderung sein.

Hier ein ganz simples Beispiel.
Wir beschreiben eine Klasse mit dem Namen \lpy{Geldboerse}, inder man 1-Euro Münzen und 1-Cent Münzen speichern kann.
Die Anzahl der Münzen soll nach dem Erstellen nicht mehr geändert werden.
Zwei Geldbörsen sind genau dann gleich, wenn sie dieselbe Anzahl von 1-Euro und 1-Cent Münzen enthalten.
Als Hashfunktion nehmen wir die Anzahl der 1-Euro Münzen modulo 100.
Wir bemerken, dass das eine zulässige Hashfunktion ist.
Der Code sieht dann so aus:
\begin{lstlisting}
class Geldboerse:
  """Eine sehr naive Klasse die eine Geldboerse darstellt."""
  def __init__(self, euro, cent):
    self._euro = euro
    self._cent = cent
  
  def __str__(self): # Macht Geldboerse als String interpretierbar
    return 'Euro: {}, Cent: {}'.format(self._euro, self._cent)
  
  def __eq__(self, other): # Implementiert x == y
    if self._euro == other._euro and self._cent == other._cent:
      return True
    else:
      return False
  
  def __hash__(self): # Eine sehr naive Hashfunktion
    return self._euro % 100
\end{lstlisting}

Wenn wir nun doch zulassen, dass man die Anzahl der Münzen geändert werden kann,
so wäre unsere oben angegebene Hashfunktion nicht mehr zulässig.
Es ist also gar nicht so leicht, eine zulässige Hashfunktion für beliebige Klassen zu schreiben und in der Tat,
man will meistens nur solche Objekte hashbar machen, die sich nach dem Erstellen nur unwesentlich oder garnicht mehr verändern.

