\subsection{Eine sehr naive Matrixklasse}
\label{section:klassen:eine_sehr_naive_matrixklasse}
Als einführendes Beispiel definieren wir eine sehr naive Matrixklasse.
Dabei ver\-nach\-läs\-sig\-en wir alle Plausibilitätsprüfungen
(zum Beispiel prüfen wir nicht, ob der Typ von \lpy{zeilen} wirklich \lpy{int} ist).
Bei einer gut geschriebenen Klasse dürfen diese Plausibilitätsprüfungen natürlich nicht fehlen.
\begin{lstlisting}
class Matrix:
  """Eine sehr naive Matrixklasse"""
  
  def __init__(self, zeilen=0, spalten=0):
    self.zeilen = zeilen
    self.spalten = spalten
    self.elemente = self.zeilen*self.spalten*[0.0]
  
  def getitem(self, i, j):
    """Gibt den Koeffizienten der Matrix in der i-ten Zeile und
       der j-ten Spalte zurueck."""
    return self.elemente[i*self.spalten + j]
  
  def setitem(self, i, j ,z):
    """Setzt den Koeffizienten der Matrix in der i-ten Zeile und
       der j-ten Spalte auf den Wert z."""
    self.elemente[i*self.spalten + j] = z
  
  def string(self):
    """Gibt einen String zurueck, der die Matrix beschreibt."""
    s = "Zeilen: {} Spalten: {}\n".format(self.zeilen, self.spalten)
    for i in range(self.zeilen):
      for j in range(self.spalten):
        s = s + "{}; ".format(self.getitem(i,j))
      s = s + "\n"
    return s
\end{lstlisting}
Mit dieser sehr naiven Definition, erstellt man eine Matrix mit drei Zeilen und vier Spalten wie folgt:
\begin{lstlisting}
m = Matrix(zeilen=3, spalten=4)
\end{lstlisting}
Auf die Memberfunktionen greift man mit dem Präfix \lpy{m.} zu:
\begin{lstlisting}
m.setitem(1,1,3.0) # Setzt den Eintrag in Zeile 1 und Spalte 1 auf 3.0.
print(m.string())  # Druckt die Darstellung der Matrix aus.
\end{lstlisting}
