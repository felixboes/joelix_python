\subsection{Verzeichnisse}
\label{section:std_data_types:verzeichnisse}
Objekte von Typ \lpy{dict} nennt man ``Verzeichnisse''.
Ein Verzeichnis verhält sich wie eine partiell definierte mathematische Funktion,
deren Definitionsbereich die Menge der ``hashbaren Objekte'' ist und deren Wertebereich die Menge aller Objekte ist.
In anderen Worten:
Genau wie partiell definierte Funktionen, kann man sich ein Verzeichnis \lpy{d} wie eine Menge von Tupeln \lpy{(key, d[key])} vorstellen,
wobei die linke Seite eines jeden Tupels höchstens einmal als linke Seite vorkommt.
Wenn man bei dieser Vorstellung bleibt, sagt man dass \lpy{d} auf \lpy{key} definiert ist, wenn \lpy{d} das Tupel \lpy{(key, d[key])} enthält.
Diejenigen Objekte, auf denen ein Verzeichnis definiert ist, nennt man ``Schlüssel''.
Schlüssel sind immer hashbare Objekte (siehe unten).
Ist \lpy{key} ein Schlüssel einer Liste \lpy{d}, so ist \lpy{d[key]} ein ``Wert'' des Verzeichnisses \lpy{d}.

Noch haben wir nicht erklärt, was hashbare Objekte sind.
Hier sei gesagt, dass alle vom \Python Standard definierten Objekte hashbar sind, sofern sie immutable sind
(also Objekte vom Typ \lpy{int}, \lpy{float}, \lpy{complex}, \lpy{tuple}, \lpy{str}, \lpy{bool} und \lpy{none},
aber nicht \lpy{list} und \lpy{dict}).
Wir erklären hashbare Objekte in Abschnitt \ref{section:klassen:hashbare_objekte}.


\subsubsection{Verzeichnisse erstellen}
\label{section:std_data_types:verzeichnisse:verzeichnisse_erstellen}
Verzeichnisse können auf folgende Arten erstellt werden:
\begin{lstlisting}
# Leere Verzeichnisse
verz1 = {}
print(verz1)
verz2 = dict()
print(verz2)

# Nichtleeres Verzeichnis
verz3 = { key1 : value1, key2 : value2, key3 : value3, ... }
print(verz3)
\end{lstlisting}
\lpy{verz1} und \lpy{verz2} sind leere Verzeichnisse, \lpy{verz3} verdeutlicht die Syntax eines nichtleeren Verzeichnisses. Die 
``Schlüssel''-``Wert''-Paare werden durch Doppelpunkte getrennt, die Paare voneinander durch Kommata. Ein Beispiel für ein nichtleeres 
Verzeichnis könnte folgendermaßen aussehen:
\begin{lstlisting}
#Verzeichnis speichert Telefonnummern:
telefonbuch = {'Mama':333, 'Papa':333, 'Tina':11, 'Peter':'Keine Nr'}
\end{lstlisting}

Außerdem kann man ein Verzeichnis aus einer Liste von Tupeln erstellen (solange die linke Seite eines jeden Tupels ein hashbares Objekt ist).
Hier ein Beispiel:
\begin{lstlisting}
verz_eins = { 42 : 'Wahrheit', True : 'Wahrheit' }
verz_zwei = dict([(42, 'Wahrheit'), (True, 'Wahrheit')])
print(verz_eins, verz_zwei)
\end{lstlisting}


\subsubsection{Verzeichnisse nutzen}
\label{section:std_data_types:verzeichnisse:verzeichnisse_nutzen}
Mit den folgenden Funktionen, kann man sich die Größe eines Verzeichnisses \lpy{d} ausgeben lassen,
auf Werte zugreifen, sowie (neue) Schlüssel-Wert Paare setzen oder löschen.
\begin{lstlisting}
len(d)        # Groesse des Verzeichnisses
d[key]        # Gibt den Wert d[key] zurueck oder loest eine Ausnahme aus
d.get(key, x) # Gibt den Wert d[key] zurueck falls er existiert. Sonst
              # wird das Paar (key, x) erstellt und x zurueckgegeben.
d[key] = x    # Setzt d[key] auf x (auch wenn key noch nicht existiert)
del d[key]    # Entfernt (key, d[key]) oder loest eine Ausnahme aus
d.pop(key, x) # Entfernt (key, d[key]) und gibt d[key] zurueck falls es
              # existiert. Sonst wird x zurueckgegeben.
d.popitem()   # Entfernt bel. Paar (key, d[key]) und gibt d[key] zurueck
d.clear()     # Entfernt alle Paare aus dem Verzeichniss.
\end{lstlisting}
Um nachzuschauen, ob ein Schlüssel in einem Verzeichniss \lpy{d} definiert ist oder nicht nutzen wir die folgenden Operationen.
\begin{lstlisting}
key in d      # Gibt True  zurueck gdw key ein Schluessel von d ist
key not in d  # Gibt False zurueck gdw key ein Schluessel von d ist
\end{lstlisting}
Eine oberflächeliche Kopie eines Verzeichnisses \lpy{d} zu erstellen, benutzt man die folgende Operation.
\begin{lstlisting}
d.copy()
\end{lstlisting}
Die Konzepte der oberflächlichen und tiefen Kopien sowie die daraus resultierenden Konsequenzen erklären wir in Abschnitt \ref{section:std_data_types:sequenzen:kopien} am Beispiel der Listen.
Analoge Aussagen gelten für Verzeichnisse.

An dieser Stelle gehen wir noch ein wenig auf die Gleichheit von Verzeichnissen ein.
Der Vergleich \lpy{s == t} wertet genau dann als \lpy{True} aus, wenn die Paare von \lpy{s} und \lpy{t} übereinstimmen.
Das heißt aber nicht, dass \lpy{s} und \lpy{t} dieselbe Identität haben, also an der selben Stelle im Speicher stehen.
Hier ein Beispiel.
\begin{lstlisting}
s = {1: 'Eins'}
t = {1: 'Eins'}
print( s == t )          # True
print( id(s) )           # 140601528318536
print( id(t) )           # 140601528344776
\end{lstlisting}

Um eine Sequenz der Schlüsse, der Werte oder der Schlüssel-Wert-Paare eines gegebenen Verzeichnisses \lpy{d} zu erhalten, nutzen wir die folgenden Operationen.
\begin{lstlisting}
d.keys()   # Gibt eine Sequenz der Schluessel zurueck
d.values() # Gibt eine Sequenz der Werte zurueck
d.items()  # Gibt eine Sequenz der Schluessel-Wert-Paare zurueck
\end{lstlisting}
Zum Beispiel druckt man die Schluessel-Wert-Paare eines Verzeichnisses \lpy{d} so aus:
\begin{lstlisting}
d = { 42 : 'Wahrheit', True : 'Wahrheit' }

for key,val in d.items():
  print("Schluessel: '{}', Wert: '{}'".format(key, val))
\end{lstlisting}
