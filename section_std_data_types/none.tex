\subsection{None}
\label{section:std_data_types:none}
Es gibt genau ein Objekt \lpy{None} vom sogenannte Typ \lpy{NoneType}.
Man benutzt dieses Objekt um die Abwesenheit eines Werts oder eines Parameters auszudrücken.
Zum Beispiel ist \lpy{None} der Rückgabetyp einer Funktion, die zurückkehrt ohne ein Objekt mit \lpy{return} zurückzugeben.
Hier ein Beispiel:
\begin{lstlisting}
def ich_geb_nix_zurueck():
  print("Ich hab nix")

x = ich_geb_nix_zurueck()
print( type(x) )
\end{lstlisting}

Ganz oft wird \lpy{None} als Standardparameter für optionale Parameter einer Funktion festgelegt.
Damit lässt sich mit ganz leicht prüfen, ob der Benutzer einen optionalen Parameter übergeben hat oder nicht.
Hier ein simples Beispiel:
\begin{lstlisting}
def sprich_falls_du_was_zu_sagen_hast( was=None ):
  if was is None: # Behandle als ob nichts uebergeben wurde
    print("Ich hab nix zu sagen!")
  else:           # Behandle als ob etwas uebergeben wurde
    print("Ich hab was zu sagen: '{}'.".format(was))

sprich_falls_du_was_zu_sagen_hast()     # Ich hab nix zu sagen!
sprich_falls_du_was_zu_sagen_hast(None) # Ich hab nix zu sagen!
sprich_falls_du_was_zu_sagen_hast(42)   # Ich hab was zu sagen: '42'.
\end{lstlisting}

